\chapter{State of the art}\label{sec:sota}

% TODO: creare una sezione related work alla fine
% spostare parte corposa dello "stato dell'arte" in fondo
% lasciare solo intuizioni e robe di base

% In this section, I provide an overview of existing research on the 
% implementability problem, 
% a topic primarily investigated in Message Sequence 
% Charts (MSC), Choreographies, and Multiparty Session Types (MPST).

\section{Message Sequence Charts}
Message Sequence Charts (MSCs) are a standardized graphical formalism,
introduced in 1992 \cite{MSCStandard}, used to describe trace languages for specifying
communication behavior. Thanks to their simplicity and intuitive
semantics, MSCs have been widely adopted in industry.
Figure~\ref{fig:msc-cli-ser} illustrates a simple example based on a
minimal client–server architecture. An extension of this formalism,
known as High-Level Message Sequence Charts (HMSCs), was later
introduced \cite{HMSCStandard}. HMSCs enable the definition of
MSCs as nodes connected by transitions and are used to model more
complex patterns of message flows by capturing sequences, alternatives,
or iterations of atomic MSC scenarios.

\begin{figure}[!ht]
\centering
\begin{msc}[draw frame=none, draw head=none, msc keyword=, head height=0px, label distance=0.5ex, foot height=0px, foot distance=0px]{}
	\declinst{P1}{Client}{}
	\declinst{P2}{Server}{}

	\mess{request}{P1}{P2}
	\nextlevel
	\mess{answer}{P2}{P1}
\end{msc}
\caption{Simple example of a client-server architecture.}
\label{fig:msc-cli-ser}
\end{figure}

% TODO: rimanere ad alto livello
% ma aggiungere reference alle definizioni formali 
% nella parte di definizioni formali fare un confronto.
% citare il confronto qua e rimandare dopo

The \emph{realizability problem} was first introduced for MSC languages 
in~\cite{alur2000inference,alur2003inference}. It asks whether there exists a distributed 
implementation that can realize all behaviors of a finite set of MSCs 
without introducing additional ones. A stronger variant, called 
\emph{safe realizability}, requires the implementation to also be \textbf{deadlock-free}. 
% In this setting, distributed implementations are defined using communicating 
% automata with explicit accepting states, and deadlock-freedom corresponds 
% to the ability to reach an accepting state from any reachable configuration. 
For finite sets of MSCs, weakly realizability is \verb|coNP|-complete and safe 
realizability is shown to be decidable in P-time~\cite{alur2005realizability}.
The problem was subsequently studied for infinite MSC languages, defined 
as MSC-Graphs (MSGs). For bounded MSGs, safe realizability 
remains decidable, but weak realizability 
is undecidable~\cite{alur2005realizability}. Extensions of these results to non-FIFO 
semantics were investigated in~\cite{morin2002recognizable}, corresponding 
to bag semantics under peer-to-peer communication. 
Later, Lohrey proved that in the general case, safe realizability 
is undecidable~\cite{lohrey2003realizability}, though it is decidable (and 
EXPSPACE-complete) for globally cooperative MSGs.
Most positive results assume bounded channels, but \cite{bollig2025high} introduces 
a new class of HMSCs that allows unbounded channels while maintaining implementability.
A summary of the main complexity results is given in Table~\ref{tab:realizability}.

\begin{table}[!ht]
	\centering
	\begin{tabular}{|l|c|c|c|}
		\hline
		& \textbf{Finite set} & \textbf{Bounded graphs} & \textbf{Unbounded} \\
		\hline
		\textbf{Weak} & coNP-complete & undecidable & undecidable \\
		\hline
		\textbf{Safe} & P-time & EXPSPACE-complete & undecidable \\
		\hline
	\end{tabular}
	\caption{Summary of results on realizability.}
	\label{tab:realizability}
\end{table}

% TODO: CONFRONTO di def implementability
% ALUR/LOHREY e STUTZ (sender-driven) e EMILIO (da riguardare)
% con esempi

%% Sezione da rifare meglio

% TODO: dal nostro punto di vista un global type è molto simile a una
% coreografia, quindi riguardare e riscrivere meglio. Ci sta parlare 
% di coreografie, ma prima penso a MPST.

% \section{Choreographies}
% Choreographies \cite{montesi2014choreographic} are another formalism to describe  
% distributed communication protocols. Unlike MSCs or MPST, which focus either on 
% trace-based semantics or type systems, choreographies emphasize the 
% global specification of interactions as a high-level description of the 
% intended message exchanges. Their goal is to ensure that a distributed 
% implementation can be derived in which each participant follows a local 
% behavior consistent with the global description. This setting naturally 
% connects to the realizability problem, since the key question is whether 
% a choreography can be faithfully implemented by a system of local 
% processes.

% Early work~\cite{fu2004conversation} introduced \emph{synchronizability} 
% as a criterion for checking realizability of choreographies. Later work 
% by Basu and Bultan~\cite{basu2011choreography} refined this notion and 
% established connections between synchronizability and decidability of 
% verification tasks. However, synchronizability itself was shown to be 
% undecidable in general~\cite{finkel2023synchronizability}, limiting its 
% applicability.  
% An alternative line of work, pioneered by Barbanera et 
% al.~\cite{barbanera2020choreography,barbanera2022formal}, introduced 
% syntactic well-formedness conditions on choreographies. Well-formed 
% choreographies are guaranteed to be both safe and realizable, since 
% their projection onto communicating automata preserves language 
% equivalence and ensures deadlock-freedom.


\section{Multiparty Session Types}
Multiparty Session Types (MPST)~\cite{honda2008multiparty} 
provide a type-theoretic framework to specify and verify communication 
protocols among multiple participants. They ensure that communication 
follows a predefined structure, preventing errors such as deadlocks, 
orphan messages, and unspecified receptions. The 
\textbf{global specification} describes the overall communication 
protocol. From this, one derives the \textbf{local behaviors} of each 
participant via a \emph{projection} operation. The system's 
\textbf{processes} form the \emph{implementation}, defining how 
participants interact. With the definition of a \emph{typing system} 
and suitable \emph{type-checking rules}, one ensures that the 
implementation conforms to the local specification, thereby 
guaranteeing properties such as \emph{well-formedness}.  

% In this setting, the analogue of realizability is often called 
% \emph{session fidelity}: a property ensuring that the combined local 
% types behave exactly according to the global type. When the global 
% type satisfies syntactic restrictions, its projection is guaranteed 
% to be both realizable and deadlock-free, which corresponds to safe 
% realizability in the MSC setting.  

\subsection{Projectability}
A central notion in MPST is \emph{projectability}, which asks whether 
a global type can be faithfully projected into local specifications for 
each participant. If projection succeeds, the resulting local types 
interact without mismatches or unintended behaviors, effectively 
bridging global specifications and distributed implementations~\cite{honda2008multiparty}.  
Projection algorithms, however, often reject natural protocols that 
fail to meet restrictive syntactic conditions. This tension between 
expressivity and safety has motivated extensions of the theory, with 
\cite{castagna2012global} being the only algorithm aiming for full 
completeness.

Recent work has focused on strengthening the connection between MPST 
and automata-theoretic formalisms. Stutz and Zufferey 
showed that implementability is decidable by encoding global types 
into HMSCs that are globally cooperative~\cite{stutz2022comparing,stutz2023asynchronous}. 
Building on this, Li et al.~\cite{li2023complete} proposed a complete 
projection function for MPST, guaranteeing that every implementable 
global type admits a correct distributed implementation.

\subsection{Mixed and Sender driven choice}
A key restriction appears in branching. In the original framework \cite{honda2008multiparty,carbone2012structured}, 
choice is \textbf{sender-driven}: the first sender dictates the branch, 
ensuring safety but excluding many common patterns where multiple 
participants influence the decision~\cite{carbone2012structured}.  
Allowing \textbf{mixed choice} increases expressivity by permitting 
several initiators, but it also makes the implementability problem 
undecidable in general~\cite{barbanera2020choreography}.  

\paragraph{Generalized projection operators}
Stutz’s thesis~\cite{stutz2024implementability} connects MPST to 
High-level MSCs (HMSCs), introducing a generalized projection operator 
for sender-driven choice where a sender may branch towards different 
receivers. This captures patterns beyond classical MPST projection.  
He also proves that while syntactic projection is incomplete, 
an automata-theoretic encoding into HMSCs yields decidability for 
sender-driven choice, with implementability shown to be in 
\verb|PSPACE|-the first precise complexity bound for this fragment.

\paragraph{Protocol State Machines}
Later, \cite{stutz2025automata} proposed \emph{Protocol State Machines} 
(PSMs), an automata-based formalism subsuming both MPST and HMSCs. 
PSMs show that many syntactic restrictions of global types are not 
true expressivity limits. Yet, the implementability problem for PSMs 
with unrestricted mixed choice remains undecidable, resolving the 
open question that mixed-choice global types are undecidable in general.  

In summary, projectability is well understood for sender-driven choice, 
where decidability and complexity bounds are established, but moving 
towards mixed choice inevitably leads to undecidability. Automata-based 
techniques such as HMSCs and PSMs provide the most powerful tools for 
extending the theory while preserving decidability in restricted cases.

\section{Reduction to synchronous semantic}
The main idea of this work is that reasoning about implementability 
becomes more tractable under \emph{synchronous} 
semantics for automata-based solutions to the implementability problem. 
In synchronous communication, send and receive actions 
are tightly coupled, effectively removing nondeterminism 
caused by asynchronous message buffering. Several results exploit this 
observation by reducing the implementability problem under richer 
communication models (e.g.\ asynchronous or peer-to-peer FIFO) to the 
simpler synchronous case~\cite{alur2005realizability,di2023partial}.

Formally, one can show that if a global type is implementable in 
synchronous semantics, then under certain conditions it is also 
implementable in more general models such as peer-to-peer or mailbox 
semantics. This reduction requires constraints such as 
\emph{orphan-freedom} (no message is left unmatched) and
\emph{deadlock-freedom}.
The following theorem, currently a work in progress by my supervisors \cite{di2025realisability}, 
provides a characterization of a connection between 
peer-to-peer semantics and synchronous semantics:

\begin{theorem}
	A global type $G$ is implementable in \verb|p2p| iff:
	\begin{enumerate}
		\item $L_{\text{p2p}}(proj(G))$ is a set of sync MSCs;
		\item $proj(G)$ is orphan-free in p2p;
		\item $L_{\text{p2p}}(proj(G))$ is deadlock-free;
		\item $G$ is implementable in sync.
	\end{enumerate}
\end{theorem}

This result highlights the central role of synchronous semantics as a 
\emph{core model}: implementability in peer-to-peer systems 
can be reduced to the synchronous case provided the additional safety 
conditions are met. My own contribution focuses on the last item of the 
theorem, namely the problem of checking whether a given global type is 
implementable in synchronous semantics. This question is at the heart 
of the undecidability results presented in Section~\ref{sec:proof}, 
and motivates the need for the identification of new subclasses that 
enable new verification techniques along with tool support.

% TODO: aggiungere una comparazione sulle varie nozioni di implementabilità:
% dire quali sono uguali e quali diverse