\chapter{Introduction} \label{chap:intro}
\pagestyle{plain}
\setcounter{page}{1}

Informally, a \textit{distributed system} is a collection of independent 
computing entities (interchangeably called processes, actors, 
nodes, or participants) that communicate and coordinate their 
actions through message passing over a medium of communication 
(typically an \textbf{asynchronous network}), with the goal of solving a 
common problem. For example, a client-server application can be seen 
as a form of distributed system, where the shared objective is to provide 
services to an end user.

Distributed systems address a wide range of challenges which are 
difficult to solve without such an architecture, for example reliability
under failures safety in critical systems, and consistency of data.
% Being able to \textbf{scale} to handle large volumes of incoming requests 
% would be nearly impossible without distributed systems. Likewise, achieving 
% \textbf{fault-tolerant} and high-performance applications would be 
% extremely difficult without them. 
% For these reasons,
Distributed systems are 
widely adopted in domains such as \textit{cloud computing}, critical 
infrastructures, and telecommunication-oriented applications (i.e.\ 
autonomous cars, aerospace systems, etc.). Given their ubiquity, it is 
crucial to study every aspect of their \textbf{design}, \textbf{execution}, 
and \textbf{verification}.

One recurring difficulty is writing \textbf{correct programs} in this 
context. Avoiding programming and logical errors is inherently hard, even 
for experienced developers. To mitigate this, many abstractions have been 
introduced, and computer scientists have focused their efforts on developing 
\textit{formal frameworks} that provide developers with guarantees about 
their programs. Formal methods for distributed systems offer 
mathematically rigorous techniques to specify, design, and 
verify such systems. They are valuable during development, helping 
detect errors early, and during analysis, enabling the study of critical 
properties such as \textbf{safety}, \textbf{liveness}, and 
\textbf{deadlock-freedom}. Two primary verification approaches are 
\textit{model checking} and \textit{by-construction} verification. Model 
checking systematically explores a system's state space to confirm 
properties, while by-construction verification guarantees correctness 
through the design process itself, preventing errors from being introduced. 

There exists several models to reason about distributed systems.
Different model are specialized in different aspects of a system, and we
are interested in the ones about the exchange of information, such as
Calculus of Communicating Systems (CCS), the $\pi$-calculus, and Petri nets.
In this work, however, we focus on 
\textit{Multiparty Session Types} (MPST)~\cite{honda2008multiparty} 
and \textit{choreographies}~\cite{montesi2014choreographic}, 
since these formalisms place particular emphasis on structured and 
verifiable communication protocols, making them especially well suited 
for protocol design.
In MPST, communication is specified by a \emph{global type}, which 
describes the entire interaction among participants. 
This global type is then \emph{projected} into 
\emph{local types}, one for each participant. 
Local types serve as contracts that guarantee each component is compliant to 
the described protocol, therefore ensuring certain properties, 
such as deadlock-freedom, at compile time. 
The implementability problem in MPST is comparable to verifying 
whether a given global type can be correctly projected into local 
types, preserving the intended behavior.

\section{Goal}
The goal of this work is to study the \textbf{implementability
problem}, which concerns whether a global specification can be
faithfully realized by a set of \textit{local processes} in a
distributed system.
In essence, it asks: does an implementation really \textbf{respect}
the behavior described by a given specification model?

To illustrate the relevance of this problem, consider the following
example.
\begin{example}
Given four processes $A, B, C, D$ distributed over
a network, and four messages $x, y, z, w$ to be exchanged according to
the description in Listing~\ref{lst:not-impl-exm}, is it possible to
implement it in a real world system?

\begin{lstlisting}[caption={Example specification of message exchanges},
                   label={lst:not-impl-exm},
                   keywordstyle=\color{blue}\bfseries,morekeywords={sends,If,then}]
A sends B either message x or y.

If A sends B message x,
    then C sends D message z.

If A sends B message y,
then C sends D message w.
\end{lstlisting}
While the specification can be expressed using several of the
formalisms mentioned earlier, only some are capable of revealing that
it is, in fact, impossible to implement in a real distributed system.
The reason is that process $C$ cannot determine which message to send
to $D$ without knowing which message $A$ sent to $B$, because this 
information is not locally available to $C$.
\end{example}

This problem is examined from a theoretical
perspective to provide a more formal and precise understanding of
the fundamental limits that exist and why syntactical constrains
of certain models works.
In this work, we use an \textit{automata-based} approach to Global Types. 
This formalism is designed to be highly modular, 
incorporating various \textit{network semantics} (such as asynchronous, 
peer-to-peer, causal ordering, and synchronous semantics) as explicit 
parameters of the framework. This parameterization allows flexible 
analysis of different communication models within a unified setting.

The main contributions of this work are: 
\begin{itemize}
    \item a \textbf{state-of-the-art} review, highlighting
    existing research and results in this particular domain and giving
    perspective to our work;
    \item we prove \textbf{undecidability} of the \textit{weak implementability} 
    problem under the synchronous semantics of our framework;
    \item we extend and improve the model-checking tool \textsc{ReSCu}~\cite{rescurepo},  
    enabling verification of \textit{deadlock-freedom} and \textit{progress} 
    for synchronous systems.
\end{itemize}
The report begins in Section~\ref{sec:sota} with a state-of-the-art
overview, presented in a general and accessible manner, avoiding
formal definitions and proofs. Sections~\ref{sec:pre} and
\ref{sec:proof} then introduce the necessary formal definitions,
followed by the main theoretical contribution and in 
Section~\ref{sec:rescu} the practical contributions of
this work. Finally, Section~\ref{sec:end} presents concluding
remarks and outlines possible directions for future research and
development.
