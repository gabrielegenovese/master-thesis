\chapter{Introduction} \label{chap:intro}
\pagestyle{plain}
\setcounter{page}{1}

Informally, a \textit{distributed system} is a collection of independent 
computing entities (interchangeably called processes, actors, 
nodes, or participants) that communicate and coordinate their 
actions through message passing over a medium of communication 
(typically an \textbf{asynchronous network}), with the goal of solving a 
common problem. For example, a client-server application can be seen 
as a form of distributed system, where the shared objective is to provide 
services to an end user.

Distributed systems make it possible to address challenges that are
hard to solve without such an architecture, such as high availability
and elastic scalability. However, these benefits come with their own
set of challanges that computer scientists need to
address, for example, ensuring reliability in the presence of failures
in critical systems, and maintaining data consistency.
Distributed systems are 
widely adopted in domains such as \textit{cloud computing}, critical 
infrastructures, and telecommunication-oriented applications (i.e.\ 
autonomous cars, aerospace systems, etc.). Given their ubiquity, it is 
crucial to study every aspect of their \textbf{design}, \textbf{execution}, 
and \textbf{verification}.
To manage these complexities in a mathematical way, researchers rely on
formal abstractions and rigorous methodologies. These allow us to move
from ad-hoc engineering practices to systematic approaches with
provable guarantees.

One recurring difficulty is writing \textbf{correct programs} in this 
context. Avoiding programming and logical errors is inherently hard, even 
for experienced developers. To mitigate this, many abstractions have been 
introduced, and computer scientists have focused their efforts on developing 
\textit{formal frameworks} that provide developers with guarantees about 
their programs. Formal methods for distributed systems offer 
mathematically rigorous techniques to specify, design, and 
verify such systems.
% Among the many existing formal frameworks, some focus on concurrency,
% others on resource management, and others still on communication. Since
% our interest lies in how processes exchange information, we narrow our
% attention to models that explicitly capture communication behaviour.
They are valuable during development, helping 
detect errors early, and during analysis, enabling the study of critical 
properties such as \textbf{safety}, \textbf{liveness}, and 
\textbf{deadlock-freedom}. Two primary verification approaches are 
\textit{model checking} and \textit{by-construction} verification. Model 
checking systematically explores a system's state space to confirm 
properties, while by-construction verification guarantees correctness 
through the design process itself, preventing errors from being introduced. 

There exists several models to reason about distributed systems.
Different model are specialized in different aspects of a system, and we
are interested in the ones about the exchange of information, such as
Calculus of Communicating Systems (CCS), the $\pi$-calculus, and Petri nets.
In this work, however, we focus on 
\textit{Multiparty Session Types} (MPST)~\cite{honda2008multiparty},
% and \textit{choreographies}~\cite{montesi2014choreographic}, 
since this formalism place particular emphasis on structured and 
verifiable communication protocols, making them especially well suited 
for protocol design.
In MPST, communication is specified by a \emph{global type}, which 
describes the entire interaction among all participants. 
This global type is then \emph{projected} into 
\emph{local types}, one for each participant. 
Local types serve as contracts that guarantee each component is compliant 
to the described protocol, therefore ensuring certain properties, 
such as deadlock-freedom, at compile time. 

\section{Goal}
The goal of this work is to investigate the \textbf{implementability
problem} for MPST, which asks whether a global specification can be
faithfully realised by a collection of \textit{local processes} in a
distributed system. This question naturally arises in top-down
development methodologies, such as MPST or choreographic
frameworks~\cite{montesi2014choreographic}, where the design begins
from a \emph{global perspective} and the local behaviour of each
participant is derived afterwards.

The implementability problem is central to ensuring that the
distributed implementation does not diverge from the intended
specification. In essence, the challenge is to determine whether the
set of projected local processes can really \textbf{respect} the
behaviour prescribed by the global model, while preserving essential
properties such as correctness, progress, and deadlock-freedom. 

A releted-work analysis is provided in
Chapter~\ref{sec:rel}, where we examine how similar problems have been
addressed in other formal frameworks.
To illustrate the relevance of this problem, consider the following
example.

\bigskip

\begin{example}
Consider four processes $A, B, C,$ and $D$ communicating over an
asynchronous network, with four messages $x, y, z,$ and $w$ to be
exchanged as specified in Listing~\ref{lst:not-impl-exm}. A natural
question arises: can such a specification be faithfully implemented in a
real distributed system?

\bigskip

\begin{lstlisting}[caption={Example specification of message exchanges},
                   label={lst:not-impl-exm},
                   keywordstyle=\color{blue}\bfseries,morekeywords={sends,If,then}]
A sends B either message x or y.

If A sends B message x,
    then C sends D message z.

If A sends B message y,
then C sends D message w.
\end{lstlisting}

While the specification can be expressed using several of the
formalisms mentioned earlier, only some are capable of revealing that
it is, in fact, impossible to implement in a real distributed system.
The reason is that process $C$ cannot determine which message to send
to $D$ without knowing which message $A$ sent to $B$, because this 
information is not locally available to $C$.
\end{example}

The implementability problem in MPST is comparable to verifying 
whether a given global type can be correctly projected into local 
types, preserving the intended behaviour.
This problem is examined from a theoretical
perspective to provide a more formal and precise understanding of
the fundamental limits that exist and why syntactical constrains
of certain models work.
In this work, we use an \textit{automata-based} approach to Global Types. 
This formalism is designed to be highly modular, 
incorporating various \textit{network semantics} (such as asynchronous, 
peer-to-peer, causal ordering, and synchronous semantics) as explicit 
parameters of the framework. This parameterization allows flexible 
analysis of different communication models within a unified setting.

To properly situate the MPST implementability problem, it is useful to
recall similar questions studied for Message Sequence Charts 
(MSCs)~\cite{alur2000inference,alur2003inference}.
These formalisms provide both historical context and technical insights, 
as some known results will be used in this work.

Message Sequence Charts (MSCs) are a standardised graphical formalism,
introduced in 1992 \cite{MSCStandard}, used to describe trace languages 
for specifying communication behaviour. Thanks to their simplicity and 
intuitive semantics, MSCs have been widely adopted in industry.
Figure~\ref{fig:msc-cli-ser} illustrates a simple example based on a
minimal client–server architecture. An extension of this formalism,
known as High-Level Message Sequence Charts (HMSCs), was later
introduced \cite{HMSCStandard}. HMSCs enable the definition of
MSCs as nodes connected by transitions and are used to model more
complex patterns of message flows by capturing sequences, alternatives,
or iterations of atomic MSC scenarios.

\begin{figure}[!ht]
\centering
\begin{msc}[draw frame=none, draw head=none, msc keyword=, head height=0px, label distance=0.5ex, foot height=0px, foot distance=0px]{}
	\declinst{P1}{Client}{}
	\declinst{P2}{Server}{}

	\mess{request}{P1}{P2}
	\nextlevel
	\mess{answer}{P2}{P1}
\end{msc}
\caption{Simple example of a client-server architecture.}
\label{fig:msc-cli-ser}
\end{figure}

The \emph{weak implementability problem} for MSCs asks whether there 
exists a distributed implementation that can realise all behaviours of a 
finite set of MSCs without introducing additional ones. A stronger variant, 
called \emph{safe implementability}, requires the implementation to also be 
\textbf{deadlock-free}. This problem has some synonyms in the term, with 
slightly different definition, but it can be called also realisability and 
projectability.

With MSCs, the work \cite{di2023partial} presents some interesting communication
semantics. I will describe a few them informally, using examples to
highlight the differences from the main semantics considered in this work,
which is \verb|synch|, that is also the only one formally defined in 
Definition~\ref{def:synchronous}. In Chapter~\ref{sec:rel}, the discussion
continue presenting other communication semantics, and summarizing
the relevance of the work by Di Giusto et al.~\cite{di2023partial}.
Some examples are shown in Figure~\ref{fig:asy},~and~\ref{fig:sync},
whose \emph{membership} to these classes
can be verified with an online tool for MSCs~\cite{MSCTool}. 

\paragraph{Fully asynchronous}

In the fully asynchronous communication model (\verb|asy|), messages can be 
received at any time after they have been sent, and send events are 
non-blocking. This model can be viewed as an unordered ``bag'' in which 
all messages are stored and retrieved by processes when needed. It is also 
referred to as \emph{non-FIFO}. The formal definition coincides with that of 
an MSC (Definition~\ref{def:msc}). Figure~\ref{fig:asy} illustrates an example of asynchronous 
communication.

\begin{figure}[!ht]
    \centering
      \begin{msc}[draw frame=none, draw head=none, msc keyword=, 
                  head height=0px, label distance=0.5ex, 
                  foot height=0px, foot distance=0px]{}
          \declinst{p}{p}{}
          \declinst{q}{q}{}

          \mess[pos=0.2]{$m_1$}{p}{q}[2]
          \nextlevel
          \mess[pos=0.8]{$m_2$}{p}{q}
      \end{msc}
  \caption{Asynchronous semantic example.}
  \label{fig:asy}
\end{figure}

\paragraph{Peer-to-peer} 
In the peer-to-peer (\verb|p2p|) communication model, any two messages sent from one 
process to another are always received in the same order as they are sent.
An alternative name is FIFO. An example is shown in Figure~\ref{fig:p2p}.

\begin{figure}[!ht]
    \centering
      \begin{msc}[draw frame=none, draw head=none, msc keyword=, 
                    head height=0px, label distance=0.5ex, 
                    foot height=0px, foot distance=0px]{}
            \declinst{p}{p}{}
            \declinst{q}{q}{}
            \declinst{r}{r}{}

            \mess[pos=0.15]{$m_1$}{p}{r}[3]
            \nextlevel
            \mess[pos=0.8]{$m_2$}{p}{q}
            \nextlevel
            \mess[pos=0.8]{$m_3$}{q}{r}
        \end{msc}
  \caption{Peer-to-peer semantic example.}
  \label{fig:p2p}
\end{figure}

\paragraph{Synchronous}
The synchronous (\verb|synch|) communication model imposes 
the existence of a scheduling such that any send event is 
immediately followed by its corresponding receive event. 
An example for this communication model is shown in 
Figure~\ref{fig:sync}.c. A formal definition is given later 
for this semantic (Definition~\ref{def:synchronous}).

\begin{figure}[!ht]
    \centering
      \begin{msc}[draw frame=none, draw head=none, msc keyword=, 
                  head height=0px, label distance=0.5ex, 
                  foot height=0px, foot distance=0px]{}
          \declinst{p}{p}{}
          \declinst{q}{q}{}
          \declinst{r}{r}{}

          \mess{$m_1$}{p}{q}
          \nextlevel
          \mess{$m_2$}{q}{r}
      \end{msc}
  \caption{Synchronous semantic example.}
  \label{fig:sync}
\end{figure}

The definition of these models will become central in
the reduction techniques explored in this work: 
simplifying the study of implementability
by reducing richer semantics to the synchronous case.

\section{Reduction to synchronous semantic}
The main idea of this work is that reasoning about implementability 
becomes more tractable under \emph{synchronous} 
semantics for automata-based solutions to the implementability problem. 
In synchronous communication, send and receive actions 
are tightly coupled, effectively removing nondeterminism 
caused by asynchronous message buffering. Several results exploit this 
observation by reducing the implementability problem under richer 
communication models (e.g.\ asynchronous or peer-to-peer FIFO) to the 
simpler synchronous case~\cite{alur2005realizability,di2023partial}.

Formally, one can show that if a global type is implementable in 
synchronous semantics, then under certain conditions it is also 
implementable in more general models such as peer-to-peer or mailbox 
semantics. This reduction requires constraints such as 
\emph{orphan-freedom} (no message is left unmatched) and
\emph{deadlock-freedom}.  

The following theorem, currently a work in progress by my 
supervisors~\cite[Theorem 5.3]{di2025realisability}, 
provides a characterization of a connection between 
peer-to-peer semantics and synchronous semantics:
a global type $G$ is deadlock-free realisable in \verb|p2p| iff
the following four conditions hold
\begin{itemize}
  \item the language of $G$'s local type is in synchronous semantics;
  \item all $G$'s projections are orphan-free;
  \item all the traces of the MSCs' language of $G$ are deadlock-free
  in \verb|p2p|;
  \item $G$ is realisable in synchronous semantics.
\end{itemize}

The second and third conditions are already known to be decidable and 
can be automatically verified. The focus of this thesis is instead on 
the fourth condition, namely checking whether a global type is 
implementable in synchronous semantics. The undecidability result 
presented in Chapter~\ref{sec:proof} shows that this condition cannot 
be verified in general. Consequently, the theorem above must be refined 
by introducing further restrictions that ensure decidability.  

This observation motivates the second part of the thesis: 
Chapter~\ref{sec:rescu} presents the extension of the 
\textsc{ReSCu} tool, which provides practical verification of 
properties such as \emph{deadlock-freedom} and \emph{progress}. These 
results should be understood as building blocks toward identifying 
restricted subclasses of synchronous systems that admit decidable 
implementability checks, complementing the undecidability findings of 
the theoretical contribution.

Given the context, the developments presented in this thesis can be 
grouped into two main
contributions, one theoretical and one practical, both closely
connected.

\section{Contributions}
The main contributions of this work are: 
\begin{itemize}
    \item a proof of the \textbf{undecidability} of the 
    \textit{weak implementability} problem under the synchronous 
    semantics of our framework;
    \item an extension and improvement of the model-checking tool 
    \textsc{ReSCu}~\cite{rescurepo}, enabling the verification of 
    \textit{deadlock-freedom} and \textit{progress} for synchronous 
    systems;
    \item a related-work overview, highlighting existing research and 
    results in this particular domain, providing a structured summary 
    and comparison with our approach, and setting the perspective for 
    the contributions that follow.
\end{itemize}

These two contributions are closely connected: they both address the 
implementability problem, but from two complementary angles. The first 
contribution establishes undecidability, showing that in the general 
case the weak implementability problem cannot be solved for synchronous semantic. 
This motivates the second contribution: once undecidability is proven, 
there is a clear need to identify suitable restrictions of the problem 
that yield decidability results. The model-checking framework presented 
in the second part of the thesis is designed as a foundational step 
towards this direction, providing practical verification techniques that 
can serve as building blocks for further decidability analyses.  

The thesis is structured as follows. 
Chapter~\ref{chap:intro} (the present chapter) gives a high-level 
description of the frameworks used, avoiding formal definitions and 
proofs for accessibility. Chapters~\ref{sec:pre} and~\ref{sec:proof} 
then introduce the formal definitions and present the main theoretical 
contribution. Chapter~\ref{sec:rescu} develops the practical 
contributions through the \textsc{ReSCu} tool. 
Chapter~\ref{sec:rel} presents 
a detailed overview of related work, comparing different approaches in 
the literature and highlighting how this thesis departs from them. Finally, 
Chapter~\ref{sec:end} concludes with a discussion of the results and 
outlines directions for future research and development.
