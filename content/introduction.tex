\chapter{Introduction} \label{chap:intro}
\pagestyle{plain}
\setcounter{page}{1}

Informally, a \textit{distributed system} is a collection of independent 
computing entities (interchangeably called processes, actors, 
nodes, or participants) that communicate and coordinate their 
actions through message passing over a medium of communication 
(typically an \textbf{asynchronous network}), with the goal of solving a 
common problem. For example, a client-server application can be seen 
as a form of distributed system, where the shared objective is to provide 
services to an end user.

Distributed systems make it possible to address challenges that are
hard to solve without such an architecture, such as high availability
and elastic scalability. However, these benefits come with their own
set of difficulties that computer scientists have long sought to
overcome-for example, ensuring reliability in the presence of failures
in critical systems, and maintaining data consistency.
Distributed systems are 
widely adopted in domains such as \textit{cloud computing}, critical 
infrastructures, and telecommunication-oriented applications (i.e.\ 
autonomous cars, aerospace systems, etc.). Given their ubiquity, it is 
crucial to study every aspect of their \textbf{design}, \textbf{execution}, 
and \textbf{verification}.

One recurring difficulty is writing \textbf{correct programs} in this 
context. Avoiding programming and logical errors is inherently hard, even 
for experienced developers. To mitigate this, many abstractions have been 
introduced, and computer scientists have focused their efforts on developing 
\textit{formal frameworks} that provide developers with guarantees about 
their programs. Formal methods for distributed systems offer 
mathematically rigorous techniques to specify, design, and 
verify such systems. They are valuable during development, helping 
detect errors early, and during analysis, enabling the study of critical 
properties such as \textbf{safety}, \textbf{liveness}, and 
\textbf{deadlock-freedom}. Two primary verification approaches are 
\textit{model checking} and \textit{by-construction} verification. Model 
checking systematically explores a system's state space to confirm 
properties, while by-construction verification guarantees correctness 
through the design process itself, preventing errors from being introduced. 

There exists several models to reason about distributed systems.
Different model are specialized in different aspects of a system, and we
are interested in the ones about the exchange of information, such as
Calculus of Communicating Systems (CCS), the $\pi$-calculus, and Petri nets.
In this work, however, we focus on 
\textit{Multiparty Session Types} (MPST)~\cite{honda2008multiparty} 
and \textit{choreographies}~\cite{montesi2014choreographic}, 
since these formalisms place particular emphasis on structured and 
verifiable communication protocols, making them especially well suited 
for protocol design.
In MPST, communication is specified by a \emph{global type}, which 
describes the entire interaction among participants. 
This global type is then \emph{projected} into 
\emph{local types}, one for each participant. 
Local types serve as contracts that guarantee each component is compliant to 
the described protocol, therefore ensuring certain properties, 
such as deadlock-freedom, at compile time. 
The implementability problem in MPST is comparable to verifying 
whether a given global type can be correctly projected into local 
types, preserving the intended behaviour.

\section{Goal}
The goal of this work is to study the \textbf{implementability
problem}, which concerns whether a global specification can be
faithfully realised by a set of \textit{local processes} in a
distributed system.
In essence, it asks: does an implementation really \textbf{respect}
the behaviour described by a given specification model?

To illustrate the relevance of this problem, consider the following
example.
\begin{example}
Given four processes $A, B, C, D$ distributed over
a network, and four messages $x, y, z, w$ to be exchanged according to
the description in Listing~\ref{lst:not-impl-exm}, is it possible to
implement it in a real world system?

\begin{lstlisting}[caption={Example specification of message exchanges},
                   label={lst:not-impl-exm},
                   keywordstyle=\color{blue}\bfseries,morekeywords={sends,If,then}]
A sends B either message x or y.

If A sends B message x,
    then C sends D message z.

If A sends B message y,
then C sends D message w.
\end{lstlisting}
While the specification can be expressed using several of the
formalisms mentioned earlier, only some are capable of revealing that
it is, in fact, impossible to implement in a real distributed system.
The reason is that process $C$ cannot determine which message to send
to $D$ without knowing which message $A$ sent to $B$, because this 
information is not locally available to $C$.
\end{example}

This problem is examined from a theoretical
perspective to provide a more formal and precise understanding of
the fundamental limits that exist and why syntactical constrains
of certain models work.
In this work, we use an \textit{automata-based} approach to Global Types. 
This formalism is designed to be highly modular, 
incorporating various \textit{network semantics} (such as asynchronous, 
peer-to-peer, causal ordering, and synchronous semantics) as explicit 
parameters of the framework. This parameterization allows flexible 
analysis of different communication models within a unified setting.

The main contributions of this work are: 
\begin{itemize}
    \item a related-work overview, highlighting
    existing research and results in this particular domain and giving
    perspective to our work;
    \item we prove \textbf{undecidability} of the \textit{weak implementability} 
    problem under the synchronous semantics of our framework;
    \item we extend and improve the model-checking tool \textsc{ReSCu}~\cite{rescurepo},  
    enabling verification of \textit{deadlock-freedom} and \textit{progress} 
    for synchronous systems.
\end{itemize}
The thesis begins with a brief overview of the background
in Section~\ref{sec:sota}, presented in a general and accessible manner, avoiding
formal definitions and proofs. Chapters~\ref{sec:pre} and
\ref{sec:proof} then introduce the necessary formal definitions,
followed by the main theoretical contribution and in 
Chapter~\ref{sec:rescu} the practical contributions of
this work. Finally, Chapter~\ref{sec:end} presents concluding
remarks and outlines possible directions for future research and
development.

% discorso più ampio sulla roadmap e sulla struttura della tesi, e su cosa di basa etc...

\section{Background}\label{sec:sota}
In this overview, we give a brief understanding of the
state of the art regarding Message Sequence Charts, Multiparty 
Session Types and the implementability 
(or realisability) problem.

\subsection{Message Sequence Charts}
Message Sequence Charts (MSCs) are a standardized graphical formalism,
introduced in 1992 \cite{MSCStandard}, used to describe trace languages for specifying
communication behaviour. Thanks to their simplicity and intuitive
semantics, MSCs have been widely adopted in industry.
Figure~\ref{fig:msc-cli-ser} illustrates a simple example based on a
minimal client–server architecture. An extension of this formalism,
known as High-Level Message Sequence Charts (HMSCs), was later
introduced \cite{HMSCStandard}. HMSCs enable the definition of
MSCs as nodes connected by transitions and are used to model more
complex patterns of message flows by capturing sequences, alternatives,
or iterations of atomic MSC scenarios.

\begin{figure}[!ht]
\centering
\begin{msc}[draw frame=none, draw head=none, msc keyword=, head height=0px, label distance=0.5ex, foot height=0px, foot distance=0px]{}
	\declinst{P1}{Client}{}
	\declinst{P2}{Server}{}

	\mess{request}{P1}{P2}
	\nextlevel
	\mess{answer}{P2}{P1}
\end{msc}
\caption{Simple example of a client-server architecture.}
\label{fig:msc-cli-ser}
\end{figure}

With MSCs, \cite{di2023partial} presents some interesting communication
semantics. I will describe a few them informally, using examples to
highlight the differences from the main semantics considered in this work,
which is \verb|synch|, that is also the only one formally defined in 
Definition~\ref{def:synchronous}. In Section~\ref{sec:rel}, the discussion
continue presenting other communication semantics, and summarizing
the relevance of the work by Di Giusto, et al.~\cite{di2023partial}.
Some examples are shown in 
Figure~\ref{fig:asy},~\ref{fig:p2p}~and~\ref{fig:sync},
whose \emph{membership} 
can be verified with an online tool for MSCs~\cite{MSCTool}. 

\paragraph{Fully asynchronous}

In the fully asynchronous communication model (\verb|asy|), messages can be 
received at any time after they have been sent, and send events are 
non-blocking. This model can be viewed as an unordered ``bag'' in which 
all messages are stored and retrieved by processes when needed. It is also 
referred to as \emph{non-FIFO}. The formal definition coincides with that of 
an MSC (Definition~\ref{def:msc}). Figure~\ref{fig:asy} illustrates an example of asynchronous 
communication.

\begin{figure}[!ht]
    \centering
      \begin{msc}[draw frame=none, draw head=none, msc keyword=, 
                  head height=0px, label distance=0.5ex, 
                  foot height=0px, foot distance=0px]{}
          \declinst{p}{p}{}
          \declinst{q}{q}{}

          \mess[pos=0.2]{$m_1$}{p}{q}[2]
          \nextlevel
          \mess[pos=0.8]{$m_2$}{p}{q}
      \end{msc}
  \caption{Asynchronous semantic example.}
  \label{fig:asy}
\end{figure}

\paragraph{Peer-to-peer} 
In the peer-to-peer (\verb|p2p|) communication model, any two messages sent from one 
process to another are always received in the same order as they are sent.
An alternative name is FIFO. An example is shown in Figure~\ref{fig:p2p}.

\begin{figure}[!ht]
    \centering
      \begin{msc}[draw frame=none, draw head=none, msc keyword=, 
                    head height=0px, label distance=0.5ex, 
                    foot height=0px, foot distance=0px]{}
            \declinst{p}{p}{}
            \declinst{q}{q}{}
            \declinst{r}{r}{}

            \mess[pos=0.15]{$m_1$}{p}{r}[3]
            \nextlevel
            \mess[pos=0.8]{$m_2$}{p}{q}
            \nextlevel
            \mess[pos=0.8]{$m_3$}{q}{r}
        \end{msc}
  \caption{Peer-to-peer semantic example.}
  \label{fig:p2p}
\end{figure}

\paragraph{Synchronous}
The synchronous (\verb|synch|) communication model imposes 
the existence of a scheduling such that any send event is 
immediately followed by its corresponding receive event. 
An example for this communication model is shown in 
Figure~\ref{fig:sync}.c. A formal definition is given later 
for this semantic (Definition~\ref{def:synchronous}).

\begin{figure}[!ht]
    \centering
      \begin{msc}[draw frame=none, draw head=none, msc keyword=, 
                  head height=0px, label distance=0.5ex, 
                  foot height=0px, foot distance=0px]{}
          \declinst{p}{p}{}
          \declinst{q}{q}{}
          \declinst{r}{r}{}

          \mess{$m_1$}{p}{q}
          \nextlevel
          \mess{$m_2$}{q}{r}
      \end{msc}
  \caption{Synchronous semantic example.}
  \label{fig:sync}
\end{figure}

% TODO: rimanere ad alto livello
% ma aggiungere reference alle definizioni formali 
% nella parte di definizioni formali fare un confronto.
% citare il confronto qua e rimandare dopo

\subsubsection{The implementability problem for MSCs}

The \emph{implementability problem} was first introduced for MSC languages 
in~\cite{alur2000inference,alur2003inference}. It asks whether there exists a distributed 
implementation that can realise all behaviours of a finite set of MSCs 
without introducing additional ones. A stronger variant, called 
\emph{safe implementability}, requires the implementation to also be \textbf{deadlock-free}. 
This problem has some synonyms in the term, with slightly different definition,
but it can be called also realisability and projectability.
% In this setting, distributed implementations are defined using communicating 
% automata with explicit accepting states, and deadlock-freedom corresponds 
% to the ability to reach an accepting state from any reachable configuration. 


\subsection{Multiparty Session Types}
Multiparty Session Types (MPST)~\cite{honda2008multiparty} 
provide a type-theoretic framework to specify and verify communication 
protocols among multiple participants. They ensure that communication 
follows a predefined structure, preventing errors such as deadlocks, 
orphan messages, and unspecified receptions. The 
\textbf{global specification} describes the overall communication 
protocol. From this, one derives the \textbf{local behaviours} of each 
participant via a \emph{projection} operation. The system's 
\textbf{processes} form the \emph{implementation}, defining how 
participants interact. With the definition of a \emph{typing system} 
and suitable \emph{type-checking rules}, one ensures that the 
implementation conforms to the local specification, thereby 
guaranteeing properties such as \emph{well-formedness}.  
Figure~\ref{fig:mpstschema} show a schema summarizing the principal
parts of the framework.

\begin{figure}[!ht]
\centering
\begin{tikzpicture}[
      node distance=1.2cm,
      every node/.style={font=\sffamily},
      rect/.style={rectangle, draw=black, minimum width=1cm, minimum height=1cm},
      circ/.style={circle, draw=black, minimum size=1cm},
      arrow/.style={-{Stealth[scale=1.1]}, thick}
  ]

  % Nodes
  \node[rect] (G) {\textcolor{red}{$\mathcal{G}$}};
  \node[circ, below=of G] (TB) {\textcolor{blue}{$L_\text{B}$}};
  \node[circ, left=of TB] (TA) {\textcolor{blue}{$L_\text{A}$}};
  \node[circ, right=of TB] (TC) {\textcolor{blue}{$L_\text{C}$}};

  \node[rect, below=of TA] (PA) {\textcolor{brown}{$P_\text{A}$}};
  \node[rect, below=of TB] (PB) {\textcolor{brown}{$P_\text{B}$}};
  \node[rect, below=of TC] (PC) {\textcolor{brown}{$P_\text{C}$}};

  \node[rect,draw=none,right=of TC] (LC) {\textbf{\textcolor{blue}{2. Local type}}};
  \node[rect,draw=none,above=of LC] (LG) {\textbf{\textcolor{red}{1. Global type}}};
  \node[rect,draw=none,below=of LC] (LG) {\textbf{\textcolor{brown}{3. Processes}}};

  % Arrows
  \draw[arrow] (G) -- (TA) node[midway, left] {Projection};
  \draw[arrow] (G) -- (TB);
  \draw[arrow] (G) -- (TC);

  \draw[arrow] (PA) -- (TA) node[midway, left] {Type checking};
  \draw[arrow] (PB) -- (TB);
  \draw[arrow] (PC) -- (TC);
  \draw[arrow] (TC) -- (PC);
  \draw[arrow] (TB) -- (PB);
  \draw[arrow] (TA) -- (PA);

\end{tikzpicture}
\caption{Intuitive schema of MPST framework}
\label{fig:mpstschema}
\end{figure}

% In this setting, the analogue of realisability is often called 
% \emph{session fidelity}: a property ensuring that the combined local 
% types behave exactly according to the global type. When the global 
% type satisfies syntactic restrictions, its projection is guaranteed 
% to be both realisable and deadlock-free, which corresponds to safe 
% realisability in the MSC setting.  

\subsubsection{Projectability}
A central notion in MPST is \emph{projectability}, which asks whether 
a global type can be faithfully projected into local specifications for 
each participant. If projection succeeds, the resulting local types 
interact without mismatches or unintended behaviours, effectively 
bridging global specifications and distributed implementations~\cite{honda2008multiparty}.  
Projectability is, therefore, comparable to the implementability
problem as they have the same aim.
Projection algorithms, however, often reject natural protocols that 
fail to meet restrictive syntactic conditions. This difference between 
expressivity and safety has motivated extensions of the theory, with 
\cite{castagna2012global} being the only algorithm aiming for full 
completeness.

% \subsubsection{Mixed and Sender driven choice}
A key restriction in the definition of MPST appears in branching. 
In the original framework \cite{honda2008multiparty,carbone2012structured}, 
choice is \textbf{sender-driven}: the first sender dictates the branch, 
ensuring safety but excluding many common patterns where multiple 
participants influence the decision~\cite{carbone2012structured}.  
Allowing \textbf{mixed choice} increases expressivity by permitting 
several initiators, but it also makes the implementability problem 
undecidable in general~\cite{stutz2024implementability}.  

\subsection{Reduction to synchronous semantic}
The main idea of this work is that reasoning about implementability 
becomes more tractable under \emph{synchronous} 
semantics for automata-based solutions to the implementability problem. 
In synchronous communication, send and receive actions 
are tightly coupled, effectively removing nondeterminism 
caused by asynchronous message buffering. Several results exploit this 
observation by reducing the implementability problem under richer 
communication models (e.g.\ asynchronous or peer-to-peer FIFO) to the 
simpler synchronous case~\cite{alur2005realizability,di2023partial}.

Formally, one can show that if a global type is implementable in 
synchronous semantics, then under certain conditions it is also 
implementable in more general models such as peer-to-peer or mailbox 
semantics. This reduction requires constraints such as 
\emph{orphan-freedom} (no message is left unmatched) and
\emph{deadlock-freedom}.
The following theorem, currently a work in progress by my 
supervisors~\cite[Theorem 5.3]{di2025realisability}, 
provides a characterization of a connection between 
peer-to-peer semantics and synchronous semantics:
a global type $G$ is deadlock-free realisable in p2p iff
the following four conditions hold
\begin{itemize}
  \item the language of $G$'s local type is in synchronous semantics
  \item all $G$'s projections are orphan-free
  \item all the traces of the MSCs' language of $G$ is deadlock-free
  in \verb|p2p|
  \item $G$ is realisable in synchronous semantics
\end{itemize}

This result highlights the central role of synchronous semantics as a 
\emph{core model}: implementability in peer-to-peer systems 
can be reduced to the synchronous case provided the additional safety 
conditions are met. My own contribution focuses on the last item of the 
theorem, namely the problem of checking whether a given global type is 
implementable in synchronous semantics. This question is at the heart 
of the undecidability results presented in Section~\ref{sec:proof}, 
and motivates the need for the identification of new subclasses that 
enable new verification techniques along with tool support.