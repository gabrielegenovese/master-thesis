\chapter{Conclusion}\label{sec:end}
This work addressed the \emph{realisability problem} for Global 
Types, a central concern in the verification of distributed systems. 
After a brief overview of the problem, we positioned our contribution 
within an ongoing research effort, bridging well-established 
theoretical foundations with practical tool development.  

On the theoretical side, we introduced the necessary background 
notions (i.e.\ CFSMs, Global Types, MSCs, and communication models) and 
formalised weak realisability. The main contribution was to connect 
the realisability problem to classical undecidability results, in 
particular through a reduction to the Relaxed Post Correspondence 
Problem (RPCP). This result, presented in 
Chapter~\ref{sec:proof}, establishes a fundamental limitation 
of realisability under synchronous semantics and lays the groundwork 
for exploring decidable subclasses and practical approximations.

On the practical side, detailed in Chapter~\ref{sec:rescu}, we improved 
and extended the \textsc{ReSCu} tool, used for checking realisability 
and other semantic properties of Symbolic Communicating Machines 
(SCMs). The input grammar was refined for greater usability, and new 
verification routines were implemented, including checks for progress 
and deadlock-freedom. The tool now also generates visual 
representations of synchronous systems, along with illustrative 
examples. These extensions strengthen \textsc{ReSCu} both as a research 
prototype and as a practical aid for automated verification.ù

In Chapter~\ref{sec:rel}, we analysed part of the state of the art on 
realisability, comparing our definitions and results with existing 
approaches in the literature. This analysis helped clarify how our 
formalisation of weak and safe realisability fits within, and extends, 
previous frameworks, particularly those by 
Alur~et~al.~\cite{alur2005realizability}, 
Lohrey~et~al.~\cite{lohrey2003realizability}, and 
Stutz~et~al.~\cite{stutz2024implementability}. The comparison also 
highlighted key conceptual differences, especially in how communication 
semantics and closure properties are handled, further motivating our 
formal treatment.

\section{Future Work}
Future research directions include extending the theoretical results
beyond weak realisability toward a decidability result for
\emph{safe realisability}, thereby incorporating properties such as
deadlock-freedom directly into the analysis of Global Types. This line
of investigation will build upon the techniques developed in this work
and extend the existing results of Lohrey~et~al.~\cite{lohrey2003realizability}.  

Another important direction is the exploration of alternative
communication semantics, such as causal-order and mailbox-based models,
which were mentioned throughout this work. Investigating the
realisability problem under these semantics could shed light on how
communication constraints and buffering behaviours affect implementability
and decidability. Establishing precise connections between synchronous
semantics and these more general models may also lead to new transfer
results, showing under which conditions realisability in one model
implies realisability in another.  

On the practical side, a natural objective is to further enhance
\textsc{ReSCu} to support these theoretical extensions, ultimately
aiming for a complete and automated framework to decide realisability
for restricted classes of Global Types. This would enable systematic
benchmarking against existing tools and the validation of the approach
on real-world communication protocols.  

Finally, it would be valuable to investigate the role of \emph{fairness}
and other semantic variants, within the verification process. 
These extensions could help
model more realistic distributed environments and provide a deeper
understanding of how fairness assumptions influence the decidability and
correctness of realisable systems.
