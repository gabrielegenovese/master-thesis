\chapter{Conclusion}\label{sec:end}
This work addressed the \emph{realisability problem} for Global 
Types, a central concern in the verification of distributed systems. 
After surveying the state of the art, I positioned our contribution 
within an ongoing research effort, bridging well-established 
theoretical foundations with practical tool development.  

On the theoretical side, I introduced the necessary background 
notions (i.e. CFSMs, Global Types, MSCs, and communication models) and 
formalized weak realisability. The main contribution was to connect 
the realisability problem to classical undecidability results, in 
particular through a reduction to the Relaxed Post Correspondence 
Problem (RPCP).  

On the practical side, I improved and extended the 
\textsc{ReSCu} tool, used for checking realisability and other semantic 
properties of Symbolic Communicating Machines (SCMs). The input grammar 
was refined for greater usability, and new verification routines were 
implemented, including checks for progress and deadlock-freedom. The tool 
now also generates visual representations of synchronous systems, along 
with illustrative examples. These extensions strengthen 
\textsc{ReSCu} both as a research prototype and as a practical aid for 
automated verification.  

Overall, the contributions span two complementary directions: a refined 
theoretical understanding of realisability, and concrete advances in 
tool support for experimenting with increasingly expressive models. 

\section{Future Work}
Future directions include extending the theoretical results beyond weak 
realisability toward a decidability result of \emph{safe realisability} 
(therefore, including deadlock-freedom) for 
Global Types, building on the techniques developed here and extending
an existing proof made by Lohrey, et al. \cite{lohrey2003realizability}.
On the practical side, a natural goal is to further enhance \textsc{ReSCu} to  
support these results, ultimately aiming for a complete algorithm to decide 
realisability for restricted classes of Global Types. This would 
enable systematic benchmarking against existing methods and real-world 
protocols.