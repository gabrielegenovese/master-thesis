\chapter{Preliminaries}\label{sec:pre}
In this section, the fundamental concepts and definitions necessary 
to contextualize the main contributions of this work are presented. 
I, first, introduce automaton, executions, and
Message Sequence Charts (MSC), followed by 
an examination of communication model's semantics that are particularly 
interesting. 
Then, the notions of Global Type and Realisability are 
defined within the scope of this work, along with the foundational 
elements required to understand the theoretical contributions.

\section{Standard notions on automata}
For a string $s$, let $s^l$ denote the $l$-th character of the string.

\bigskip

\begin{definition}[NFA]\label{def:nfa}
A non-deterministic finite automaton (NFA) is a tuple  
$\mathcal{A} = (Q, \Sigma, \delta, q_0, F)$, where $Q$ is a finite set of  
states, $\Sigma$ is a finite alphabet,  
$\delta : Q \times (\Sigma \cup \{\varepsilon\}) \to Q$ is the transition  
relation, $q_0 \in Q$ is the initial state, and $F \subseteq Q$ is the set  
of accepting states.  

We write $\delta^*(s,w)$ to denote the set of states $s'$ reachable from  
$s$ along a path labelled with $w$. The language accepted by $\mathcal{A}$,  
denoted $\mathcal{L}_{\text{words}}(\mathcal{A})$, is the set of words  
$w \in \Sigma^*$ such that $\delta^*(q_0,w) \cap F \neq \emptyset$.  
\end{definition}

\bigskip

\begin{definition}[DFA]\label{def:dfa}
A deterministic finite automaton (DFA) is an NFA where the transition  
relation $\delta$ is a partial function  
$\delta : Q \times \Sigma \to Q$. The DFA is complete if $\delta$ is  
total.  

% Every DFA $\mathcal{A} = (Q, \Sigma, \delta, q_0, F)$ accepts the same  
% language as the complete DFA $\mathcal{A}' = (Q \cup \{\perp\}, \Sigma,  
% \delta', q_0, F)$, where $\delta'(q,a) = \perp$ if $\delta(q,a)$ is  
% undefined, and $\delta'(\perp,a) = \perp$.  
\end{definition}

\bigskip

\begin{definition}[Determinization]\label{def:det}
To every NFA $\mathcal{A} = (Q, \Sigma, \delta, q_0, F)$, we associate  
the DFA $\text{det}(\mathcal{A}) = (Q', \Sigma, \delta', q'_0, F')$,  
where $Q' = 2^Q$, $q'_0 = \{q_0\}$, $F'$ is the set of subsets of $Q$  
that contain at least one accepting state, and $\delta'$ is defined by  
$\delta'(S,a) = \bigcup\{\delta^*(s,a) \mid s \in S\}$ for all  
$S \in Q'$, $a \in \Sigma$.  
\end{definition}

We write $\acceptcompletion{\mathcal A}{}$ for the automaton obtained
from $\mathcal A$ by setting $F=Q$.

% \begin{definition}[Product NFA]\label{def:product-nfa}
% Given two NFAs $\mathcal{A}_1 = (Q_1, \Sigma, \delta_1, q_{0,1}, F_1)$ and  
% $\mathcal{A}_2 = (Q_2, \Sigma, \delta_2, q_{0,2}, F_2)$, the product NFA is  
% defined as  
% \[
% \mathcal{A}_1 \otimes \mathcal{A}_2 \stackrel{\text{def}}{=}  
% (Q_1 \times Q_2, \Sigma, \delta, (q_{0,1}, q_{0,2}), F_1 \times F_2),
% \]  
% where the transition relation $\delta$ is defined as follows:  

% \begin{itemize}  
%   \item for all $(s_1,s_2) \in Q_1 \times Q_2$, $a \in \Sigma$,  
%   $(s'_1,s'_2) \in Q_1 \times Q_2$:  
%   $((s_1,s_2),a,(s'_1,s'_2)) \in \delta$  
%   iff $(s_1,a,s'_1) \in \delta_1$ and $(s_2,a,s'_2) \in \delta_2$.  

%   \item for all $(s_1,s_2) \in Q_1 \times Q_2$, $(s'_1,s'_2) \in Q_1 \times Q_2$:  
%   $((s_1,s_2),\varepsilon,(s'_1,s'_2)) \in \delta$  
%   iff $(s_1,\varepsilon,s'_1) \in \delta_1$ or $(s_2,\varepsilon,s'_2) \in \delta_2$.  
% \end{itemize}  

% It holds that  
% $\mathcal{L}_{\text{words}}(\mathcal{A}_1 \otimes \mathcal{A}_2) =  
% \mathcal{L}_{\text{words}}(\mathcal{A}_1) \cap  
% \mathcal{L}_{\text{words}}(\mathcal{A}_2)$.  
% \end{definition}

% \begin{definition}[Dual DFA]\label{def:dual-dfa}
% If $\mathcal{A} = (Q, \Sigma, \delta, q_0, F)$ is a DFA, its dual is  
% defined as  
% \[
% \mathcal{A}^{\text{dual}} = (Q, \Sigma, \delta, q_0, Q \setminus F).
% \]  
% If $\mathcal{A}$ is complete, then  
% $\mathcal{L}_{\text{words}}(\mathcal{A}^{\text{dual}}) =  
% \Sigma^* \setminus \mathcal{L}_{\text{words}}(\mathcal{A})$.  
% \end{definition}

\section{Execution, Communication Models and MSC}
We assume a finite set of \emph{processes} 
$\Procs=\{p,q,\ldots,P1,P2,\ldots\}$ and a finite set of 
messages (labels) $\Msg=\{\msg_1,\msg_2,\ldots\}$.  
We consider two kinds of actions:  
\begin{itemize}
	\item \emph{send actions}, of the form $\send{p}{q}{\msg}$, 
	executed by process $p$ when sending message $\msg$ to $q$;  
	\item \emph{receive actions}, of the form $\recv{p}{q}{\msg}$, 
	executed by process $q$ when receiving $\msg$ from $p$.  
\end{itemize}

Furthermore, we write $\Act$ for the set $\Procs \times \Procs \times \{!,?\} \times \Msg$  
of all actions, and $\Actp$ for the subset of actions executable by $p$ 
(i.e., $\send{p}{q}{\msg}$ or $\recv{q}{p}{\msg}$).  
When processes are clear from the context, we abbreviate 
send and receive actions as $!\msg$ and $?\msg$, respectively.  

An \emph{event} $\event$ of a sequence of actions $w \in \Act^*$  
is an index $i \in \{1,\ldots,\length{w}\}$.  
It is a \emph{send event} (resp. \emph{receive event}) if $w[i]$  
is a send (resp. receive) action.  
We denote by $\sendeventsof{w}$ (resp. $\receiveeventsof{w}$)  
the set of send (resp. receive) events of $w$, and  
$\eventsof{w} = \sendeventsof{w} \cup \receiveeventsof{w}$.  
When all events are labelled with distinct actions, 
we identify an event with its action.  

\subsubsection*{Executions.}
An execution is a well-defined sequence of actions $e\in\Act^*$, where a
receive action is always preceded by a unique corresponding send action. % TODO : in che senso?

\bigskip

\begin{definition}[Execution]\label{def:execution}
An \emph{execution} over $\Procs$ and $\Msg$ is a sequence 
of actions $e \in \Act^*$ together with an injective mapping 
$\source_{e} : \receiveeventsof{e} \to \sendeventsof{e}$  
such that for each receive event $i$ labelled $\recv{p}{q}{\msg}$,  
its source $\source_{e}(i)$ is labelled $\send{p}{q}{\msg}$ and  
$\source_{e}(i) < i$.  
\end{definition}

For a set of executions $\mathcal{E}$, let 
$\prefixclosureof{\mathcal{E}}$ be the set of all prefixes 
of executions in $\mathcal{E}$.  
The \emph{projection} $\projofon{e}{p}$ of $e$ on process $p$  
is the subsequence of actions in $\Actp$.  
A send event $s$ is \emph{matched} if there exists a receive event $r$  
such that $\source(r) = s$.  
An execution is \emph{orphan-free} if all send events are matched,  
i.e., if $\source$ is surjective onto $\sendeventsof{e}$.  

\subsubsection*{Communication Models.}
In this thesis, we focus on two communication models: 
the peer-to-peer model ($\ppmodel$) and the synchronous model ($\synchmodel$).  
Nonetheless, this work forms part of a 
broader and more general project. Some results presented here 
naturally extend to a wide range of communication models, often requiring 
only mild additional assumptions. Please, refer to the related work chapter
(Chapter~\ref{sec:rel}, Section~\ref{sec:hier}).
From this perspective, we introduce a general definition of a communication 
model. 

\bigskip

\begin{definition}[Communication model]\label{def:communication-model}
A \emph{communication model} $\acommunicationmodel$  
is a set $\executionsofmodel{\acommunicationmodel}$ of executions.  
\end{definition}

In the \emph{synchronous model} $\synchmodel$,  
every send is immediately followed by its matching receive:  

\bigskip

\begin{definition}[$\synchmodel$]\label{def:synchronous}
An execution $e = (w,\source)$ belongs to 
$\executionsofmodel{\synchmodel}$ if for every send event 
$s \in \sendeventsof{e}$, the event $s+1$ is a receive event 
with $\source(s+1) = s$.  
\end{definition}

In the peer-to-peer communication model $\ppmodel$, every pair of 
processes $(p,q)$ is connected by a dedicated FIFO channel.  
Messages sent by $p$ to $q$ are delivered in the same order in which 
they were issued: if $p$ first sends $m_1$ and then $m_2$, the channel 
guarantees that $m_2$ cannot overtake $m_1$. Concretely:
\begin{itemize}
    \item if $m_1$ is never received, then $m_2$ cannot be received either;
    \item if both are received, then $m_1$ is delivered before $m_2$.
\end{itemize}

\bigskip

% TODO: togliere def, mettere versione intuitiva nei related
\begin{definition}[Peer-to-peer model $\ppmodel$]
\label{def:queue-based-communication-model}
The set $\executionsofmodel{\ppmodel}$ consists of all executions $e$ 
such that for any two send events 
$s_1=\send{p}{q}{\msg_1}$ and $s_2=\send{p}{q}{\msg_2}$ in 
$\sendeventsof{e}$, with $s_1 < s_2$, one of the following holds:
\begin{itemize}
    \item $s_2$ remains unmatched; or
    \item there exist receive events $r_1, r_2$ such that $r_1 < r_2$, 
          $\source(r_1)=s_1$, and $\source(r_2)=s_2$.
\end{itemize}
\end{definition}

It is easy to observe that 
$\executionsofmodel{\synchmodel} \subseteq \executionsofmodel{\ppmodel}$.  
Furthermore, the \emph{source function} $\source_e$ is defined as follows.  
If $e$ is an execution in $\ppmodel$, the source of the $i$-th receive 
event of $q$ from $p$ is the $i$-th corresponding send event of $p$ to $q$.  
If instead $e$ belongs to $\synchmodel$, then for every receive event $i$ 
we define $\source_e(i) = i-1$.  

\subsubsection*{Message Sequence Charts.}
While executions correspond to a total order of events in a system,  
message sequence charts (MSCs) provide a distributed view, using  
a partial order on events.  
For a tuple $\mmsc=(w_p)_{p\in\Procs}$, each $w_p \in \Actsp$ is a  
sequence of actions executed by process $p$, according to some  
total, locally observable order.  
We write $\eventsof{\mmsc}$ for the set  
$\{(p,i) \mid p \in \Procs \text{ and } 0 \leq i < \length{w_p}\}$.  
The label $\actionof{\event}$ of an event $\event=(p,i)$ is the action  
$w_p[i]$. The event $\event$ is a send (resp. receive) event if it is  
labelled with a send (resp. receive) action.  
We write $\sendeventsof{\mmsc}$ (resp. $\receiveeventsof{\mmsc}$)  
for the set of send (resp. receive) events of $\mmsc$. We also write  
$\messageof{\event}$ for the message sent or received at $\event$, and  
$\processof{\event}$ for the process executing $\event$.  
Finally, we write $\verticalorderstrict{\event_1}{\event_2}$ if there  
exists a process $p$ and indices $i<j$ such that  
$\event_1=(p,i)$ and $\event_2=(p,j)$.  

\bigskip

\begin{definition}[Message Sequence Chart]\label{def:msc}
	An \emph{MSC} over $\Procs$ and $\Msg$ is a tuple 
	$\mmsc = \big((w_p)_{p\in\Procs},\source\big)$ where
    \begin{enumerate}
        \item for each process $p$, $w_p\in\Actsp$ is a finite sequence 
			of actions;
		\item $\source : \receiveeventsof{\mmsc} \to \sendeventsof{\mmsc}$ is 
			an injective function from receive events to send events such that
			for all receive event $\event$ labelled with $\recv{p}{q}{\msg}$,
			$\source(\event)$ is labelled with $\send{p}{q}{\msg}$.
    \end{enumerate}
\end{definition}

For an execution $\execution$,  $\mscof{\execution}$ is the MSC 
$\big((w_p)_{p\in\Procs},\source\big)$ where $w_p$ is the 
subsequence of $\execution$ restricted to the actions of $p$,
and $\source$ is the lifting of $\source_{\execution}$ to the
events of $(w_p)_{p\in\Procs}$.

\bigskip

\begin{example}\label{exmp:msc}
Consider the MSC depicted in Figure~\ref{fig:msc-exmp}.  
It consists of $\Procs=\{p,q,r\}$ and $\Msg=\{\msg_1,\msg_2,\msg_3\}$
with $\mmsc=((w_p,w_q,w_r),\source)$, where
$w_p = {}!\msg_1{}?\msg_2$, 
$w_q = {}?\msg_1{}!\msg_2{}!\msg_3$, 
$w_r = {}?\msg_3$,
$\source((p,2)) = (q,2)$,
$\source((q,1)) = (p,1)$, and
$\source((r,1)) = (q,3)$.

\begin{figure}[!ht]
\centering
\begin{msc}[draw frame=none, draw head=none, msc keyword=, head height=0px, label distance=0.5ex, foot height=0px, foot distance=0px]{}
	\declinst{p}{p}{}
	\declinst{q}{q}{}
	\declinst{r}{r}{}

	\mess{$m_1$}{p}{q}
	\nextlevel
	\mess{$m_2$}{q}{p}
	\nextlevel
	\mess{$m_3$}{q}{r}
\end{msc}
\caption{Simple example with an exchange of three messages.}
\label{fig:msc-exmp}
\end{figure}
\end{example}

Given a set of processes $\Procs$, an MSC 
$M=\big((w_p)_{p\in\Procs},\source\big)$ is said to be a 
\emph{prefix} of another MSC 
$M'=\big((w'_p)_{p\in\Procs},\source'\big)$, denoted by 
$M \prefixorder M'$, if the following conditions hold:  
\begin{itemize}
    \item for every $p \in \Procs$, the sequence $w_p$ is a prefix 
    of $w'_p$;  
    \item for every receive event $e$ of $M$, it holds that 
    $\source'(e)=\source(e)$.  
\end{itemize}

The \emph{concatenation} of two MSCs $M_1$ and $M_2$ is the MSC 
$M_1 \cdot M_2$ obtained by stacking $M_1$ vertically above $M_2$. 
Formally, let 
$M_1=((w_p^1)_{p\in\Procs},\source_1)$ and 
$M_2=((w_p^2)_{p\in\Procs},\source_2)$. Then:  
\begin{inparaenum}[(i)]
   \item for each process $p$, the sequence is 
   $w_p = w_p^1 \cdot w_p^2$;  
   \item the source function $\source$ is defined so that 
   $\source(e)=\source_i(e)$ for all receive events $e$ belonging 
   to $M_i$, with $i\in\{1,2\}$.  
\end{inparaenum}


\subsubsection*{Happens-before relation and linearisations}
In a given MSC $M$, an event $\event$ happens before $\event'$, if 
 \begin{itemize}
	\item $\event$ and $\event'$ are events
	of a same process $p$ and happen in that order on 
	the timeline of $p$
	\item  $\event$ is send event matched by $\event'$
	\item a sequence of such situations defines
	a path from $\event$ to $\event'$
\end{itemize}

\bigskip

\begin{definition}[Happens-before relation]
Let $\mmsc$ be an MSC. The happens-before relation over $M$
is the binary relation $\happensbeforestrict$ defined as 
the least transitive relation over $\eventsof{\mmsc}$ such that:
\begin{itemize}
   \item for all 
   $p,i,j$, if $i<j$, then $(p,i)\happensbeforestrict (p,j)$, and
   \item for all receive events 
   $\event$, $\source(\event) \happensbeforestrict \event$.
\end{itemize}
\end{definition}

\bigskip

\begin{example}
Consider the Example~\ref{exmp:msc}. 
The following happens-before relations are valid:
$${}!\msg_1{}\happensbeforestrict {}?\msg_1 \happensbeforestrict {}!\msg_2
\happensbeforestrict {}!\msg_3 \happensbeforestrict {}?\msg_3$$
and
$${}!\msg_1{}\happensbeforestrict {}?\msg_1 \happensbeforestrict {}!\msg_2
\happensbeforestrict {}?\msg_2.$$
\end{example}

\bigskip

\begin{definition}[Linearisation]\label{def:linearisation}
	A \emph{linearisation} of an MSC $\mmsc$ is a
	total order $\alinearisation$ on $\eventsof{\mmsc}$
	that refines $\happensbeforestrict$:  for all events $\event,\event'$, 
	if $\event\happensbeforestrict \event'$, then $\event\alinearisation \event'$. 
\end{definition}
We write $\linearisationsof{\mmsc}{}$ for the set of all linearisations
of $\mmsc$. 
We often identify a linearisation with the execution it induces.

\bigskip

\begin{example}\label{exmp:lin}
Considering the Example~\ref{exmp:msc},	
let $\mmsc$ be the MSC in Figure~\ref{fig:msc-exmp}. 
The elements of the set $\linearisationsof{\mmsc}{}$ are
$$!m_1?m_1!m_2?m_2!m_3?m_3,$$
$$!m_1?m_1!m_2!m_3?m_2?m_3,$$
$$!m_1?m_1!m_2!m_3?m_3?m_2.$$
\end{example}

Given an MSC $\mmsc$, we write 
$\linearisationsof{\mmsc}{\acommunicationmodel}$ to denote
$\linearisationsof{\mmsc}{}\cap\executionsofmodel{\acommunicationmodel}$;
the executions of $\linearisationsof{\mmsc}{\acommunicationmodel}$ 
are called the linearisations of $\mmsc$ in the communication 
model \verb|com|.

\bigskip

\begin{definition}[$\acommunicationmodel$-linearisable MSC]
	\label{def:linearisable-msc}
	An MSC $\mmsc$ is \textit{linearisable} in a communication 
	model $\acommunicationmodel$
	if $\linearisationsof{\mmsc}{\acommunicationmodel}\neq\emptyset$.
	We write $\mscsetofmodel{\acommunicationmodel}$ for the set of all MSCs 
	linearisable in $\acommunicationmodel$.
\end{definition}

\bigskip

\begin{example}
Consider the Example~\ref{exmp:msc} and the respective linearisation
listed in Example~\ref{exmp:lin}. 
The MSC $\mmsc$ is \textit{linearisable} in the $\synchmodel$ communication 
model because
$\linearisationsof{\mmsc}{\synchmodel}\neq\emptyset$.
The only element of $\linearisationsof{\mmsc}{\synchmodel}$ is
$$!m_1?m_1!m_2?m_2!m_3?m_3.$$
All the send events are followed by the respective
receive events.
\end{example}

%% TODO: non so se sono utili
% Finally, we introduce a property that will be helpful in the next paragraph 
% for giving an alternative characterisation of deadlock-freedom of a system 
% of communicating finite state machines.

% \begin{definition}[Causally-closed communication model]\label{def:causally-closed-communication-model}
% 	A communication model $\acommunicationmodel$ is \emph{causally-closed} if for all MSCs $M$,
% 	$\linearisationsof{\mmsc}{\acommunicationmodel}\neq\emptyset$ implies that
% 	$\linearisationsof{\mmsc}{\acommunicationmodel}=\linearisationsof{\mmsc}{}$.
% \end{definition}

  
% %   \input{proofs/lem-pp-is-causally-closed.tex}
%   %The proof of the following Lemma is in Appendix \ref{app:pp-is-causally-closed}.


\subsubsection*{Communicating finite state machines.}
We recall the definition of communicating finite state machines~\cite{BrandZafiropulo}.
These definitions will later be used also for Chapter~\ref{sec:rescu}.

\bigskip

\begin{definition}[CFSM]\label{def:cfsm}
    A communicating finite state machine (CFSM) is an NFA 
	with $\varepsilon$-transitions $\acfsm$ over the alphabet $\Act$.
    A system of CFSMs is a tuple $\cfsms = (\acfsm_p)_{p\in\Procs}$.
\end{definition}

Given a system of CFSMs $\cfsms=(\acfsm_p)_{p\in\Procs}$,
we write $\acceptcompletion{\cfsms}{}$ for the system of CFSMs 
$\acceptcompletion{\cfsms}=(\acceptcompletion{\acfsm_p})_{p\in\Procs}$
where all states are accepting, i.e., $F_p = Q_p$.

\bigskip

% Let's now define the asynchronous product of an automaton.
% TODO: capire questa definizione di prodotto ASYNC(che serve per la prova)
% capire se c'è differenza con quello SYNC (che serve per il tool)
% \begin{definition}[Asynchronous product automaton]
% Let $\{A_i\}_{i=1}^n$ be a set of automata. For each ordered pair 
% $(i,j)$ of process indices, we use two buffers: $B^s_{i,j}$ 
% (pending messages sent by $P_i$ but not yet accessible by $P_j$) 
% and $B^r_{i,j}$ (messages delivered to $P_j$ but not yet consumed). 
% All buffers are words over the alphabet $\Sigma$.

% The asynchronous product automaton 
% $A = \prod_{i=1}^n A_i$ over $\hat{\Sigma}$ is given by: % TODO: capire cos'è \hat{\Sigma}

% \begin{itemize}
%   \item \textbf{States:} A state $q$ consists of the local states 
%   $q_i$ of each $A_i$, together with the buffer contents.
%   \item \textbf{Initial state:} $q_0$ has all components in their 
%   start states $q_i^0$ and all buffers empty.
%   \item \textbf{Transitions:} $\delta \subseteq Q \times 
%   (\hat{\Sigma} \cup \{\tau\}) \times Q$.
%     \begin{enumerate}
%       \item For $x \in \hat{\Sigma}_i$, $(q,x,q') \in \delta$ iff:  
%       (a) states of all other processes $k \neq i$ are unchanged,  
%       (b) $(q_i,x,q'_i) \in \delta_i$,  
% 		% TODO: capire semantica dei buffer dal paper di Alur
%       (c) if $x = receive(j,i,a)$ then the buffer $B^r_{j,i}$ consume the message $a$ (if present),  
%       (d) if $x = send(i,j,a)$ then the message $a$ is appended in the buffer $B^s_{i,j}$,  
%       (e) all other buffers remain unchanged.
%       \item There is a $\tau$-transition $(q,\tau,q')$ if $q$ and 
%       $q'$ differ only in that one buffer $B^s_{i,j}$ loses its 
%       head symbol $a$, and $B^r_{i,j}$ appends $a$.
%     \end{enumerate}
%   \item \textbf{Accepting states:} $q$ is accepting if all $q_i$ 
%   are accepting and all buffers are empty.
% \end{itemize}

% The language $L(A)$ of $A$ consists of all words in $\hat{\Sigma}^*$ 
% taking $q_0$ to an accepting state, interpreting $\tau$ as 
% $\varepsilon$. For any set of automata $\{A_i\}$, the language 
% $L(\prod_i A_i)$ contains only complete, well-formed words. For a 
% given MSC $M$, $L(\prod_i A_i)$ either contains all linearisations 
% of $M$, or none.
% \end{definition}

% \bigskip

\begin{definition}[Executions of CFSMs in a model]
\label{def:executions-of-cfsms}
Given a system $\cfsms=(\acfsm_p)_{p\in\Procs}$ and a 
model $\acommunicationmodel$,  
$\executionsofcfsms{\cfsms}{\acommunicationmodel}$ is the set of 
executions $e \in \executionsofmodel{\acommunicationmodel}$ such that  
$\projofon{e}{p}\in\languageofnfa{\acfsm_p}$ for all $p$.  
\end{definition}

% \begin{remark}
% 	Let $\acommunicationmodel$ be a communication model, 
% 	$\cfsms$ a syste\m of CFSMs, and $e,e'\in\executionsofmodel{acommunicationmodel}$ 
%   such that $\mscof{e}=\mscof{e'}$, 
% 	then $e\in\executionsof{\cfsms}{\acommunicationmodel}$ 
% 	if and only if $e'\in\executionsof{\cfsms}{\acommunicationmodel}$. 
%   This follows from the fact that $\projofon{e}{p}= \projofon{e'}{p}$ for all $p$.
% \end{remark}

We write $\mscsofcfsms{\cfsms}{\acommunicationmodel}$ for the set 
$\{\mscof{e} \mid e\in\executionsofcfsms{\cfsms}{\acommunicationmodel}\}$.

A system is orphan-free if, whenever all machines have reached 
an accepting state, no message
remains in transit, i.e., no message is sent but not received.

\bigskip

\begin{definition}[Orphan-free]\label{def:orphan-free}
A system $\cfsms$ is \emph{orphan-free} in a model $\acommunicationmodel$  
if all its executions in $\executionsofcfsms{\cfsms}{\acommunicationmodel}$  
are orphan-free.  
\end{definition}

All synchronous executions are orphan-free by definition.

A system is deadlock-free if, 
any \emph{partial} execution can be extended/completed to an accepting execution.

\bigskip

\begin{definition}[Deadlock-free]\label{def:deadlock-free}
A system $\cfsms$ is \emph{deadlock-free} in $\acommunicationmodel$  
if for every execution 
$e \in \executionsofcfsms{\acceptcompletion{\cfsms}}{\acommunicationmodel}$,  
there exists a completion $e'$ with $e \prefixorder e'$ and  
$e' \in \executionsofcfsms{\cfsms}{\acommunicationmodel}$.  
\end{definition}

% \begin{remark}\label{rem:equivalent-formulation-of-deadlock-free-cfsms}
% 	A system of CFSMs $\cfsms$ is \emph{deadlock-free} for a communication model
% 	$\acommunicationmodel$ if and only if 
% 	$$
% 	\executionsofcfsms{\acceptcompletion{\cfsms}}{\acommunicationmodel}\subseteq
% 	\prefixclosureof{\executionsofcfsms{\cfsms}{\acommunicationmodel}}
% 	$$
% \end{remark}

% TODO: capire se serve
The following result shows that, for either $\ppmodel$ or 
$\synchmodel$ communication models, the deadlock-freedom property 
of a system of CFSMs can be expressed as a property on the MSCs
of the system. 
%The proof can be found in Appendix \ref{app:deadlock-free-as-a-property-on-mscs-for-p2p-and-synch}.

\bigskip

\begin{lemma}[Deadlock-freedom as an MSC property]\label{lem:ddlckmsc}
	\label{prop:deadlock-free-as-a-property-on-mscs-for-p2p-and-synch}
	Assume $\acommunicationmodel$ is $\ppmodel$ (respectively, $\acommunicationmodel$ is $\synchmodel$).
	Then a system of CFSMs $\cfsms$ is deadlock-free for $\acommunicationmodel$ if and only if
	$$
	\mscsofcfsms{\acceptcompletion\cfsms}{\acommunicationmodel}
	\subseteq
	\prefixclosureof{\mscsofcfsms{\cfsms}{\acommunicationmodel}}.
	$$
\end{lemma}

The proof for Lemma~\ref{lem:ddlckmsc} can be found in \cite[Proposition 2.1]{di2025realisability}.

% TODO SPIEGARE PERCHE' FUNZIONA
% %\input{proofs/prop-deadlock-free-as-a-property-on-mscs-for-p2p-and-synch.tex}

% %\iflong
% %\include{proofs/prop-deadlock-free-as-a-property-on-mscs-for-p2p-and-synch}
% %\else
% %The proof of Proposition~\ref{prop:deadlock-free-as-a-property-on-mscs-for-p2p-and-synch} can be found in Appendix~\ref{app:deadlock-free-as-a-property-on-mscs-for-p2p-and-synch}.
% %\fi

\section{Global Types}
This part will further highlight the basic notions to understand the formal proof 
for the theorem presented in Chapter~\ref{sec:proof} and, in particular, Global Type
and Weakly-realisable. We begin by extending the definition of linearisability so 
that it applies to all communication models.
In our setting, Global Types are automata that describe a language of MSCs, 
as considered in this recent work by Di Giusto, et al.~\cite{di2025realisability}.

%% TODO: commenti di ivan: non è proprio un type, lui definisce 
%% così i choreography automata , magari specificare che le definizioni corrispondono
%% e citare il paper che usa questa nozione

%% !! CA da citare nei related 

\bigskip

\begin{definition}[Global Type]
	An \emph{arrow} is a triple $(p,q,m)\in\Procs\times\Procs\times\Messages$ 
	with $p\ne q$; we often write $\marrow{p}{q}{m}$ instead of $(p,q,m)$, and 
	write $\labelalphabet$ to denote the finite set of arrows.
	A Global Type $\gt$ is a deterministic finite state automaton over the 
	alphabet $\labelalphabet$.
\end{definition}
% TODO: dare più contesto ad esempio

We use the notation $p \xleftrightarrow{m} q$ to denote the
\emph{round-trip} exchange of a message $m$: first $p$ sends $m$
to $q$, and then $q$ sends back the same message $m$ to $p$.
This will serve as an acknowledgment message for $p$.

\bigskip

\begin{example}
An example of a Global Type expressed as an automaton is the following.
Consider the not-implementable specification stated in 
Listing~\ref{lst:not-impl-exm}. The protocol can be modelled with the 
Global Type in Figure~\ref{fig:gt-exm}.

\begin{figure}[!ht]
	\centering
	\begin{tikzpicture}[node distance=1.5cm, auto, scale=0.8]
		\node[state, initial, initial text={}] (s0) {1};
		\node[state] (s1) [above right=of s0] {2};
		\node[state] (s2) [below right=of s0] {3};
		\node[state,accepting] (s3) [right=of s1] {4};
		\node[state,accepting] (s4) [right=of s2] {5};

		\draw[->] (s0) to node[above,sloped]{$\gtlabel{A}{B}{x}$}(s1);
		\draw[->] (s0) to node[above,sloped]{$\gtlabel{A}{B}{y}$}(s2);
		\draw[->] (s1) to node[above,sloped]{$\gtlabel{C}{D}{z}$}(s3);
		\draw[->] (s2) to node[above,sloped]{$\gtlabel{C}{D}{w}$}(s4);
	\end{tikzpicture}
	\caption{An automaton representing the specification's global type given in Listing~\ref{lst:not-impl-exm}.}
	\label{fig:gt-exm}
\end{figure}
\end{example}

\bigskip

% --------------relationship between MSCs and Global Types--------------

I can now formally define the relationship between MSCs and Global Types. 
Intuitively, Global Types represent a set of MSCs, allowing us to reason 
about multiple message sequence scenarios. 

A Global Type defines a language of MSCs in two different ways, one
existential and one universal. Let $\labellanguageof{\gt}$ be the set of
sequences of arrows $w$ accepted by $\gt$. Note that for $w \in \Arrows^*$,
the function $w \mapsto \labeltomsc{w}$ with
$\labeltomsc{w} \in \mscsetofmodel{\synchmodel}$ is not injective, as two
arrows with disjoint pairs of processes commute. We write $w_1 \sim w_2$ if
$\labeltomsc{w_1} = \labeltomsc{w_2}$, and $[w]$ for the equivalence class
of $w$ with respect to $\sim$.

Informally, the existential MSC language $\existentialmsclanguageof{\gt}{}$ of a 
Global Type $\gt$ is the set of MSCs that admit at least one representation as a
sequence of arrows in $\labellanguageof{\gt}$, and the universal MSC
language $\universalmsclanguageof{\gt}$ of a Global Type $\gt$ is the set of
MSCs whose representations as sequences of arrows are all in
$\labellanguageof{\gt}$:
% TODO: ivan dice di controllare? la def di chor automata (o forse hanno una roba simile)

\bigskip

\begin{definition}[$\existentialmsclanguageof{\gt}$]
	$$
		\existentialmsclanguageof{\gt} \eqdef \{\labeltomsc{w} \mid
		w \in \labellanguageof{\gt}\}
  $$
\end{definition}

\bigskip

\begin{definition}[$\universalmsclanguageof{\gt}$]
  $$
		\universalmsclanguageof{\gt} \eqdef \{\labeltomsc{w} \mid
		[w] \subseteq \labellanguageof{\gt}\}.
	$$
\end{definition}
% TODO: dare più contesto ed esempio

\bigskip

\begin{definition}[Commutation-closed]
    A global type $\gt$ is \emph{commutation-closed} if
    $$
    \existentialmsclanguageof{\gt} = \universalmsclanguageof{\gt}.
    $$
\end{definition}
% TODO: togliere exist e forall dalla nozione
% We write $\msclanguageofcc{\gt}$ for the common language. % TODO: mi serve?

I now introduce the definition of implementability following the one given 
by Alur, et al. \cite{alur2005realizability}, referred to as 
\textit{Weak-realisability}.
To formalize it, we first define the notions of 
\textit{weak implication} and \textit{weak closure}.

%% TODO: Spiegare meglio con intuizioni

\bigskip

\begin{definition}[Weakly-imply]
	Let $\setmsc$ be a set of MSCs and $M$ an MSC. $\setmsc$
	\textit{weakly implies} $M$, if for any sequence of automata
	$\langle A_i \ |\ 1\leq i\leq n\rangle$, if every MSC in $\setmsc$ is in
	$L(\prod_i A_i)$ then $M \in L(\prod_i A_i)$.
\end{definition}

In order to understand the meaning of \emph{weak implication},
consider the following example.

\bigskip

\begin{example}
Define two MSCs, MSC1 and MSC2. Both perform the same four
communications, but in different orders.  
MSC1 first sends message $a$ from P1 to P2, then from P1 to P3, 
then sends $b$ from P4 to P2, and finally from P4 to P3.  
MSC2 instead starts with P4 sending $b$ to P2, then to P3, 
followed by P1 sending $a$ to P2 and then to P3.  

Now define a third MSC $M$ with the same four messages but in a
different order: P1 sends $a$ to P2, P4 sends $b$ to P2, P4 sends 
$b$ to P3, and finally P1 sends $a$ to P3.  

We say $M$ is weakly implied by MSC1 and MSC2. Indeed, by looking
at each process projection we recover the same behaviour in one of
them: for P1 and P4 in both, for P2 in MSC1, and for P3 in MSC2.  
Figure~\ref{fig:weak-impl} illustrates the three MSCs.

\begin{figure}[!ht]
\centering
\begin{tabular}{ccc}
\begin{minipage}{0.32\textwidth}
\scalebox{0.55}{%
\begin{msc}[left environment distance=0cm, draw frame=none, draw head=none, msc keyword=, head height=0px, label distance=0.5ex, foot height=0px, foot distance=0px]{}
	\declinst{P1}{P1}{}
	\declinst{P2}{P2}{}
	\declinst{P3}{P3}{}
	\declinst{P4}{P4}{}

	\mess{a}{P1}{P2}
	\nextlevel
	\mess[pos=0.25]{a}{P1}{P3}
	\nextlevel
	\nextlevel
	\mess[pos=0.25]{b}{P4}{P2}
	\nextlevel
	\mess{b}{P4}{P3}
\end{msc}
} 
\end{minipage}
&
\begin{minipage}{0.32\textwidth}
\scalebox{0.55}{%
\begin{msc}[left environment distance=0cm, draw frame=none, draw head=none, msc keyword=, head height=0px, label distance=0.5ex, foot height=0px, foot distance=0px]{}
	\declinst{P1}{P1}{}
	\declinst{P2}{P2}{}
	\declinst{P3}{P3}{}
	\declinst{P4}{P4}{}

	\mess[pos=0.25]{b}{P4}{P2}
	\nextlevel
	\mess{b}{P4}{P3}
	\nextlevel
	\nextlevel
	\mess{a}{P1}{P2}
	\nextlevel
	\mess[pos=0.25]{a}{P1}{P3}
\end{msc}
}
\end{minipage}
&
\begin{minipage}{0.32\textwidth}
\scalebox{0.55}{%
\begin{msc}[left environment distance=0cm, draw frame=none, draw head=none, msc keyword=, head height=0px, label distance=0.5ex, foot height=0px, foot distance=0px]{}
	\declinst{P1}{P1}{}
	\declinst{P2}{P2}{}
	\declinst{P3}{P3}{}
	\declinst{P4}{P4}{}

	\mess{a}{P1}{P2}
	\nextlevel
	\mess[pos=0.25]{b}{P4}{P2}
	\nextlevel
	\nextlevel
	\mess{b}{P4}{P3}
	\nextlevel
	\mess[pos=0.25]{a}{P1}{P3}
\end{msc}
}
\end{minipage} \\
MSC1 & MSC2 & $M$
\end{tabular}
\caption{The MSC $M$ is weakly implied by MSC1 and MSC2}
\label{fig:weak-impl}
\end{figure}
\end{example}

\bigskip

\begin{definition}[Weakly-closure $\setmsc^w$]
	The weak-closure $\setmsc^w$ of a set $\setmsc$ of MSCs contains all the MSCs
	weakly implied by $\setmsc$.
\end{definition}

% todo: esempio

\bigskip

\begin{definition}[Weakly-realisable]
	An MSC $M$ is said to be weakly-realisable if the set of MSCs
	$L(M)$ is weakly realisable. A set of MSCs $\setmsc$ is said to be weakly
	realisable if $\setmsc=\setmsc^w$.
\end{definition}
It is important to note that this definition does not include the
property of deadlock-freedom. 

% todo: esempio

% CINZIA, X TESI:
% sarebbe interessante mettere la definizione di PPDP e mostrare 
% l'equivalenza tra le due

% VERSIONE ALTERNATIVA VECCHIA
% \begin{definition}[Projection]
% For an MSC $M$ and a process $P_i$, let $M|_i$ denote the sequence of events be
% belonging to the process $P_i$ in $M$.
% \end{definition}
% \begin{definition}[Weakly-imply]
% The set of MSCs $\setmsc$ weakly implies an MSC $M$ iff for all $1\leq i\leq n$, there
% exist an MSC $M_i\in L$ such that $M|_i=M_i|_i$
% \end{definition}
% \begin{definition}[Weakly-realisable]
% ? TODO ?
% \end{definition}

We are now ready to present the main contributions of this work.
