\chapter{Preliminaries}\label{sec:pre}
In this section, the fundamental concepts and definitions necessary 
to contextualize the main contributions of this work are presented. 
We, first, introduce automaton, executions, and
Message Sequence Charts (MSC), followed by 
an examination of communication model's semantics that are particularly 
interesting. 
Then, the notions of Global Type and Realisability are 
defined within the scope of this work, along with the foundational 
elements required to understand the theoretical contributions.

\section{Standard notions on automata}

\textbf{Words and alphabet.} $\Sigma$ is used to denote a set of finite words.
$\labelalphabet$ denotes a set of finite action used for a global type.
For a string $s$, let $s^l$ denote the $l$-th character of the string.

\bigskip

\begin{definition}[NFA]\label{def:nfa}
A non-deterministic finite automaton (NFA) is a tuple  
$\mathcal{A} = (Q, \Sigma, \delta, q_0, F)$, where $Q$ is a finite set of  
states, $\Sigma$ is a finite alphabet,  
$\delta : Q \times (\Sigma \cup \{\varepsilon\}) \to Q$ is the transition  
relation, $q_0 \in Q$ is the initial state, and $F \subseteq Q$ is the set  
of accepting states.  

We write $\delta^*(s,w)$ to denote the set of states $s'$ reachable from  
$s$ along a path labelled with $w$. The language accepted by $\mathcal{A}$,  
denoted $\mathcal{L}_{\text{words}}(\mathcal{A})$, is the set of words  
$w \in \Sigma^*$ such that $\delta^*(q_0,w) \cap F \neq \emptyset$.  
\end{definition}

\bigskip

\begin{definition}[DFA]\label{def:dfa}
A deterministic finite automaton (DFA) is an NFA where the transition  
relation $\delta$ is a partial function  
$\delta : Q \times \Sigma \to Q$. The DFA is complete if $\delta$ is  
total.  
\end{definition}

\bigskip

\begin{definition}[Determinization]\label{def:det}
To every NFA $\mathcal{A} = (Q, \Sigma, \delta, q_0, F)$, we associate  
the DFA $\text{det}(\mathcal{A}) = (Q', \Sigma, \delta', q'_0, F')$,  
where $Q' = 2^Q$, $q'_0 = \{q_0\}$, $F'$ is the set of subsets of $Q$  
that contain at least one accepting state, and $\delta'$ is defined by  
$\delta'(S,a) = \bigcup\{\delta^*(s,a) \mid s \in S\}$ for all  
$S \in Q'$, $a \in \Sigma$.  
\end{definition}

We write $\acceptcompletion{\mathcal A}{}$ for the automaton obtained
from $\mathcal A$ by setting $F=Q$.

% \begin{definition}[Product NFA]\label{def:product-nfa}
% Given two NFAs $\mathcal{A}_1 = (Q_1, \Sigma, \delta_1, q_{0,1}, F_1)$ and  
% $\mathcal{A}_2 = (Q_2, \Sigma, \delta_2, q_{0,2}, F_2)$, the product NFA is  
% defined as  
% \[
% \mathcal{A}_1 \otimes \mathcal{A}_2 \stackrel{\text{def}}{=}  
% (Q_1 \times Q_2, \Sigma, \delta, (q_{0,1}, q_{0,2}), F_1 \times F_2),
% \]  
% where the transition relation $\delta$ is defined as follows:  

% \begin{itemize}  
%   \item for all $(s_1,s_2) \in Q_1 \times Q_2$, $a \in \Sigma$,  
%   $(s'_1,s'_2) \in Q_1 \times Q_2$:  
%   $((s_1,s_2),a,(s'_1,s'_2)) \in \delta$  
%   iff $(s_1,a,s'_1) \in \delta_1$ and $(s_2,a,s'_2) \in \delta_2$.  

%   \item for all $(s_1,s_2) \in Q_1 \times Q_2$, $(s'_1,s'_2) \in Q_1 \times Q_2$:  
%   $((s_1,s_2),\varepsilon,(s'_1,s'_2)) \in \delta$  
%   iff $(s_1,\varepsilon,s'_1) \in \delta_1$ or $(s_2,\varepsilon,s'_2) \in \delta_2$.  
% \end{itemize}  

% It holds that  
% $\mathcal{L}_{\text{words}}(\mathcal{A}_1 \otimes \mathcal{A}_2) =  
% \mathcal{L}_{\text{words}}(\mathcal{A}_1) \cap  
% \mathcal{L}_{\text{words}}(\mathcal{A}_2)$.  
% \end{definition}

% \begin{definition}[Dual DFA]\label{def:dual-dfa}
% If $\mathcal{A} = (Q, \Sigma, \delta, q_0, F)$ is a DFA, its dual is  
% defined as  
% \[
% \mathcal{A}^{\text{dual}} = (Q, \Sigma, \delta, q_0, Q \setminus F).
% \]  
% If $\mathcal{A}$ is complete, then  
% $\mathcal{L}_{\text{words}}(\mathcal{A}^{\text{dual}}) =  
% \Sigma^* \setminus \mathcal{L}_{\text{words}}(\mathcal{A})$.  
% \end{definition}

\section{Execution, Communication Models and MSC}
We assume a finite set of \emph{processes} 
$\Procs=\{p,q,\ldots,P1,P2,\ldots\}$ and a finite set of 
messages (labels) $\Msg=\{\msg_1,\msg_2,\ldots\}$.  
We consider two kinds of actions:  
\begin{itemize}
	\item \emph{send actions}, of the form $\send{p}{q}{\msg}$, 
	executed by process $p$ when sending message $\msg$ to $q$;  
	\item \emph{receive actions}, of the form $\recv{p}{q}{\msg}$, 
	executed by process $q$ when receiving $\msg$ from $p$.  
\end{itemize}

Furthermore, we write $\Act$ for the set $\Procs \times \Procs \times \{!,?\} \times \Msg$  
of all actions, and $\Actp$ for the subset of actions executable by $p$ 
(i.e., $\send{p}{q}{\msg}$ or $\recv{q}{p}{\msg}$).  
When processes are clear from the context, we abbreviate 
send and receive actions as $!\msg$ and $?\msg$, respectively.  

An \emph{event} $\event$ of a sequence of actions $w \in \Act^*$  
is an index $i \in \{1,\ldots,\length{w}\}$.  
It is a \emph{send event} (resp. \emph{receive event}) if $w[i]$  
is a send (resp. receive) action.  
We denote by $\sendeventsof{w}$ (resp. $\receiveeventsof{w}$)  
the set of send (resp. receive) events of $w$, and  
$\eventsof{w} = \sendeventsof{w} \cup \receiveeventsof{w}$.  
When all events are labelled with distinct actions, 
we identify an event with its action.  

\subsubsection*{Executions.}
An execution is a well-defined sequence of actions $e\in\Act^*$, where a
receive action is always preceded by a unique corresponding send action. % TODO : in che senso?

\bigskip

\begin{definition}[Execution]\label{def:execution}
An \emph{execution} over $\Procs$ and $\Msg$ is a sequence 
of actions $e \in \Act^*$ together with an injective mapping 
$\source_{e} : \receiveeventsof{e} \to \sendeventsof{e}$  
such that for each receive event $i$ labelled $\recv{p}{q}{\msg}$,  
its source $\source_{e}(i)$ is labelled $\send{p}{q}{\msg}$ and  
$\source_{e}(i) < i$.  
\end{definition}

For a set of executions $\mathcal{E}$, let 
$\prefixclosureof{\mathcal{E}}$ be the set of all prefixes 
of executions in $\mathcal{E}$.  
The \emph{projection} $\projofon{e}{p}$ of $e$ on process $p$  
is the subsequence of actions in $\Actp$.  
A send event $s$ is \emph{matched} if there exists a receive event $r$  
such that $\source(r) = s$.  
An execution is \emph{orphan-free} if all send events are matched,  
i.e., if $\source$ is surjective onto $\sendeventsof{e}$.  

\subsubsection*{Communication Models.}
In this thesis, we focus on a communication model: 
the synchronous model ($\synchmodel$).  
Nonetheless, this work forms part of a 
broader and more general project. Some results presented here 
naturally extend to a wide range of communication models, often requiring 
only mild additional assumptions. Please, refer to the related work chapter
(Chapter~\ref{sec:rel}, Section~\ref{sec:hier}).
From this perspective, we introduce a general definition of a communication 
model. 

\bigskip

\begin{definition}[Communication model]\label{def:communication-model}
A \emph{communication model} $\acommunicationmodel$  
is a set $\executionsofmodel{\acommunicationmodel}$ of executions.  
\end{definition}

In the \emph{synchronous model} $\synchmodel$,  
every send is immediately followed by its matching receive:  

\bigskip

\begin{definition}[$\synchmodel$]\label{def:synchronous}
An execution $e = (w,\source)$ belongs to 
$\executionsofmodel{\synchmodel}$ if for every send event 
$s \in \sendeventsof{e}$, the event $s+1$ is a receive event 
with $\source(s+1) = s$.  
\end{definition}
 
Furthermore, the \emph{source function} $\source_e$ is defined as follows.  

\bigskip

\begin{definition}[$\source$ function for $\synchmodel$]\label{def:src}
If $e$ is an execution in $\synchmodel$, then for every receive event $i$ 
we define $\source_e(i) = i-1$.
\end{definition}


\subsubsection*{Message Sequence Charts.}
While executions correspond to a total order of events in a system,  
message sequence charts (MSCs) provide a distributed view, using  
a partial order on events.  
For a tuple $\mmsc=(w_p)_{p\in\Procs}$, each $w_p \in \Actsp$ is a  
sequence of actions executed by process $p$, according to some  
total, locally observable order.  
We write $\eventsof{\mmsc}$ for the set  
$\{(p,i) \mid p \in \Procs \text{ and } 0 \leq i < \length{w_p}\}$.  
The label $\actionof{\event}$ of an event $\event=(p,i)$ is the action  
$w_p[i]$. The event $\event$ is a send (resp. receive) event if it is  
labelled with a send (resp. receive) action.  
We write $\sendeventsof{\mmsc}$ (resp. $\receiveeventsof{\mmsc}$)  
for the set of send (resp. receive) events of $\mmsc$. We also write  
$\messageof{\event}$ for the message sent or received at $\event$, and  
$\processof{\event}$ for the process executing $\event$.  
Finally, we write $\verticalorderstrict{\event_1}{\event_2}$ if there  
exists a process $p$ and indices $i<j$ such that  
$\event_1=(p,i)$ and $\event_2=(p,j)$.  

\bigskip

\begin{definition}[Message Sequence Chart]\label{def:msc}
	An \emph{MSC} over $\Procs$ and $\Msg$ is a tuple 
	$\mmsc = \big((w_p)_{p\in\Procs},\source\big)$ where
    \begin{enumerate}
        \item for each process $p$, $w_p\in\Actsp$ is a finite sequence 
			of actions;
		\item $\source : \receiveeventsof{\mmsc} \to \sendeventsof{\mmsc}$ is 
			an injective function from receive events to send events such that
			for all receive event $\event$ labelled with $\recv{p}{q}{\msg}$,
			$\source(\event)$ is labelled with $\send{p}{q}{\msg}$.
    \end{enumerate}
\end{definition}

For an execution $\execution$,  $\mscof{\execution}$ is the MSC 
$\big((w_p)_{p\in\Procs},\source\big)$ where $w_p$ is the 
subsequence of $\execution$ restricted to the actions of $p$,
and $\source$ is the lifting of $\source_{\execution}$ to the
events of $(w_p)_{p\in\Procs}$.

\bigskip

\begin{example}\label{exmp:msc}
Consider the MSC depicted in Figure~\ref{fig:msc-exmp}.  
It consists of $\Procs=\{p,q,r\}$ and $\Msg=\{\msg_1,\msg_2,\msg_3\}$
with $\mmsc=((w_p,w_q,w_r),\source)$, where
$w_p = {}!\msg_1{}?\msg_2$, 
$w_q = {}?\msg_1{}!\msg_2{}!\msg_3$, 
$w_r = {}?\msg_3$,
$\source((p,2)) = (q,2)$,
$\source((q,1)) = (p,1)$, and
$\source((r,1)) = (q,3)$.

\begin{figure}[!ht]
\centering
\begin{msc}[draw frame=none, draw head=none, msc keyword=, head height=0px, label distance=0.5ex, foot height=0px, foot distance=0px]{}
	\declinst{p}{p}{}
	\declinst{q}{q}{}
	\declinst{r}{r}{}

	\mess{$m_1$}{p}{q}
	\nextlevel
	\mess{$m_2$}{q}{p}
	\nextlevel
	\mess{$m_3$}{q}{r}
\end{msc}
\caption{Simple example with an exchange of three messages.}
\label{fig:msc-exmp}
\end{figure}
\end{example}

Given a set of processes $\Procs$, an MSC 
$M=\big((w_p)_{p\in\Procs},\source\big)$ is said to be a 
\emph{prefix} of another MSC 
$M'=\big((w'_p)_{p\in\Procs},\source'\big)$, denoted by 
$M \prefixorder M'$, if the following conditions hold:  
\begin{itemize}
    \item for every $p \in \Procs$, the sequence $w_p$ is a prefix 
    of $w'_p$;  
    \item for every receive event $e$ of $M$, it holds that 
    $\source'(e)=\source(e)$.  
\end{itemize}

The \emph{concatenation} of two MSCs $M_1$ and $M_2$ is the MSC 
$M_1 \cdot M_2$ obtained by stacking $M_1$ vertically above $M_2$. 
Formally, let 
$M_1=((w_p^1)_{p\in\Procs},\source_1)$ and 
$M_2=((w_p^2)_{p\in\Procs},\source_2)$. Then:  
\begin{inparaenum}[(i)]
   \item for each process $p$, the sequence is 
   $w_p = w_p^1 \cdot w_p^2$;  
   \item the source function $\source$ is defined so that 
   $\source(e)=\source_i(e)$ for all receive events $e$ belonging 
   to $M_i$, with $i\in\{1,2\}$.  
\end{inparaenum}


\subsubsection*{Happens-before relation and linearisations}
In a given MSC $M$, an event $\event$ happens before $\event'$, if 
$\event$ and $\event'$ are events
of a same process $p$ and happen in that order on 
the timeline of $p$; $\event$ is send event matched by $\event'$; and
a sequence of such situations defines a path from $\event$ to $\event'$.

\bigskip

\begin{definition}[Happens-before relation]
Let $\mmsc$ be an MSC. The happens-before relation over $M$
is the binary relation $\happensbeforestrict$ defined as 
the least transitive relation over $\eventsof{\mmsc}$ such that:
\begin{itemize}
   \item for all 
   $p,i,j$, if $i<j$, then $(p,i)\happensbeforestrict (p,j)$, and
   \item for all receive events 
   $\event$, $\source(\event) \happensbeforestrict \event$.
\end{itemize}
\end{definition}

\bigskip

\begin{example}
Consider the Example~\ref{exmp:msc}. 
The following happens-before relations are valid:
$${}!\msg_1{}\happensbeforestrict {}?\msg_1 \happensbeforestrict {}!\msg_2
\happensbeforestrict {}!\msg_3 \happensbeforestrict {}?\msg_3$$
and
$${}!\msg_1{}\happensbeforestrict {}?\msg_1 \happensbeforestrict {}!\msg_2
\happensbeforestrict {}?\msg_2.$$
\end{example}

\bigskip

\begin{definition}[Linearisation]\label{def:linearisation}
	A \emph{linearisation} of an MSC $\mmsc$ is a
	total order $\alinearisation$ on $\eventsof{\mmsc}$
	that refines $\happensbeforestrict$:  for all events $\event,\event'$, 
	if $\event\happensbeforestrict \event'$, then $\event\alinearisation \event'$. 
\end{definition}
We write $\linearisationsof{\mmsc}{}$ for the set of all linearisations
of $\mmsc$. 
We often identify a linearisation with the execution it induces.

\bigskip

\begin{example}\label{exmp:lin}
Considering the Example~\ref{exmp:msc},	
let $\mmsc$ be the MSC in Figure~\ref{fig:msc-exmp}. 
The elements of the set $\linearisationsof{\mmsc}{}$ are
$$!m_1?m_1!m_2?m_2!m_3?m_3,$$
$$!m_1?m_1!m_2!m_3?m_2?m_3,$$
$$!m_1?m_1!m_2!m_3?m_3?m_2.$$
\end{example}

Given an MSC $\mmsc$, we write 
$\linearisationsof{\mmsc}{\acommunicationmodel}$ to denote
$\linearisationsof{\mmsc}{}\cap\executionsofmodel{\acommunicationmodel}$;
the executions of $\linearisationsof{\mmsc}{\acommunicationmodel}$ 
are called the linearisations of $\mmsc$ in the communication 
model \verb|com|.

\bigskip

\begin{definition}[$\acommunicationmodel$-linearisable MSC]
	\label{def:linearisable-msc}
	An MSC $\mmsc$ is \textit{linearisable} in a communication 
	model $\acommunicationmodel$
	if $\linearisationsof{\mmsc}{\acommunicationmodel}\neq\emptyset$.
	We write $\mscsetofmodel{\acommunicationmodel}$ for the set of all MSCs 
	linearisable in $\acommunicationmodel$.
\end{definition}

\bigskip

\begin{example}
Consider the Example~\ref{exmp:msc} and the respective linearisation
listed in Example~\ref{exmp:lin}. 
The MSC $\mmsc$ is \textit{linearisable} in the $\synchmodel$ communication 
model because
$\linearisationsof{\mmsc}{\synchmodel}\neq\emptyset$.
The only element of $\linearisationsof{\mmsc}{\synchmodel}$ is
$$!m_1?m_1!m_2?m_2!m_3?m_3.$$
All the send events are followed by the respective
receive events.
\end{example}

%% TODO: non so se sono utili
% Finally, we introduce a property that will be helpful in the next paragraph 
% for giving an alternative characterisation of deadlock-freedom of a system 
% of communicating finite state machines.

% \begin{definition}[Causally-closed communication model]\label{def:causally-closed-communication-model}
% 	A communication model $\acommunicationmodel$ is \emph{causally-closed} if for all MSCs $M$,
% 	$\linearisationsof{\mmsc}{\acommunicationmodel}\neq\emptyset$ implies that
% 	$\linearisationsof{\mmsc}{\acommunicationmodel}=\linearisationsof{\mmsc}{}$.
% \end{definition}

  
% %   \input{proofs/lem-pp-is-causally-closed.tex}
%   %The proof of the following Lemma is in Appendix \ref{app:pp-is-causally-closed}.


\subsubsection*{Communicating finite state machines.}
We recall the definition of communicating finite state machines~\cite{BrandZafiropulo}.

\bigskip

\begin{definition}[CFSM]\label{def:cfsm}
    A communicating finite state machine (CFSM) is an NFA 
	with $\varepsilon$-transitions $\acfsm$ over the alphabet $\Act$.
    A system of CFSMs is a tuple $\cfsms = (\acfsm_p)_{p\in\Procs}$.
\end{definition}

Given a system of CFSMs $\cfsms=(\acfsm_p)_{p\in\Procs}$,
we write $\acceptcompletion{\cfsms}{}$ for the system of CFSMs 
$\acceptcompletion{\cfsms}=(\acceptcompletion{\acfsm_p})_{p\in\Procs}$
where all states are accepting, i.e., $F_p = Q_p$.

\bigskip

% Let's now define the asynchronous product of an automaton.
% TODO: capire questa definizione di prodotto ASYNC(che serve per la prova)
% capire se c'è differenza con quello SYNC (che serve per il tool)
% \begin{definition}[Asynchronous product automaton]
% Let $\{A_i\}_{i=1}^n$ be a set of automata. For each ordered pair 
% $(i,j)$ of process indices, we use two buffers: $B^s_{i,j}$ 
% (pending messages sent by $P_i$ but not yet accessible by $P_j$) 
% and $B^r_{i,j}$ (messages delivered to $P_j$ but not yet consumed). 
% All buffers are words over the alphabet $\Sigma$.

% The asynchronous product automaton 
% $A = \prod_{i=1}^n A_i$ over $\hat{\Sigma}$ is given by: % TODO: capire cos'è \hat{\Sigma}

% \begin{itemize}
%   \item \textbf{States:} A state $q$ consists of the local states 
%   $q_i$ of each $A_i$, together with the buffer contents.
%   \item \textbf{Initial state:} $q_0$ has all components in their 
%   start states $q_i^0$ and all buffers empty.
%   \item \textbf{Transitions:} $\delta \subseteq Q \times 
%   (\hat{\Sigma} \cup \{\tau\}) \times Q$.
%     \begin{enumerate}
%       \item For $x \in \hat{\Sigma}_i$, $(q,x,q') \in \delta$ iff:  
%       (a) states of all other processes $k \neq i$ are unchanged,  
%       (b) $(q_i,x,q'_i) \in \delta_i$,  
% 		% TODO: capire semantica dei buffer dal paper di Alur
%       (c) if $x = receive(j,i,a)$ then the buffer $B^r_{j,i}$ consume the message $a$ (if present),  
%       (d) if $x = send(i,j,a)$ then the message $a$ is appended in the buffer $B^s_{i,j}$,  
%       (e) all other buffers remain unchanged.
%       \item There is a $\tau$-transition $(q,\tau,q')$ if $q$ and 
%       $q'$ differ only in that one buffer $B^s_{i,j}$ loses its 
%       head symbol $a$, and $B^r_{i,j}$ appends $a$.
%     \end{enumerate}
%   \item \textbf{Accepting states:} $q$ is accepting if all $q_i$ 
%   are accepting and all buffers are empty.
% \end{itemize}

% The language $L(A)$ of $A$ consists of all words in $\hat{\Sigma}^*$ 
% taking $q_0$ to an accepting state, interpreting $\tau$ as 
% $\varepsilon$. For any set of automata $\{A_i\}$, the language 
% $L(\prod_i A_i)$ contains only complete, well-formed words. For a 
% given MSC $M$, $L(\prod_i A_i)$ either contains all linearisations 
% of $M$, or none.
% \end{definition}

% \bigskip

\begin{definition}[Executions of CFSMs in $\acommunicationmodel$]
\label{def:executions-of-cfsms}
Given a system $\cfsms=(\acfsm_p)_{p\in\Procs}$ and a 
model $\acommunicationmodel$,  
$\executionsofcfsms{\cfsms}{\acommunicationmodel}$ is the set of 
executions $e \in \executionsofmodel{\acommunicationmodel}$ such that  
$\projofon{e}{p}\in\languageofnfa{\acfsm_p}$ for all $p$.  
\end{definition}

% \begin{remark}
% 	Let $\acommunicationmodel$ be a communication model, 
% 	$\cfsms$ a syste\m of CFSMs, and $e,e'\in\executionsofmodel{acommunicationmodel}$ 
%   such that $\mscof{e}=\mscof{e'}$, 
% 	then $e\in\executionsof{\cfsms}{\acommunicationmodel}$ 
% 	if and only if $e'\in\executionsof{\cfsms}{\acommunicationmodel}$. 
%   This follows from the fact that $\projofon{e}{p}= \projofon{e'}{p}$ for all $p$.
% \end{remark}

We write $\mscsofcfsms{\cfsms}{\acommunicationmodel}$ for the set 
$\{\mscof{e} \mid e\in\executionsofcfsms{\cfsms}{\acommunicationmodel}\}$.

A system is orphan-free if, whenever all machines have reached 
an accepting state, no sent message
remains in transit, i.e. no message is sent but not received.

\bigskip

\begin{definition}[Orphan-free]\label{def:orphan-free}
A system $\cfsms$ is \emph{orphan-free} in a model $\acommunicationmodel$  
if all its executions in $\executionsofcfsms{\cfsms}{\acommunicationmodel}$  
are orphan-free.  
\end{definition}

\textbf{Remark.} All synchronous executions are orphan-free by definition.

A system is said to be \emph{deadlock-free} if every \emph{partial} execution 
(i.e., an execution that may have stopped before reaching a final state) 
can always be extended to a complete or accepting execution.  
Intuitively, this means that the system can never reach a state in which all 
processes are waiting indefinitely, and no further progress is possible.  

\bigskip

\begin{definition}[Deadlock-freedom]\label{def:deadlock-free}
A system $\cfsms$ is \emph{deadlock-free} in $\acommunicationmodel$  
if, for every execution 
$e \in \executionsofcfsms{\acceptcompletion{\cfsms}}{\acommunicationmodel}$,  
there exists a completion $e'$ such that $e \prefixorder e'$ and  
$e' \in \executionsofcfsms{\cfsms}{\acommunicationmodel}$.  
\end{definition}

\textbf{Remark.} This definition captures the idea that, in a deadlock-free system, 
no reachable configuration represents a permanent waiting state, every 
intermediate behaviour can eventually lead to a valid full execution.  
In other words, the system’s communication and synchronisation structure 
guarantees continuous potential for progress.  
For a detailed discussion of this property and its verification within 
our framework, refer to Chapter~\ref{sec:rescu}.

% \begin{remark}\label{rem:equivalent-formulation-of-deadlock-free-cfsms}
% 	A system of CFSMs $\cfsms$ is \emph{deadlock-free} for a communication model
% 	$\acommunicationmodel$ if and only if 
% 	$$
% 	\executionsofcfsms{\acceptcompletion{\cfsms}}{\acommunicationmodel}\subseteq
% 	\prefixclosureof{\executionsofcfsms{\cfsms}{\acommunicationmodel}}
% 	$$
% \end{remark}

\section{Global Types}
This section introduces the fundamental notions required to understand the 
formal proof presented in Chapter~\ref{sec:proof}, focusing in particular 
on the concepts of \emph{Global Types} and \emph{Weak Realisability}. 

\bigskip

\begin{definition}[Global Type]\label{def:gt}
	An \emph{arrow} is a triple $(p,q,m)\in\Procs\times\Procs\times\Messages$ 
	with $p\ne q$; we often write $\marrow{p}{q}{m}$ instead of $(p,q,m)$, and 
	use $\labelalphabet$ to denote the finite set of arrows.
	A \emph{Global Type} $\gt$ is a \emph{deterministic finite automaton} (DFA) over 
	the alphabet $\labelalphabet$.
\end{definition}

We use the notation $p \xleftrightarrow{m} q$ to denote the
\emph{round-trip} exchange of a message $m$: first $p$ sends $m$
to $q$, and then $q$ sends back the same message $m$ to $p$.
This serves as an acknowledgment message for $p$, ensuring that both 
directions of communication are explicitly represented in the model.

\textbf{Remark.}
In our setting, Global Types are viewed as automata that generate a 
language of MSCs (Definition~\ref{def:gt} and~\ref{def:langexist}), 
following the line of work by 
Di~Giusto~et~al.~\cite{di2025realisability}. 
It is important to note that this notion of Global Types differs 
significantly from the one commonly adopted in the Multiparty Session 
Types (MPST) literature, where Global Types are usually defined as 
structured syntactic descriptions of communication protocols. 
Here, the automata-based interpretation aligns with other lines of work that 
study of the \emph{realisability problem}, providing a formal foundation for 
reasoning about synchrony, causality, and implementability. 
For a comparison and discussion of similarities and differences 
with other formalisms, refer to Chapter~\ref{sec:rel}.

\bigskip

\begin{example}
An example of a Global Type expressed as an automaton is the following.
Consider the not-realisable specification stated in 
Listing~\ref{lst:not-impl-exm}. 
The specification automaton depicted in Figure~\ref{fig:gt-exm} is a DFA
$\mathcal{A} = (Q, \Sigma, \delta, q_0, F)$, where:
\begin{itemize}
	\item $\mathbb{P} = \{A, B, C, D\}$;
    \item $Q = \{1, 2, 3, 4, 5\}$ is the set of states;
    \item $\Sigma = \{\marrow{A}{B}{x}, \marrow{A}{B}{y}, 
	\marrow{C}{D}{z}, \marrow{C}{D}{w}\}$ is the alphabet $\labelalphabet$;
    \item The transition function $\delta : Q \times \Sigma \to Q$ is defined as:
          \[
          \delta(1, \marrow{A}{B}{x}) = 2, \quad
          \delta(1, \marrow{A}{B}{y}) = 3, \quad
          \delta(2, \marrow{C}{D}{z}) = 4, \quad
          \delta(3, \marrow{C}{D}{w}) = 5;
          \]
    \item The initial state is $q_0 = 1$;
    \item The set of accepting states is $F = \{4, 5\}$.
\end{itemize}

% Hence, the automaton accepts the language:
% \[
% \mathcal{L}_{\text{words}}(\mathcal{A}) =
% \{\marrow{A}{B}{x}\marrow{C}{D}{z},\ \marrow{A}{B}{y}\marrow{C}{D}{w}\}.
% \]
\bigskip

\begin{figure}[!ht]
	\centering
	\begin{tikzpicture}[node distance=1.5cm, auto, scale=0.9]
		\node[state, initial, initial text={}] (s0) {1};
		\node[state] (s1) [above right=of s0] {2};
		\node[state] (s2) [below right=of s0] {3};
		\node[state,accepting] (s3) [right=of s1] {4};
		\node[state,accepting] (s4) [right=of s2] {5};

		\draw[->] (s0) to node[above,sloped]{$\gtlabel{A}{B}{x}$}(s1);
		\draw[->] (s0) to node[above,sloped]{$\gtlabel{A}{B}{y}$}(s2);
		\draw[->] (s1) to node[above,sloped]{$\gtlabel{C}{D}{z}$}(s3);
		\draw[->] (s2) to node[above,sloped]{$\gtlabel{C}{D}{w}$}(s4);
	\end{tikzpicture}
	\caption{Listing~\ref{lst:not-impl-exm}'s global type.}
	\label{fig:gt-exm}
\end{figure}
\end{example}

\bigskip

% --------------relationship between MSCs and Global Types--------------

We can now formally define the relationship between MSCs and Global Types. 
Intuitively, Global Types represent a set of MSCs, allowing us to reason 
about multiple message sequence scenarios. 

A Global Type defines a language of MSCs in two different ways, one
existential and one universal. Let $\labellanguageof{\gt}$ be the set of
sequences of arrows $w$ accepted by $\gt$.
Informally, the existential MSC language $\existentialmsclanguageof{\gt}{}$ of a 
Global Type $\gt$ is the set of MSCs that admit at least one representation as a
sequence of arrows in $\labellanguageof{\gt}$, and the universal MSC
language $\universalmsclanguageof{\gt}$ of a Global Type $\gt$ is the set of
MSCs whose representations as sequences of arrows are all in
$\labellanguageof{\gt}$. We will just give the formal definition of
$\existentialmsclanguageof{\gt}{}$:

\bigskip

\begin{definition}[$\existentialmsclanguageof{\gt}$]\label{def:langexist}
	$$
		\existentialmsclanguageof{\gt} \eqdef \{\labeltomsc{w} \mid
		w \in \labellanguageof{\gt}\}
  	$$
\end{definition}

% TODO: dare più contesto ed esempio

When a Global Type is implemented in a concrete system,
its semantics depends on the chosen communication model.

\bigskip

\begin{definition}[Global Type Language]\label{def:global-type-language}
    Let $\gt$ be a global type and $\acommunicationmodel$ a communication model. 
    The language of $\gt$ in $\acommunicationmodel$ is $
    \executionsof{\gt}{\acommunicationmodel}\eqdef
    \bigcup\{\linearizationsof{M}{\acommunicationmodel}\mid M\in\existentialmsclanguageof{\gt}\}$.
\end{definition}

We now give the definitions of \emph{weak} and \emph{safe} realisability.

\bigskip

\begin{definition}[Weak realisability]\label{def:gt-weak-realis}
A global type $\gt$ is \emph{weak realisable} in the
communication model $\acommunicationmodel$ if there is a
system CFSM $\cfsms$ such that the following condition hold:
$\executionsof{\cfsms}{\acommunicationmodel} =
\executionsof{\gt}{\acommunicationmodel}$.
\end{definition}

Although this work does not focus on safe realisability, we will still
define it formally to highlight the main differences and similarities 
with other works in Chapter~\ref{sec:rel}.

\bigskip

\begin{definition}[Safe realisability]\label{def:gt-safe-realis}
    A global type $\gt$ is \emph{safe realisable}
    in the communication model $\acommunicationmodel$
    if there is a system $\cfsms$ that is \emph{weak realisable} and  
    $\cfsms$ is deadlock-free in $\acommunicationmodel$ 
	(Def.~\ref{def:deadlock-free}).
\end{definition}

The definition of Weak realisability corresponds to  
the property of \emph{global type conformance}: all system executions 
faithfully follow the behaviours prescribed by the global type.
When $\acommunicationmodel$ is $\ppmodel$ or $\synchmodel$, 
our notion of safe realisability coincides with the notion of 
\emph{safe realisability} introduced in~\cite{alur2005realizability}.
This equivalence does not extend to more general communication models, 
such as the mailbox model~\cite{di2025realisability}.
% TODO: come dare supporto a queste affermazioni?
We are now ready to present the main contributions of this work.