\chapter{ReSCu}\label{sec:rescu}
In the previous chapter, I examined the theoretical aspects of the
implementability problem for MPST, culminating in the main result: the
\textbf{undecidability} of weak implementability under synchronous
semantics. That analysis not only establishes a fundamental limitation,
but also highlights the need to explore alternative approaches, such as
identifying restricted subclasses or designing practical techniques that
can still support verification in real-world scenarios.  
This chapter changes the focus from undecidability to decidability. I present
\textsc{ReSCu} (first introduced in
\cite{desgeorges2023rsc,di2023multiparty,guizouarn2023communicating}),
a verification tool that provides automated support for reasoning about
realizability. The tool enables the analysis of properties such as
\textit{deadlock-freedom} and \textit{progress}, serving as a
\emph{building block} toward the broader goal of decidable
implementability checks.  

I describe the features of \textsc{ReSCu}, the input language it adopts,
and its implementation details, with particular emphasis on the
extensions and modifications I introduced to improve its capabilities
\cite{rescuoriginalrepo}. The updated public repository, which includes
the new features and illustrative examples, is available at:
\begin{center}
\verb|https://github.com/gabrielegenovese/rescu|~\cite{rescurepo}.
\end{center}

\section{Characteristics}
\textsc{ReSCu} is a command-line tool that can check both membership in the 
class of \verb|synch| systems (called Realisable with Synchronous Communication 
or, in brief, RSC from now on) and reachability of regular sets of configurations. It 
accepts input systems with arbitrary topologies and supports both FIFO and 
bag buffers. The tool provides several options: 
\verb|-isrsc| checks whether the system is RSC, and \verb|-mc| checks reachability of 
bad configurations. Both checks can be combined in a single run. The \verb|-fifo| option 
overrides buffer types by treating all as FIFO. When a system is unsafe, the 
\verb|-counter| option (used with \verb|-mc|) produces an RSC execution that leads 
to the bad configuration, while the same option used with \verb|-isrsc| outputs the %borderline
violation execution if the system is not RSC. Additional features include 
a progress display to estimate remaining runtime during long computations, and 
\verb|-to_dot|, which exports the system to DOT format for visualization.
One of the most similar tools is \textsc{McScm} \cite{heussner2012mcscm}, that
uses a framework with different verification techniques. 
Symbolic Communicating Machines (SCM), defined and used in \cite[Definition 5.1]{le2008abstract}
serve as the input format of the tool. SCMs are Communicating 
Finite-State Machines (CFSM, Definition~\ref{def:cfsm}) 
extended with the use of channels and a finite set of variables (that 
corresponds to message).
The grammar has been updated to provide greater flexibility and clarity. In
particular, transition guards have been made optional (with a default value
\verb|: when true|), and a new \verb|final| keyword has been introduced to
explicitly specify final states. The updated grammar is shown in
Listing~\ref{lst:scm-grammar}. 

% \bigskip
% \begin{definition}[SCM]\label{def:scm}
% A \emph{Symbolic Communicating Machine} (SCM) with $N$ queues is defined by a tuple 
% $\langle C, V, c_0, \Theta_0, P, \Delta \rangle$ where:
% \begin{itemize}
%     \item $C$ is a nonempty finite set of \emph{locations} (control states).
%     \item $V = \{v_1, \ldots, v_n\}$ is a finite set of variables.  
%     The domain of values of a variable $v$ is denoted by $D_v$, and the set of valuations 
%     of all variables in $V$ by $D_V$.
%     \item $c_0 \in C$ is the initial control state, and $\Theta_0 \subseteq D_V$, 
%     a predicate on $V$, is the initial condition.
%     \item $P = \{p_1, \ldots, p_l\}$ is a finite set of formal parameters that are used 
%     to send/receive values to/from FIFO queues.  
%     We assume that all queues use the same set of parameters $D_P$.
%     \item $\Delta$ is a finite set of \emph{transitions}.  
%     A transition $\delta$ is either an input 
%     $\langle c_1, G, i? \vec{p}, A, c_2 \rangle$ 
%     or an output $\langle c_1, G, i! \vec{p}, A, c_2 \rangle$, where:
%     \begin{enumerate}
%         \item $c_1$ and $c_2$ are respectively the origin and destination locations;
%         \item $i \in [1..N]$ is a queue number;
%         \item $\vec{p}$ is the vector of formal parameters, which holds the values sent or 
%         received to/from the queue $i$;
%         \item $G(\vec{v}, \vec{p}) \subseteq D_V \times D_P$ is a predicate on the variables 
%         and the formal parameters (also called \emph{guard});
%         \item $A$ is an assignment of the form $\vec{v}' := A(\vec{v}, \vec{p})$, 
%         where $A : D_V \times D_P \to D_V$, which defines the values of the variables 
%         after the transition.
%     \end{enumerate}
% \end{itemize}
% \end{definition}

\bigskip

\begin{lstlisting}[language={},caption={Modified SCM grammar},
    keywordstyle=\color{blue}\bfseries,label={lst:scm-grammar}] 
prog         ::= <header> <aut_list> [<bad_confs>]
header       ::= scm <ident>:<channels> [<bags>] <parameters>
channels     ::= nb_channels = <int>;
bags         ::= //# bag_buffers = <int_list>
int_list     ::= <int>
               | <int_list>, <int>
parameters   ::= parameters = <param_list>
param_list   ::= <param>
               | <param> <param_list>
param        ::= {int | real} <ident>;
aut_list     ::= automaton <ident>:<initial>;<final>; <state_list>
initial      ::= initial : <int_list>;
final        ::= final : <int_list>;
state_list   ::= <state>
               | <state_list> <state>
state        ::= state <int> : <trans_list>
trans_list   ::= <transition>
               | <trans_list> <transition>
guard        ::= : when true | <nothing>
transition   ::= to <int> : when true , <int> <action> <ident>
action       ::= "!" | "?"
bad_confs    ::= bad_states: <bad_list>
bad_list     ::= (<bad_conf>)
               | <bad_list> (<bad_conf>)
bad_conf     ::= <bad_state>
               | <bad_state> with <bad_buffers>
bad_state    ::= automaton <ident>: in <int>: true [<bad_state>]
bad_buffers  ::= <regular_expression>
nothing      ::= 
\end{lstlisting}

Given the definition of SCM and the newly introduced input grammar, 
I now present an example to illustrate how these concepts are applied 
in practice with the tool. For clarity, the example is expressed in the 
CFSM notation rather than in the SCM formalism. Consequently, channels 
and variables are omitted and replaced directly by messages. However, 
the figures are displayed in SCM format, as they are automatically 
generated by the tool.

\bigskip

\begin{example}[Ping-Pong Example]\label{exm:ping}

Let the set of processes be $\Procs = \{A,B\}$,
the set of messages $\Msg = \{\texttt{ping},\texttt{pong}\}$,
and the set of channels consist of a single FIFO channel~0 from 
$A$ to $B$ and from $B$ to $A$. The corresponding actions are
$$
\Act = \{\, (A,B,!,\texttt{ping}), (B,A,?,\texttt{ping}), 
(B,A,!,\texttt{pong}), (A,B,?,\texttt{pong}) \,\}.
$$

The system of CFSMs is $\cfsms = (\acfsm_A, \acfsm_B)$, where:
$$
\acfsm_A = (Q_A, \Sigma, \delta_A, q_{0,A}, F_A)
$$

with
\begin{itemize}
\item $Q_A = \{0,1,2\}$, initial state $q_{0,A} = 0$, final state $F_A = \{2\}$,
\item transitions: $0 \xrightarrow{(A,B,!,\texttt{ping})} 1 \xrightarrow{(B,A,?,\texttt{pong})} 2$.
\end{itemize}

$$
\acfsm_B = (Q_B, \Sigma, \delta_B, q_{0,B}, F_B)
$$

with
\begin{itemize}
\item $Q_B = \{0,1,2\}$, initial state $q_{0,B} = 0$, final state $F_B = \{2\}$,
\item transitions: $0 \xrightarrow{(B,A,?,\texttt{ping})} 1 \xrightarrow{(A,B,!,\texttt{pong})} 2$.
\end{itemize}

This CFSM system $\cfsms$ is showed in Figure~\ref{fig:ping}. The corresponding
input as SCM format for the tool is showed in Listing~\ref{lst:ping}.

\bigskip

\begin{lstlisting}[language={},caption={Tool's input for Example~\ref{exm:ping}},label={lst:ping}]
scm ping_pong :

nb_channels = 1 ;
parameters :
unit ping ;
unit pong ;

automaton A :
initial : 0
final : 2

state 0 :
to 1 : 0 ! ping ;
state 1 :
to 2 : 0 ? pong ;
state 2 :

automaton B :
initial : 0
final : 2

state 0 :
to 1 : 0 ? ping ;
state 1 :
to 2 : 0 ! pong ;
state 2 :
\end{lstlisting}
  
\begin{figure}[!ht]
\centering
\begin{tikzpicture}[>=latex',line join=bevel,]
  \pgfsetlinewidth{1bp}
%%
\pgfsetcolor{black}
  % Edge: Bdummy -> B0
  \draw [->] (99.0bp,258.17bp) .. controls (99.0bp,250.56bp) and (99.0bp,241.35bp)  .. (99.0bp,221.45bp);
  % Edge: B0 -> B1
  \draw [->] (99.0bp,184.91bp) .. controls (99.0bp,173.26bp) and (99.0bp,157.55bp)  .. (99.0bp,132.85bp);
  \definecolor{strokecol}{rgb}{0.0,0.0,0.0};
  \pgfsetstrokecolor{strokecol}
  \draw (121.12bp,158.75bp) node {0 ? ping};
  % Edge: A1 -> A2
  \draw [->] (27.0bp,96.051bp) .. controls (27.0bp,84.609bp) and (27.0bp,69.297bp)  .. (27.0bp,44.218bp);
  \draw (50.625bp,70.25bp) node {0 ? pong};
  % Edge: A0 -> A1
  \draw [->] (27.0bp,184.91bp) .. controls (27.0bp,173.26bp) and (27.0bp,157.55bp)  .. (27.0bp,132.85bp);
  \draw (48.375bp,158.75bp) node {0 ! ping};
  % Edge: B1 -> B2
  \draw [->] (99.0bp,96.051bp) .. controls (99.0bp,84.609bp) and (99.0bp,69.297bp)  .. (99.0bp,44.218bp);
  \draw (121.88bp,70.25bp) node {0 ! pong};
  % Edge: Adummy -> A0
  \draw [->] (27.0bp,258.17bp) .. controls (27.0bp,250.56bp) and (27.0bp,241.35bp)  .. (27.0bp,221.45bp);
  % Node: A2
\begin{scope}
  \definecolor{strokecol}{rgb}{0.0,0.0,0.0};
  \pgfsetstrokecolor{strokecol}
  \draw (27.0bp,22.0bp) ellipse (18.0bp and 18.0bp);
  \draw (27.0bp,22.0bp) ellipse (22.0bp and 22.0bp);
  \draw (27.0bp,22.0bp) node {2};
\end{scope}
  % Node: Bdummy
\begin{scope}
  \definecolor{strokecol}{rgb}{0.0,0.0,0.0};
  \pgfsetstrokecolor{strokecol}
  % \draw (99.0bp,276.0bp) ellipse (27.0bp and 18.0bp);
  \draw (99.0bp,276.0bp) node {B};
\end{scope}
  % Node: B0
\begin{scope}
  \definecolor{strokecol}{rgb}{0.0,0.0,0.0};
  \pgfsetstrokecolor{strokecol}
  \draw (99.0bp,203.0bp) ellipse (18.0bp and 18.0bp);
  \draw (99.0bp,203.0bp) node {0};
\end{scope}
  % Node: B1
\begin{scope}
  \definecolor{strokecol}{rgb}{0.0,0.0,0.0};
  \pgfsetstrokecolor{strokecol}
  \draw (99.0bp,114.5bp) ellipse (18.0bp and 18.0bp);
  \draw (99.0bp,114.5bp) node {1};
\end{scope}
  % Node: A1
\begin{scope}
  \definecolor{strokecol}{rgb}{0.0,0.0,0.0};
  \pgfsetstrokecolor{strokecol}
  \draw (27.0bp,114.5bp) ellipse (18.0bp and 18.0bp);
  \draw (27.0bp,114.5bp) node {1};
\end{scope}
  % Node: B2
\begin{scope}
  \definecolor{strokecol}{rgb}{0.0,0.0,0.0};
  \pgfsetstrokecolor{strokecol}
  \draw (99.0bp,22.0bp) ellipse (18.0bp and 18.0bp);
  \draw (99.0bp,22.0bp) ellipse (22.0bp and 22.0bp);
  \draw (99.0bp,22.0bp) node {2};
\end{scope}
  % Node: A0
\begin{scope}
  \definecolor{strokecol}{rgb}{0.0,0.0,0.0};
  \pgfsetstrokecolor{strokecol}
  \draw (27.0bp,203.0bp) ellipse (18.0bp and 18.0bp);
  \draw (27.0bp,203.0bp) node {0};
\end{scope}
  % Node: Adummy
\begin{scope}
  \definecolor{strokecol}{rgb}{0.0,0.0,0.0};
  \pgfsetstrokecolor{strokecol}
  % \draw (27.0bp,276.0bp) ellipse (27.0bp and 18.0bp);
  \draw (27.0bp,276.0bp) node {A};
\end{scope}
%
\end{tikzpicture}
\caption{Simple Ping-Pong example.}
\label{fig:ping}
\end{figure}

\end{example}

\section{Progress and Deadlock-Freedom}
I extended \textsc{ReSCu} with verification routines that focus on two
fundamental correctness properties of distributed systems: \emph{progress} and
\emph{deadlock-freedom}. To enable this, the tool constructs the synchronous
system using the synchronous product operation whenever the input SCM is recognized 
as realisable in synchronous communication semantic (RSC). 
Once the system is proven to be RSC, we can safely construct a 
well-formed synchronous product from it, and, given the synchronous product, 
the tool elaborates the other two additional checks.

\textbf{Remark.} 
The discussion in this chapter assumes \emph{complete nondeterministic 
fairness} over choices. In other words, whenever the system encounters 
a nondeterministic branching, all possible continuations are treated 
equally and none of them can be ignored. This assumption ensures that the 
verification does not overlook executions simply because they are less 
probable, and it avoids trivial counterexamples where a branch is never 
explored. In practice, relaxing fairness assumptions can yield more 
realistic analyses (e.g.\ prioritising certain branches or modelling 
schedulers with biases), but at the cost of complicating the 
verification procedures. Exploring weaker or alternative fairness models 
is therefore an interesting direction for future work, especially for 
applications where nondeterminism is influenced by external constraints 
such as message delays or resource contention.

% We consider the definitions given in Chapter~\ref{sec:pre}.
% Definition~\ref{def:cfsm} corresponds to the concept of an SCM.
We now present the
definition of the Synchronous Product for CFSMs, which I have implemented in the
tool, and it serves as a key component for the analysis.

\bigskip

\begin{definition}[Synchronous Product]\label{def:syncprod}
Let $\mathcal{S} = (\mathcal{A}_p)_{p \in \mathbb{P}}$ be a system of CFSMs, where 
$\mathcal{A}_p = (L_p, \mathit{Act}_p, \delta_p, l_{0,p}, F_p)$ is the CFSM associated 
to process $p$.  

The \emph{synchronous product} of $\mathcal{S}$ is the global type 
$P = \mathrm{prod}_s(\mathcal{S}) = (L, \mathit{Arr}, \delta, l_0, F)$,
where
\begin{itemize}
    \item $L = \prod_{p \in \mathbb{P}} L_p$ is the set of global locations,
    \item $l_0 = (l_{0,p})_{p \in \mathbb{P}}$ is the initial global state,
    \item $F = \prod_{p \in \mathbb{P}} F_p$ is the set of global final states,
    \item $\delta$ is the transition relation defined as follows:  
    $(\vec{l}, \; p \xrightarrow{m} q, \; \vec{l}\,') \in \delta$ if
    \[
    (l_p,\, !m^{p \to q},\, l'_p) \in \delta_p, \quad 
    (l_q,\, ?m^{p \to q},\, l'_q) \in \delta_q, \quad 
    l'_r = l_r \;\; \text{for all } r \notin \{p,q\}.
    \]
\end{itemize}
\end{definition}

\bigskip

\begin{example}[Synchronous Product Example]\label{exm:syncping}
Consider the system of CFSMs $\cfsms = (\acfsm_A, \acfsm_B)$ 
from the Example~\ref{exm:ping}. Its synchronous product is 
$P = \mathrm{prod}_s(\cfsms) = (L, \mathit{Arr}, \delta, l_0, F)$,
where
\begin{itemize}
  \item $L = Q_A \times Q_B = \{0,1,2\} \times \{0,1,2\}$,
  \item $l_0 = (0,0)$,
  \item $F = \{(2,2)\}$,
  \item $\delta$ consists of the following transitions:
  $(0,0) \;\xrightarrow{A \xrightarrow{\texttt{ping}} B}\; (1,1)
  \;\xrightarrow{B \xrightarrow{\texttt{pong}} A}\; (2,2)$.
\end{itemize}
Thus, the synchronous product captures the joint behaviour: process $A$ 
sends $\texttt{ping}$ to $B$, then $B$ responds with $\texttt{pong}$ 
to $A$, and both processes reach their final states simultaneously.
Figure~\ref{fig:syncping} illustrates the product's automaton 
$\mathrm{prod}_s(\cfsms)$.

\begin{figure}[!ht]
  \centering
  \begin{tikzpicture}[>=latex',line join=bevel,scale=0.8]
    \pgfsetlinewidth{1bp}
  %%
  \pgfsetcolor{black}
    % Edge: A1B1 -> A2B2
    \draw [->] (27.0bp,96.051bp) .. controls (27.0bp,84.609bp) and (27.0bp,69.297bp)  .. (27.0bp,44.218bp);
    \definecolor{strokecol}{rgb}{0.0,0.0,0.0};
    \pgfsetstrokecolor{strokecol}
    \draw (69.0bp,70.25bp) node {B$\to$A:pong};
    % Edge: A0B0 -> A1B1
    \draw [->] (27.0bp,184.91bp) .. controls (27.0bp,173.26bp) and (27.0bp,157.55bp)  .. (27.0bp,132.85bp);
    \draw (67.125bp,158.75bp) node {A$\to$B:ping};
    % Edge: start_node -> A0B0
    \draw [->] (27.0bp,258.17bp) .. controls (27.0bp,250.56bp) and (27.0bp,241.35bp)  .. (27.0bp,221.45bp);
    % Node: A2B2
  \begin{scope}
    \definecolor{strokecol}{rgb}{0.0,0.0,0.0};
    \pgfsetstrokecolor{strokecol}
    \draw (27.0bp,22.0bp) ellipse (18.0bp and 18.0bp);
    \draw (27.0bp,22.0bp) ellipse (22.0bp and 22.0bp);
    \draw (27.0bp,22.0bp) node {2};
  \end{scope}
    % Node: A1B1
  \begin{scope}
    \definecolor{strokecol}{rgb}{0.0,0.0,0.0};
    \pgfsetstrokecolor{strokecol}
    \draw (27.0bp,114.5bp) ellipse (18.0bp and 18.0bp);
    \draw (27.0bp,114.5bp) node {1};
  \end{scope}
    % Node: A0B0
  \begin{scope}
    \definecolor{strokecol}{rgb}{0.0,0.0,0.0};
    \pgfsetstrokecolor{strokecol}
    \draw (27.0bp,203.0bp) ellipse (18.0bp and 18.0bp);
    \draw (27.0bp,203.0bp) node {0};
  \end{scope}
    % Node: start_node
  \begin{scope}
    \definecolor{strokecol}{rgb}{0.0,0.0,0.0};
    \pgfsetstrokecolor{strokecol}
    \draw (27.0bp,276.0bp) node {start};
  \end{scope}
  %
  \end{tikzpicture}
  \caption{Synchronous Product of the CFSM system in Example~\ref{exm:syncping}.}
  \label{fig:syncping}
\end{figure}
% TODO: mostrare pseudo codice dell'algoritmo?
  
\end{example}

After constructing the synchronous product, the tool performs several
important post-processing operations. In particular, it removes any
unreachable nodes from the resulting product, simplifying the structure
and ensuring that only relevant states are retained for further analysis.
We can now define the two SCM properties implemented as verification 
routines in the tool.

Let's consider the defintion of deadlock-freedom for CFSMs 
(Definition~\ref{def:deadlock-free}). I will instanciate the semantic of 
the system, which is $\synchmodel$. This implies 
that the system uses the synchronous product when analysing the 
executions of the system (Definition~\ref{def:syncprod}).

\bigskip

\begin{definition}[Deadlock-freedom in $\synchmodel$]
A system $\cfsms$ is \emph{deadlock-free} in $\synchmodel$  
if for every execution 
$e \in \executionsofcfsms{\acceptcompletion{\cfsms}}{\synchmodel}$,  
there exists a completion $e'$ with $e \prefixorder e'$ and  
$e' \in \executionsofcfsms{\cfsms}{\synchmodel}$.  
\end{definition}

\textbf{Remark.} The notation $\acceptcompletion{\cfsms}$ denotes the 
system obtained by treating every state as an accepting (or final) state. 
In this way, all possible partial executions of the system are taken 
into account. The deadlock-freedom condition then requires that each 
such partial execution can be extended to a complete execution of the 
original system $\cfsms$. Intuitively, this ensures that the system 
cannot ``get stuck'' in the middle of a computation, i.e.\ every 
execution fragment can always be continued.  

More precisely, a system that can reach, from its initial states, some state
that does not lead to a final state is not deadlock-free. Under this definition,
even a loop that never reaches a final state is considered a deadlock,
making the property more restrictive. This check is implemented using a
reverse search algorithm starting from the final states.

% TODO: mostrare pseudo codice dell'algoritmo?

% TODO: Specificare che questa nozione di deadlock-freedom vale per il paper di cinzia
% Vale ancora questo todo?

\bigskip

\begin{definition}[Progress]\label{def:progress}
A system of CFSMs $\cfsms$ satisfies the \emph{progress} property in $\rscmodel$
if for every execution $e \in \executionsofcfsms{\acceptcompletion{P}}{}$, 
with $P = \mathrm{prod}_s(\cfsms)$,
\begin{itemize}
    \item the execution $e$ is also a valid execution of $e\in \executionsofcfsms{P}{}$, or
    \item there exists another execution $e'\in\executionsofcfsms{\acceptcompletion{P}}{}$
          such that $e \prefixorder e'$, with $e \neq e'$.
\end{itemize}
\end{definition}

Intuitively, progress ensures that the system never reaches a state
where it is permanently stuck, except in the case of successful termination.
This is weaker than deadlock-freedom, since infinite executions are allowed
as long as they can always perform a new step. In particular, livelocks
(loops without termination) are considered to satisfy progress, but would
violate deadlock-freedom.

% TODO: mostrare pseudo codice dell'algoritmo?

% \section{Verification Algorithms}

% The verification of \emph{deadlock-freedom} and \emph{progress}
% properties in \textsc{ReSCu} are based on a classical
% \emph{reverse search} algorithm~\cite{avis1996reverse}.
% The idea is to initialise the search from ``bad'' states
% (deadlock or non-progress states) and traverse the system graph
% backwards, collecting all states that can eventually reach one of these
% bad states. If the initial state is discovered during the traversal,
% the property does not hold. Otherwise, the property is satisfied.  

% This approach is widely used in model checking since it reduces the
% problem to a systematic graph exploration, avoiding redundant
% re-computation~\cite{padovani2014deadlock}. The two algorithms below
% are specialisations of this method.

% \begin{algorithm}[H]
% \caption{Deadlock Freedom via Reverse Search}
% \label{alg:reverse-search-deadlock}
% \KwIn{System $\mathcal{S}$, initial state $s_0$}
% \KwOut{True if $\mathcal{S}$ is deadlock-free, False otherwise}

% $Visited \gets \emptyset$\;
% $Worklist \gets \{ \text{deadlock states of } \mathcal{S} \}$\;

% \While{$Worklist \neq \emptyset$}{
%     remove $s$ from $Worklist$\;
%     \If{$s \notin Visited$}{
%         add $s$ to $Visited$\;
%         \ForEach{predecessor $p$ of $s$ in $\mathcal{S}$}{
%             add $p$ to $Worklist$\;
%         }
%     }
% }

% \If{$s_0 \in Visited$}{
%     \Return False \tcp*{deadlock is reachable}
% }
% \Return True \tcp*{system is deadlock-free}
% \end{algorithm}

% \begin{algorithm}[H]
% \caption{Progress via Reverse Search}
% \label{alg:reverse-search-progress}
% \KwIn{System $\mathcal{S}$, initial state $s_0$}
% \KwOut{True if $\mathcal{S}$ ensures progress, False otherwise}

% $Visited \gets \emptyset$\;
% $Worklist \gets \{ \text{non-final states with no outgoing transitions} \}$\;

% \While{$Worklist \neq \emptyset$}{
%     remove $s$ from $Worklist$\;
%     \If{$s \notin Visited$}{
%         add $s$ to $Visited$\;
%         \ForEach{predecessor $p$ of $s$ in $\mathcal{S}$}{
%             add $p$ to $Worklist$\;
%         }
%     }
% }

% \If{$s_0 \in Visited$}{
%     \Return False \tcp*{initial state can lead to non-progress}
% }
% \Return True \tcp*{system guarantees progress}
% \end{algorithm}

Lastly, the synchronized system can be exported in DOT format
(with a default filename of \verb|sync.dot|), which allows for graphical 
visualization of its structure and behaviour. Some illustrative examples 
demonstrating these new features are included in the
\verb|examples/deadlock| folder from the online repository~\cite{rescurepo}.
Two of them are showed and explained with details in the next section.

% TODO: CINZIA, X TESI:
% pensare ad un esempio che fa vedere che non è vero in generale

\section{Examples}

To illustrate these notions, I present two examples. The first is the
classical \emph{Dining Philosophers} problem, which shows how resource
contention can lead to deadlock. The second is a minimal looping system
that demonstrates how a process may satisfy the progress property while
still failing to be deadlock-free.

\subsection{The Dining Philosophers}

% TODO: mettere anche versione giusta?
\begin{example}\label{exm:philo}
Consider two philosophers $P_0, P_1$ and two forks $F_1, F_2$, arranged
so that each philosopher needs both forks to eat. If both philosophers
pick up their left fork simultaneously, each waits indefinitely for the
other fork, producing a deadlock. This captures the essence of the Dining
Philosophers problem: concurrent processes blocking one another when
competing for shared resources.

\bigskip

\begin{lstlisting}[language={},caption={Output of Example~\ref{exm:philo}},
    label={lst:philo}]
This system is RSC.
There are some sink states:
Sink: Id=11 Configuration={{ F0:4; F1:3; P1:2; P2:2 }}
There are some deadlock states:
Deadlock: Id=4 Configuration={{ F0:2; F1:1; P1:1; P2:1 }}
Deadlock: Id=11 Configuration={{ F0:4; F1:3; P1:2; P2:2 }}
Deadlock: Id=8 Configuration={{ F0:4; F1:1; P1:1; P2:2 }}
Deadlock: Id=7 Configuration={{ F0:2; F1:3; P1:2; P2:1 }}
\end{lstlisting}

The behaviour of the four participants is shown in
Figure~\ref{fig:philo}. Running the tool on
this input produces the terminal output in
Listing~\ref{lst:philo} and the corresponding synchronous system in
Figure~\ref{fig:philo-sync}. In the
generated figure, the red state marks a configuration where no further
actions are possible, while the three yellow states correspond to
deadlocks, i.e.\ executions where both philosophers wait for each other
indefinitely. The terminal output also lists the precise configurations
of these problematic states.

\newpage

\begin{figure}[!ht]
    \centering
\begin{tikzpicture}[>=latex',line join=bevel,scale=0.79]
  \pgfsetlinewidth{1bp}
%%
\pgfsetcolor{black}
  % Edge: P23 -> P24
  \draw [->] (27.0bp,273.41bp) .. controls (27.0bp,261.76bp) and (27.0bp,246.05bp)  .. (27.0bp,221.35bp);
  \definecolor{strokecol}{rgb}{0.0,0.0,0.0};
  \pgfsetstrokecolor{strokecol}
  \draw (46.5bp,247.25bp) node {7 ? ack};
  % Edge: F00 -> F02
  \draw [->] (163.75bp,552.68bp) .. controls (158.12bp,545.99bp) and (152.42bp,537.69bp)  .. (149.5bp,529.0bp) .. controls (146.63bp,520.45bp) and (146.19bp,510.74bp)  .. (148.12bp,490.45bp);
  \draw (168.25bp,520.75bp) node {4 ? req};
  % Edge: F00 -> F01
  \draw [->] (185.51bp,547.49bp) .. controls (189.85bp,533.83bp) and (195.6bp,515.7bp)  .. (203.7bp,490.19bp);
  \draw (214.55bp,520.75bp) node {0 ? req};
  % Edge: F11 -> F13
  \draw [->] (309.26bp,647.53bp) .. controls (304.38bp,641.32bp) and (299.22bp,633.51bp)  .. (296.5bp,625.5bp) .. controls (293.62bp,617.02bp) and (292.5bp,607.4bp)  .. (292.34bp,587.21bp);
  \draw (315.25bp,617.25bp) node {3 ! ack};
  % Edge: F02 -> F04
  \draw [->] (145.21bp,455.74bp) .. controls (139.43bp,442.32bp) and (131.01bp,422.79bp)  .. (119.85bp,396.89bp);
  \draw (153.59bp,424.25bp) node {5 ! ack};
  % Edge: P21 -> P22
  \draw [->] (27.0bp,454.05bp) .. controls (27.0bp,441.53bp) and (27.0bp,424.37bp)  .. (27.0bp,398.4bp);
  \draw (46.5bp,424.25bp) node {5 ? ack};
  % Edge: P1dummy -> P10
  \draw [->] (494.0bp,643.9bp) .. controls (494.0bp,631.26bp) and (494.0bp,613.55bp)  .. (494.0bp,587.42bp);
  % Edge: F10 -> F11
  \draw [->] (347.8bp,494.92bp) .. controls (347.12bp,524.61bp) and (344.41bp,579.76bp)  .. (334.0bp,625.5bp) .. controls (333.39bp,628.17bp) and (332.63bp,630.9bp)  .. (328.0bp,644.39bp);
  \draw (363.84bp,569.0bp) node {2 ? req};
  % Edge: F10 -> F12
  \draw [->] (328.44bp,461.89bp) .. controls (317.28bp,455.28bp) and (304.21bp,445.37bp)  .. (297.5bp,432.5bp) .. controls (293.87bp,425.54bp) and (292.43bp,417.31bp)  .. (292.34bp,398.17bp);
  \draw (316.25bp,424.25bp) node {6 ? req};
  % Edge: P15 -> P16
  \draw [->] (494.0bp,96.051bp) .. controls (494.0bp,84.609bp) and (494.0bp,69.297bp)  .. (494.0bp,44.218bp);
  \draw (510.5bp,70.25bp) node {2 ! rel};
  % Edge: F14 -> F10
  \draw [->] (329.0bp,308.61bp) .. controls (331.27bp,314.45bp) and (333.57bp,321.18bp)  .. (335.0bp,327.5bp) .. controls (343.53bp,365.15bp) and (346.5bp,409.54bp)  .. (347.83bp,450.24bp);
  \draw (362.3bp,380.0bp) node {6 ? rel};
  % Edge: P11 -> P12
  \draw [->] (494.0bp,454.05bp) .. controls (494.0bp,441.53bp) and (494.0bp,424.37bp)  .. (494.0bp,398.4bp);
  \draw (513.5bp,424.25bp) node {3 ? ack};
  % Edge: P10 -> P11
  \draw [->] (494.0bp,550.67bp) .. controls (494.0bp,537.08bp) and (494.0bp,517.89bp)  .. (494.0bp,490.72bp);
  \draw (512.0bp,520.75bp) node {2 ! req};
  % Edge: P12 -> P13
  \draw [->] (494.0bp,361.91bp) .. controls (494.0bp,350.26bp) and (494.0bp,334.55bp)  .. (494.0bp,309.85bp);
  \draw (512.0bp,335.75bp) node {0 ! req};
  % Edge: F12 -> F14
  \draw [->] (293.48bp,361.86bp) .. controls (293.15bp,351.62bp) and (293.73bp,338.51bp)  .. (297.5bp,327.5bp) .. controls (299.0bp,323.12bp) and (301.22bp,318.8bp)  .. (310.27bp,305.48bp);
  \draw (316.25bp,335.75bp) node {7 ! ack};
  % Edge: F1dummy -> F10
  \draw [->] (418.03bp,553.23bp) .. controls (405.14bp,538.56bp) and (385.13bp,515.78bp)  .. (362.27bp,489.75bp);
  % Edge: P13 -> P14
  \draw [->] (494.0bp,273.41bp) .. controls (494.0bp,261.76bp) and (494.0bp,246.05bp)  .. (494.0bp,221.35bp);
  \draw (513.5bp,247.25bp) node {1 ? ack};
  % Edge: F04 -> F00
  \draw [->] (104.36bp,396.04bp) .. controls (92.695bp,418.48bp) and (75.05bp,461.59bp)  .. (90.5bp,494.5bp) .. controls (102.62bp,520.33bp) and (128.78bp,540.22bp)  .. (159.49bp,558.47bp);
  \draw (107.75bp,472.5bp) node {4 ? rel};
  % Edge: P24 -> P25
  \draw [->] (27.0bp,184.91bp) .. controls (27.0bp,173.26bp) and (27.0bp,157.55bp)  .. (27.0bp,132.85bp);
  \draw (43.5bp,158.75bp) node {4 ! rel};
  % Edge: F01 -> F03
  \draw [->] (208.36bp,454.32bp) .. controls (208.29bp,443.26bp) and (208.85bp,428.65bp)  .. (211.5bp,416.0bp) .. controls (212.1bp,413.16bp) and (212.89bp,410.24bp)  .. (217.7bp,396.7bp);
  \draw (230.25bp,424.25bp) node {1 ! ack};
  % Edge: F03 -> F00
  \draw [->] (236.67bp,393.95bp) .. controls (241.48bp,400.16bp) and (246.51bp,407.98bp)  .. (249.0bp,416.0bp) .. controls (263.95bp,464.24bp) and (263.45bp,485.98bp)  .. (237.0bp,529.0bp) .. controls (230.4bp,539.73bp) and (219.62bp,548.17bp)  .. (199.3bp,559.73bp);
  \draw (276.13bp,472.5bp) node {0 ? rel};
  % Edge: P14 -> P15
  \draw [->] (494.0bp,184.91bp) .. controls (494.0bp,173.26bp) and (494.0bp,157.55bp)  .. (494.0bp,132.85bp);
  \draw (510.5bp,158.75bp) node {0 ! rel};
  % Edge: P2dummy -> P20
  \draw [->] (27.0bp,643.9bp) .. controls (27.0bp,631.26bp) and (27.0bp,613.55bp)  .. (27.0bp,587.42bp);
  % Edge: P25 -> P26
  \draw [->] (27.0bp,96.051bp) .. controls (27.0bp,84.609bp) and (27.0bp,69.297bp)  .. (27.0bp,44.218bp);
  \draw (43.5bp,70.25bp) node {6 ! rel};
  % Edge: P20 -> P21
  \draw [->] (27.0bp,550.67bp) .. controls (27.0bp,537.08bp) and (27.0bp,517.89bp)  .. (27.0bp,490.72bp);
  \draw (45.0bp,520.75bp) node {4 ! req};
  % Edge: F0dummy -> F00
  \draw [->] (179.0bp,643.9bp) .. controls (179.0bp,632.34bp) and (179.0bp,616.54bp)  .. (179.0bp,591.29bp);
  % Edge: F13 -> F10
  \draw [->] (292.37bp,550.99bp) .. controls (292.04bp,539.46bp) and (293.14bp,524.25bp)  .. (299.5bp,512.5bp) .. controls (304.18bp,503.85bp) and (311.66bp,496.43bp)  .. (328.69bp,483.88bp);
  \draw (316.75bp,520.75bp) node {2 ? rel};
  % Edge: P22 -> P23
  \draw [->] (27.0bp,361.91bp) .. controls (27.0bp,350.26bp) and (27.0bp,334.55bp)  .. (27.0bp,309.85bp);
  \draw (45.0bp,335.75bp) node {6 ! req};
  % Node: P23
\begin{scope}
  \definecolor{strokecol}{rgb}{0.0,0.0,0.0};
  \pgfsetstrokecolor{strokecol}
  \draw (27.0bp,291.5bp) ellipse (18.0bp and 18.0bp);
  \draw (27.0bp,291.5bp) node {3};
\end{scope}
  % Node: P24
\begin{scope}
  \definecolor{strokecol}{rgb}{0.0,0.0,0.0};
  \pgfsetstrokecolor{strokecol}
  \draw (27.0bp,203.0bp) ellipse (18.0bp and 18.0bp);
  \draw (27.0bp,203.0bp) node {4};
\end{scope}
  % Node: F00
\begin{scope}
  \definecolor{strokecol}{rgb}{0.0,0.0,0.0};
  \pgfsetstrokecolor{strokecol}
  \draw (179.0bp,569.0bp) ellipse (18.0bp and 18.0bp);
  \draw (179.0bp,569.0bp) ellipse (22.0bp and 22.0bp);
  \draw (179.0bp,569.0bp) node {0};
\end{scope}
  % Node: F02
\begin{scope}
  \definecolor{strokecol}{rgb}{0.0,0.0,0.0};
  \pgfsetstrokecolor{strokecol}
  \draw (152.0bp,472.5bp) ellipse (18.0bp and 18.0bp);
  \draw (152.0bp,472.5bp) node {2};
\end{scope}
  % Node: F01
\begin{scope}
  \definecolor{strokecol}{rgb}{0.0,0.0,0.0};
  \pgfsetstrokecolor{strokecol}
  \draw (209.0bp,472.5bp) ellipse (18.0bp and 18.0bp);
  \draw (209.0bp,472.5bp) node {1};
\end{scope}
  % Node: F11
\begin{scope}
  \definecolor{strokecol}{rgb}{0.0,0.0,0.0};
  \pgfsetstrokecolor{strokecol}
  \draw (321.0bp,661.5bp) ellipse (18.0bp and 18.0bp);
  \draw (321.0bp,661.5bp) node {1};
\end{scope}
  % Node: F13
\begin{scope}
  \definecolor{strokecol}{rgb}{0.0,0.0,0.0};
  \pgfsetstrokecolor{strokecol}
  \draw (294.0bp,569.0bp) ellipse (18.0bp and 18.0bp);
  \draw (294.0bp,569.0bp) node {3};
\end{scope}
  % Node: F04
\begin{scope}
  \definecolor{strokecol}{rgb}{0.0,0.0,0.0};
  \pgfsetstrokecolor{strokecol}
  \draw (113.0bp,380.0bp) ellipse (18.0bp and 18.0bp);
  \draw (113.0bp,380.0bp) node {4};
\end{scope}
  % Node: P21
\begin{scope}
  \definecolor{strokecol}{rgb}{0.0,0.0,0.0};
  \pgfsetstrokecolor{strokecol}
  \draw (27.0bp,472.5bp) ellipse (18.0bp and 18.0bp);
  \draw (27.0bp,472.5bp) node {1};
\end{scope}
  % Node: P22
\begin{scope}
  \definecolor{strokecol}{rgb}{0.0,0.0,0.0};
  \pgfsetstrokecolor{strokecol}
  \draw (27.0bp,380.0bp) ellipse (18.0bp and 18.0bp);
  \draw (27.0bp,380.0bp) node {2};
\end{scope}
  % Node: P1dummy
\begin{scope}
  \definecolor{strokecol}{rgb}{0.0,0.0,0.0};
  \pgfsetstrokecolor{strokecol}
  % \draw (494.0bp,661.5bp) ellipse (27.0bp and 18.0bp);
  \draw (494.0bp,661.5bp) node {P1};
\end{scope}
  % Node: P10
\begin{scope}
  \definecolor{strokecol}{rgb}{0.0,0.0,0.0};
  \pgfsetstrokecolor{strokecol}
  \draw (494.0bp,569.0bp) ellipse (18.0bp and 18.0bp);
  \draw (494.0bp,569.0bp) node {0};
\end{scope}
  % Node: F10
\begin{scope}
  \definecolor{strokecol}{rgb}{0.0,0.0,0.0};
  \pgfsetstrokecolor{strokecol}
  \draw (348.0bp,472.5bp) ellipse (18.0bp and 18.0bp);
  \draw (348.0bp,472.5bp) ellipse (22.0bp and 22.0bp);
  \draw (348.0bp,472.5bp) node {0};
\end{scope}
  % Node: F12
\begin{scope}
  \definecolor{strokecol}{rgb}{0.0,0.0,0.0};
  \pgfsetstrokecolor{strokecol}
  \draw (295.0bp,380.0bp) ellipse (18.0bp and 18.0bp);
  \draw (295.0bp,380.0bp) node {2};
\end{scope}
  % Node: P15
\begin{scope}
  \definecolor{strokecol}{rgb}{0.0,0.0,0.0};
  \pgfsetstrokecolor{strokecol}
  \draw (494.0bp,114.5bp) ellipse (18.0bp and 18.0bp);
  \draw (494.0bp,114.5bp) node {5};
\end{scope}
  % Node: P16
\begin{scope}
  \definecolor{strokecol}{rgb}{0.0,0.0,0.0};
  \pgfsetstrokecolor{strokecol}
  \draw (494.0bp,22.0bp) ellipse (18.0bp and 18.0bp);
  \draw (494.0bp,22.0bp) ellipse (22.0bp and 22.0bp);
  \draw (494.0bp,22.0bp) node {6};
\end{scope}
  % Node: F14
\begin{scope}
  \definecolor{strokecol}{rgb}{0.0,0.0,0.0};
  \pgfsetstrokecolor{strokecol}
  \draw (322.0bp,291.5bp) ellipse (18.0bp and 18.0bp);
  \draw (322.0bp,291.5bp) node {4};
\end{scope}
  % Node: P11
\begin{scope}
  \definecolor{strokecol}{rgb}{0.0,0.0,0.0};
  \pgfsetstrokecolor{strokecol}
  \draw (494.0bp,472.5bp) ellipse (18.0bp and 18.0bp);
  \draw (494.0bp,472.5bp) node {1};
\end{scope}
  % Node: P12
\begin{scope}
  \definecolor{strokecol}{rgb}{0.0,0.0,0.0};
  \pgfsetstrokecolor{strokecol}
  \draw (494.0bp,380.0bp) ellipse (18.0bp and 18.0bp);
  \draw (494.0bp,380.0bp) node {2};
\end{scope}
  % Node: P13
\begin{scope}
  \definecolor{strokecol}{rgb}{0.0,0.0,0.0};
  \pgfsetstrokecolor{strokecol}
  \draw (494.0bp,291.5bp) ellipse (18.0bp and 18.0bp);
  \draw (494.0bp,291.5bp) node {3};
\end{scope}
  % Node: P26
\begin{scope}
  \definecolor{strokecol}{rgb}{0.0,0.0,0.0};
  \pgfsetstrokecolor{strokecol}
  \draw (27.0bp,22.0bp) ellipse (18.0bp and 18.0bp);
  \draw (27.0bp,22.0bp) ellipse (22.0bp and 22.0bp);
  \draw (27.0bp,22.0bp) node {6};
\end{scope}
  % Node: F1dummy
\begin{scope}
  \definecolor{strokecol}{rgb}{0.0,0.0,0.0};
  \pgfsetstrokecolor{strokecol}
  % \draw (431.0bp,569.0bp) ellipse (27.0bp and 18.0bp);
  \draw (431.0bp,569.0bp) node {F1};
\end{scope}
  % Node: P14
\begin{scope}
  \definecolor{strokecol}{rgb}{0.0,0.0,0.0};
  \pgfsetstrokecolor{strokecol}
  \draw (494.0bp,203.0bp) ellipse (18.0bp and 18.0bp);
  \draw (494.0bp,203.0bp) node {4};
\end{scope}
  % Node: P25
\begin{scope}
  \definecolor{strokecol}{rgb}{0.0,0.0,0.0};
  \pgfsetstrokecolor{strokecol}
  \draw (27.0bp,114.5bp) ellipse (18.0bp and 18.0bp);
  \draw (27.0bp,114.5bp) node {5};
\end{scope}
  % Node: F03
\begin{scope}
  \definecolor{strokecol}{rgb}{0.0,0.0,0.0};
  \pgfsetstrokecolor{strokecol}
  \draw (225.0bp,380.0bp) ellipse (18.0bp and 18.0bp);
  \draw (225.0bp,380.0bp) node {3};
\end{scope}
  % Node: P2dummy
\begin{scope}
  \definecolor{strokecol}{rgb}{0.0,0.0,0.0};
  \pgfsetstrokecolor{strokecol}
  % \draw (27.0bp,661.5bp) ellipse (27.0bp and 18.0bp);
  \draw (27.0bp,661.5bp) node {P2};
\end{scope}
  % Node: P20
\begin{scope}
  \definecolor{strokecol}{rgb}{0.0,0.0,0.0};
  \pgfsetstrokecolor{strokecol}
  \draw (27.0bp,569.0bp) ellipse (18.0bp and 18.0bp);
  \draw (27.0bp,569.0bp) node {0};
\end{scope}
  % Node: F0dummy
\begin{scope}
  \definecolor{strokecol}{rgb}{0.0,0.0,0.0};
  \pgfsetstrokecolor{strokecol}
  % \draw (179.0bp,661.5bp) ellipse (27.0bp and 18.0bp);
  \draw (179.0bp,661.5bp) node {F0};
\end{scope}
%
\end{tikzpicture}
    \caption{SCM automata representation of the Example~\ref{exm:philo}.}
    \label{fig:philo}
\end{figure}

\newpage

\begin{figure}[!ht]
    \centering

\begin{tikzpicture}[>=latex',line join=bevel,scale=0.72]
  \pgfsetlinewidth{1bp}
%%
\pgfsetcolor{black}
  % Edge: F02F10P10P21 -> F04F10P10P22
  \draw [->] (190.31bp,551.88bp) .. controls (170.67bp,548.11bp) and (139.47bp,539.27bp)  .. (120.97bp,519.48bp) .. controls (114.96bp,513.06bp) and (111.11bp,504.46bp)  .. (106.05bp,485.16bp);
  \definecolor{strokecol}{rgb}{0.0,0.0,0.0};
  \pgfsetstrokecolor{strokecol}
  \draw (161.09bp,511.23bp) node {F0$\to$P2:ack};
  % Edge: F02F10P10P21 -> F02F11P11P21
  \draw [->] (204.93bp,537.28bp) .. controls (203.79bp,526.78bp) and (203.95bp,513.42bp)  .. (209.47bp,502.98bp) .. controls (214.49bp,493.47bp) and (223.38bp,485.95bp)  .. (241.94bp,474.84bp);
  \draw (248.84bp,511.23bp) node {P1$\to$F1:req};
  % Edge: F00F11P11P20 -> F00F13P12P20
  \draw [->] (327.34bp,548.73bp) .. controls (342.31bp,543.07bp) and (363.97bp,533.26bp)  .. (379.22bp,519.48bp) .. controls (387.26bp,512.21bp) and (394.13bp,502.55bp)  .. (405.07bp,483.5bp);
  \draw (433.49bp,511.23bp) node {F1$\to$P1:ack};
  % Edge: F00F11P11P20 -> F02F11P11P21
  \draw [->] (304.68bp,538.08bp) .. controls (300.8bp,527.63bp) and (295.1bp,514.04bp)  .. (288.22bp,502.98bp) .. controls (285.26bp,498.22bp) and (281.67bp,493.44bp)  .. (270.63bp,480.53bp);
  \draw (336.25bp,511.23bp) node {P2$\to$F0:req};
  % Edge: F00F14P16P25 -> F00F10P16P26
  \draw [->] (304.33bp,104.67bp) .. controls (293.44bp,90.719bp) and (277.04bp,69.729bp)  .. (256.88bp,43.906bp);
  \draw (323.72bp,74.512bp) node {P2$\to$F1:rel};
  % Edge: F00F10P10P20 -> F02F10P10P21
  \draw [->] (242.27bp,637.06bp) .. controls (229.61bp,631.64bp) and (213.22bp,622.27bp)  .. (205.47bp,607.98bp) .. controls (201.63bp,600.89bp) and (200.83bp,592.42bp)  .. (203.04bp,573.23bp);
  \draw (244.84bp,599.73bp) node {P2$\to$F0:req};
  % Edge: F00F10P10P20 -> F00F11P11P20
  \draw [->] (271.19bp,630.36bp) .. controls (276.83bp,623.98bp) and (283.37bp,615.93bp)  .. (288.22bp,607.98bp) .. controls (292.88bp,600.33bp) and (297.04bp,591.52bp)  .. (304.61bp,572.84bp);
  \draw (336.18bp,599.73bp) node {P1$\to$F1:req};
  % Edge: F04F10P10P22 -> F04F11P11P22
  \draw [->] (104.36bp,448.85bp) .. controls (105.32bp,437.85bp) and (108.26bp,423.91bp)  .. (116.47bp,414.48bp) .. controls (131.82bp,396.84bp) and (157.16bp,388.03bp)  .. (188.23bp,381.53bp);
  \draw (155.84bp,422.73bp) node {P1$\to$F1:req};
  % Edge: F04F10P10P22 -> F04F12P10P23
  \draw [->] (85.982bp,465.61bp) .. controls (62.433bp,464.04bp) and (22.249bp,457.38bp)  .. (3.468bp,430.98bp) .. controls (-3.2418bp,421.54bp) and (1.3095bp,410.01bp)  .. (15.345bp,391.56bp);
  \draw (42.843bp,422.73bp) node {P2$\to$F1:req};
  % Edge: F04F14P10P24 -> F00F14P10P25
  \draw [->] (179.82bp,267.49bp) .. controls (181.43bp,255.51bp) and (183.53bp,239.92bp)  .. (186.82bp,215.43bp);
  % \draw (222.17bp,241.47bp) node {P2$\to$F0:rel};
  % Edge: F00F13P15P26 -> F00F10P16P26
  \draw [->] (189.41bp,100.63bp) .. controls (188.73bp,89.988bp) and (189.35bp,76.831bp)  .. (194.47bp,66.262bp) .. controls (198.56bp,57.814bp) and (205.27bp,50.41bp)  .. (221.45bp,37.266bp);
  \draw (232.34bp,74.512bp) node {P1$\to$F1:rel};
  % Edge: F00F13P12P20 -> F01F13P13P20
  \draw [->] (430.17bp,463.79bp) .. controls (449.25bp,460.4bp) and (478.53bp,451.91bp)  .. (492.22bp,430.98bp) .. controls (496.82bp,423.94bp) and (497.52bp,415.14bp)  .. (494.3bp,395.62bp);
  \draw (536.24bp,422.73bp) node {P1$\to$F0:req};
  % Edge: F00F13P12P20 -> F02F13P12P21
  \draw [->] (405.19bp,449.94bp) .. controls (399.68bp,438.85bp) and (391.14bp,424.38bp)  .. (380.22bp,414.48bp) .. controls (367.77bp,403.19bp) and (351.01bp,394.7bp)  .. (326.28bp,384.84bp);
  \draw (433.49bp,422.73bp) node {P2$\to$F0:req};
  % Edge: F02F11P11P21 -> F04F11P11P22
  \draw [->] (241.08bp,460.3bp) .. controls (228.04bp,454.97bp) and (211.02bp,445.6bp)  .. (202.97bp,430.98bp) .. controls (199.02bp,423.81bp) and (198.31bp,415.19bp)  .. (200.87bp,395.78bp);
  \draw (243.09bp,422.73bp) node {F0$\to$P2:ack};
  % Edge: F02F11P11P21 -> F02F13P12P21
  \draw [->] (270.19bp,453.36bp) .. controls (275.83bp,446.98bp) and (282.37bp,438.93bp)  .. (287.22bp,430.98bp) .. controls (291.88bp,423.33bp) and (296.04bp,414.52bp)  .. (303.61bp,395.84bp);
  \draw (335.93bp,422.73bp) node {F1$\to$P1:ack};
  % Edge: F04F11P11P22 -> F04F13P12P22
  \draw [->] (202.78bp,360.32bp) .. controls (201.54bp,349.84bp) and (201.58bp,336.48bp)  .. (206.97bp,325.98bp) .. controls (211.87bp,316.44bp) and (220.47bp,308.74bp)  .. (238.89bp,297.02bp);
  \draw (247.09bp,334.23bp) node {F1$\to$P1:ack};
  % Edge: start_node -> F00F10P10P20
  \draw [->] (259.22bp,699.15bp) .. controls (259.22bp,691.54bp) and (259.22bp,682.33bp)  .. (259.22bp,662.42bp);
  % Edge: F01F13P13P20 -> F03F13P14P20
  \draw [->] (475.28bp,365.89bp) .. controls (459.28bp,351.57bp) and (431.77bp,326.94bp)  .. (403.77bp,301.88bp);
  \draw (487.71bp,334.23bp) node {F0$\to$P1:ack};
  % Edge: F04F12P10P23 -> F04F14P10P24
  \draw [->] (43.602bp,368.33bp) .. controls (68.854bp,353.31bp) and (119.39bp,323.24bp)  .. (160.33bp,298.89bp);
  \draw (151.76bp,334.23bp) node {F1$\to$P2:ack};
  % Edge: F02F13P12P21 -> F04F13P12P22
  \draw [->] (303.57bp,361.14bp) .. controls (299.65bp,350.73bp) and (293.93bp,337.14bp)  .. (287.22bp,325.98bp) .. controls (284.38bp,321.26bp) and (280.99bp,316.48bp)  .. (270.47bp,303.33bp);
  \draw (335.87bp,334.23bp) node {F0$\to$P2:ack};
  % Edge: F03F13P14P20 -> F02F13P15P21
  \draw [->] (383.92bp,267.92bp) .. controls (380.49bp,255.69bp) and (375.97bp,239.56bp)  .. (369.09bp,215.0bp);
  % Edge: F00F14P10P25 -> F00F13P15P26
  \draw [->] (189.94bp,177.77bp) .. controls (190.25bp,170.21bp) and (190.62bp,161.19bp)  .. (191.42bp,141.39bp);
  % Edge: F02F13P15P21 -> F00F14P16P25
  \draw [->] (354.27bp,180.34bp) .. controls (348.18bp,170.98bp) and (340.21bp,158.74bp)  .. (326.99bp,138.43bp);
  % Node: F02F10P10P21
\begin{scope}
  \definecolor{strokecol}{rgb}{0.0,0.0,0.0};
  \pgfsetstrokecolor{strokecol}
  \draw (208.22bp,555.48bp) ellipse (18.0bp and 18.0bp);
  \draw (208.22bp,555.48bp) node {2};
\end{scope}
  % Node: F04F10P10P22
\begin{scope}
  \definecolor{strokecol}{rgb}{0.0,0.0,0.0};
  \pgfsetstrokecolor{strokecol}
  \draw (104.22bp,466.98bp) ellipse (18.0bp and 18.0bp);
  \draw (104.22bp,466.98bp) node {5};
\end{scope}
  % Node: F02F11P11P21
\begin{scope}
  \definecolor{strokecol}{rgb}{1.0,1.0,0.0};
  \pgfsetstrokecolor{strokecol}
  \draw (258.22bp,466.98bp) ellipse (18.0bp and 18.0bp);
  \definecolor{strokecol}{rgb}{0.0,0.0,0.0};
  \pgfsetstrokecolor{strokecol}
  \draw (258.22bp,466.98bp) node {4};
\end{scope}
  % Node: F00F11P11P20
\begin{scope}
  \definecolor{strokecol}{rgb}{0.0,0.0,0.0};
  \pgfsetstrokecolor{strokecol}
  \draw (310.22bp,555.48bp) ellipse (18.0bp and 18.0bp);
  \draw (310.22bp,555.48bp) node {1};
\end{scope}
  % Node: F00F13P12P20
\begin{scope}
  \definecolor{strokecol}{rgb}{0.0,0.0,0.0};
  \pgfsetstrokecolor{strokecol}
  \draw (412.22bp,466.98bp) ellipse (18.0bp and 18.0bp);
  \draw (412.22bp,466.98bp) node {3};
\end{scope}
  % Node: F00F14P16P25
\begin{scope}
  \definecolor{strokecol}{rgb}{0.0,0.0,0.0};
  \pgfsetstrokecolor{strokecol}
  \draw (316.22bp,120.89bp) ellipse (20.13bp and 20.13bp);
  \draw (316.22bp,120.89bp) node {27};
\end{scope}
  % Node: F00F10P16P26
\begin{scope}
  \definecolor{strokecol}{rgb}{0.0,0.0,0.0};
  \pgfsetstrokecolor{strokecol}
  \draw (242.22bp,24.13bp) ellipse (20.13bp and 20.13bp);
  \draw (242.22bp,24.13bp) ellipse (24.13bp and 24.13bp);
  \draw (242.22bp,24.131bp) node {29};
\end{scope}
  % Node: F00F10P10P20
\begin{scope}
  \definecolor{strokecol}{rgb}{0.0,0.0,0.0};
  \pgfsetstrokecolor{strokecol}
  \draw (259.22bp,643.98bp) ellipse (18.0bp and 18.0bp);
  \draw (259.22bp,643.98bp) node {0};
\end{scope}
  % Node: F04F11P11P22
\begin{scope}
  \definecolor{strokecol}{rgb}{1.0,1.0,0.0};
  \pgfsetstrokecolor{strokecol}
  \draw (206.22bp,378.48bp) ellipse (18.0bp and 18.0bp);
  \definecolor{strokecol}{rgb}{0.0,0.0,0.0};
  \pgfsetstrokecolor{strokecol}
  \draw (206.22bp,378.48bp) node {8};
\end{scope}
  % Node: F04F12P10P23
\begin{scope}
  \definecolor{strokecol}{rgb}{0.0,0.0,0.0};
  \pgfsetstrokecolor{strokecol}
  \draw (28.22bp,378.48bp) ellipse (18.0bp and 18.0bp);
  \draw (28.218bp,378.48bp) node {9};
\end{scope}
  % Node: F04F14P10P24
\begin{scope}
  \definecolor{strokecol}{rgb}{0.0,0.0,0.0};
  \pgfsetstrokecolor{strokecol}
  \draw (177.22bp,287.85bp) ellipse (20.13bp and 20.13bp);
  \draw (177.22bp,287.85bp) node {12};
\end{scope}
  % Node: F00F14P10P25
\begin{scope}
  \definecolor{strokecol}{rgb}{0.0,0.0,0.0};
  \pgfsetstrokecolor{strokecol}
  \draw (189.22bp,196.62bp) ellipse (18.6bp and 18.6bp);
  \draw (189.22bp,196.62bp) node {...};
\end{scope}
  % Node: F00F13P15P26
\begin{scope}
  \definecolor{strokecol}{rgb}{0.0,0.0,0.0};
  \pgfsetstrokecolor{strokecol}
  \draw (192.22bp,120.89bp) ellipse (20.13bp and 20.13bp);
  \draw (192.22bp,120.89bp) node {28};
\end{scope}
  % Node: F01F13P13P20
\begin{scope}
  \definecolor{strokecol}{rgb}{0.0,0.0,0.0};
  \pgfsetstrokecolor{strokecol}
  \draw (488.22bp,378.48bp) ellipse (18.0bp and 18.0bp);
  \draw (488.22bp,378.48bp) node {6};
\end{scope}
  % Node: F02F13P12P21
\begin{scope}
  \definecolor{strokecol}{rgb}{1.0,1.0,0.0};
  \pgfsetstrokecolor{strokecol}
  \draw (309.22bp,378.48bp) ellipse (18.0bp and 18.0bp);
  \definecolor{strokecol}{rgb}{0.0,0.0,0.0};
  \pgfsetstrokecolor{strokecol}
  \draw (309.22bp,378.48bp) node {7};
\end{scope}
  % Node: F04F13P12P22
\begin{scope}
  \definecolor{strokecol}{rgb}{1.0,0.0,0.0};
  \pgfsetstrokecolor{strokecol}
  \draw (257.22bp,287.85bp) ellipse (20.13bp and 20.13bp);
  \definecolor{strokecol}{rgb}{0.0,0.0,0.0};
  \pgfsetstrokecolor{strokecol}
  \draw (257.22bp,287.85bp) node {11};
\end{scope}
  % Node: start_node
\begin{scope}
  \definecolor{strokecol}{rgb}{0.0,0.0,0.0};
  \pgfsetstrokecolor{strokecol}
  \draw (259.22bp,716.98bp) node {start};
\end{scope}
  % Node: F03F13P14P20
\begin{scope}
  \definecolor{strokecol}{rgb}{0.0,0.0,0.0};
  \pgfsetstrokecolor{strokecol}
  \draw (389.22bp,287.85bp) ellipse (20.13bp and 20.13bp);
  \draw (389.22bp,287.85bp) node {10};
\end{scope}
  % Node: F02F13P15P21
\begin{scope}
  \definecolor{strokecol}{rgb}{0.0,0.0,0.0};
  \pgfsetstrokecolor{strokecol}
  \draw (364.22bp,196.62bp) ellipse (18.6bp and 18.6bp);
  \draw (364.22bp,196.62bp) node {...};
\end{scope}
%
\end{tikzpicture}

    \caption{Synchronous Product of the Example~\ref{exm:philo}.}
    \label{fig:philo-sync}
\end{figure}

\end{example}

\subsection{Example with a loop}

\begin{example}\label{exm:loop}
Now consider two processes $A$ and $B$ that exchange data. At some
point, each makes a nondeterministic choice: one branch continues
sending messages indefinitely, while the other leads to termination.
Once the choice to continue is taken, however, there is no way to
return to the terminating branch. As a result, the system may remain
stuck in an infinite loop, never reaching a final state. Although both
processes remain active, the system is effectively deadlocked.

\bigskip

\begin{lstlisting}[language={},caption={Output of Example~\ref{exm:loop}.},
    label={lst:loop}]
This system is RSC.
The system has the progress property.
There are some deadlock states:
Deadlock: Id=17 Configuration={{ A:1; B:4 }}
Deadlock: Id=15 Configuration={{ A:3; B:3 }}
\end{lstlisting}

The behaviour of this system is shown in Figure~\ref{fig:loop}. 
Executing the tool produces the output in
Listing~\ref{lst:loop} and the synchronous system in
Figure~\ref{fig:loop-sync}. In the
generated figure, yellow states highlight the deadlocked executions,
while the terminal output provides the configuration of each detected
deadlock.

\textbf{Remark.} 
If the system contains a loop with at least one possible way out, this 
execution is still considered without a deadlock thanks to the 
\textbf{fairness} assumption. 
Fairness ensures that the exit path will eventually be taken.


\begin{figure}[!ht]
    \centering
\begin{tikzpicture}[>=latex',line join=bevel,scale=1.05]
  \pgfsetlinewidth{1bp}
%%
\pgfsetcolor{black}
  % Edge: B3 -> B4
  \draw [->] (144.22bp,96.57bp) .. controls (135.94bp,85.51bp) and (126.9bp,69.041bp)  .. (131.56bp,54.0bp) .. controls (132.92bp,49.613bp) and (135.04bp,45.286bp)  .. (143.89bp,31.954bp);
  \definecolor{strokecol}{rgb}{0.0,0.0,0.0};
  \pgfsetstrokecolor{strokecol}
  \draw (145.06bp,62.25bp) node {1 ? v};
  % Edge: B4 -> B3
  \draw [->] (157.32bp,36.071bp) .. controls (157.82bp,41.766bp) and (158.31bp,48.151bp)  .. (158.56bp,54.0bp) .. controls (158.94bp,62.881bp) and (158.66bp,72.558bp)  .. (157.33bp,92.502bp);
  \draw (172.23bp,62.25bp) node {1 ? v};
  % Edge: Bdummy -> B0
  \draw [->] (186.56bp,346.67bp) .. controls (186.56bp,339.06bp) and (186.56bp,329.85bp)  .. (186.56bp,309.95bp);
  % Edge: A3 -> A1
  \draw [->] (24.844bp,127.99bp) .. controls (29.004bp,140.98bp) and (34.892bp,159.35bp)  .. (43.239bp,185.4bp);
  \draw (49.406bp,158.75bp) node {1 ! v};
  % Edge: B0 -> B1
  \draw [->] (186.56bp,273.41bp) .. controls (186.56bp,261.76bp) and (186.56bp,246.05bp)  .. (186.56bp,221.35bp);
  \draw (200.06bp,247.25bp) node {0 ? v};
  % Edge: A1 -> A2
  \draw [->] (56.87bp,186.65bp) .. controls (63.703bp,174.03bp) and (73.577bp,155.79bp)  .. (87.333bp,130.39bp);
  \draw (88.742bp,158.75bp) node {0 ! v};
  % Edge: A1 -> A3
  \draw [->] (32.341bp,194.41bp) .. controls (21.881bp,188.55bp) and (9.1337bp,179.34bp)  .. (3.0583bp,167.0bp) .. controls (-1.7681bp,157.2bp) and (0.47788bp,145.72bp)  .. (9.7171bp,125.67bp);
  \draw (15.808bp,158.75bp) node {1 ! v};
  % Edge: A0 -> A1
  \draw [->] (48.558bp,273.41bp) .. controls (48.558bp,261.76bp) and (48.558bp,246.05bp)  .. (48.558bp,221.35bp);
  \draw (61.308bp,247.25bp) node {0 ! v};
  % Edge: B1 -> B3
  \draw [->] (176.88bp,187.53bp) .. controls (173.18bp,181.42bp) and (169.2bp,174.08bp)  .. (166.56bp,167.0bp) .. controls (163.37bp,158.46bp) and (161.05bp,148.82bp)  .. (157.5bp,128.65bp);
  \draw (180.06bp,158.75bp) node {1 ? v};
  % Edge: B1 -> B2
  \draw [->] (191.51bp,185.4bp) .. controls (195.04bp,173.56bp) and (199.9bp,157.28bp)  .. (207.41bp,132.12bp);
  \draw (215.17bp,158.75bp) node {0 ? v};
  % Edge: Adummy -> A0
  \draw [->] (48.558bp,346.67bp) .. controls (48.558bp,339.06bp) and (48.558bp,329.85bp)  .. (48.558bp,309.95bp);
  % Node: A2
\begin{scope}
  \definecolor{strokecol}{rgb}{0.0,0.0,0.0};
  \pgfsetstrokecolor{strokecol}
  \draw (97.56bp,110.5bp) ellipse (18.0bp and 18.0bp);
  \draw (97.56bp,110.5bp) ellipse (22.0bp and 22.0bp);
  \draw (97.558bp,110.5bp) node {2};
\end{scope}
  % Node: B3
\begin{scope}
  \definecolor{strokecol}{rgb}{0.0,0.0,0.0};
  \pgfsetstrokecolor{strokecol}
  \draw (155.56bp,110.5bp) ellipse (18.0bp and 18.0bp);
  \draw (155.56bp,110.5bp) node {3};
\end{scope}
  % Node: B4
\begin{scope}
  \definecolor{strokecol}{rgb}{0.0,0.0,0.0};
  \pgfsetstrokecolor{strokecol}
  \draw (155.56bp,18.0bp) ellipse (18.0bp and 18.0bp);
  \draw (155.56bp,18.0bp) node {4};
\end{scope}
  % Node: Bdummy
\begin{scope}
  \definecolor{strokecol}{rgb}{0.0,0.0,0.0};
  \pgfsetstrokecolor{strokecol}
  % \draw (186.56bp,364.5bp) ellipse (27.0bp and 18.0bp);
  \draw (186.56bp,364.5bp) node {B};
\end{scope}
  % Node: B0
\begin{scope}
  \definecolor{strokecol}{rgb}{0.0,0.0,0.0};
  \pgfsetstrokecolor{strokecol}
  \draw (186.56bp,291.5bp) ellipse (18.0bp and 18.0bp);
  \draw (186.56bp,291.5bp) node {0};
\end{scope}
  % Node: A3
\begin{scope}
  \definecolor{strokecol}{rgb}{0.0,0.0,0.0};
  \pgfsetstrokecolor{strokecol}
  \draw (19.56bp,110.5bp) ellipse (18.0bp and 18.0bp);
  \draw (19.558bp,110.5bp) node {3};
\end{scope}
  % Node: A1
\begin{scope}
  \definecolor{strokecol}{rgb}{0.0,0.0,0.0};
  \pgfsetstrokecolor{strokecol}
  \draw (48.56bp,203.0bp) ellipse (18.0bp and 18.0bp);
  \draw (48.558bp,203.0bp) node {1};
\end{scope}
  % Node: B1
\begin{scope}
  \definecolor{strokecol}{rgb}{0.0,0.0,0.0};
  \pgfsetstrokecolor{strokecol}
  \draw (186.56bp,203.0bp) ellipse (18.0bp and 18.0bp);
  \draw (186.56bp,203.0bp) node {1};
\end{scope}
  % Node: B2
\begin{scope}
  \definecolor{strokecol}{rgb}{0.0,0.0,0.0};
  \pgfsetstrokecolor{strokecol}
  \draw (213.56bp,110.5bp) ellipse (18.0bp and 18.0bp);
  \draw (213.56bp,110.5bp) ellipse (22.0bp and 22.0bp);
  \draw (213.56bp,110.5bp) node {2};
\end{scope}
  % Node: A0
\begin{scope}
  \definecolor{strokecol}{rgb}{0.0,0.0,0.0};
  \pgfsetstrokecolor{strokecol}
  \draw (48.56bp,291.5bp) ellipse (18.0bp and 18.0bp);
  \draw (48.558bp,291.5bp) node {0};
\end{scope}
  % Node: Adummy
\begin{scope}
  \definecolor{strokecol}{rgb}{0.0,0.0,0.0};
  \pgfsetstrokecolor{strokecol}
  % \draw (48.56bp,364.5bp) ellipse (27.0bp and 18.0bp);
  \draw (48.558bp,364.5bp) node {A};
\end{scope}
%
\end{tikzpicture}
    \caption{SCM automata representation of the Example~\ref{exm:loop}.}
    \label{fig:loop}
\end{figure}


\begin{figure}[!ht]
    \centering

\begin{tikzpicture}[>=latex',line join=bevel,scale=0.9]
  \pgfsetlinewidth{1bp}
%%
\pgfsetcolor{black}
  % Edge: A3B3 -> A1B4
  \draw [->] (27.548bp,105.03bp) .. controls (18.113bp,97.729bp) and (7.1099bp,87.181bp)  .. (1.8895bp,74.762bp) .. controls (-3.8439bp,61.122bp) and (6.0577bp,48.005bp)  .. (26.676bp,31.31bp);
  \definecolor{strokecol}{rgb}{0.0,0.0,0.0};
  \pgfsetstrokecolor{strokecol}
  \draw (33.39bp,66.512bp) node {A$\to$B:v};
  % Edge: A0B0 -> A1B1
  \draw [->] (86.89bp,281.93bp) .. controls (86.89bp,270.28bp) and (86.89bp,254.57bp)  .. (86.89bp,229.87bp);
  \draw (118.39bp,255.77bp) node {A$\to$B:v};
  % Edge: start_node -> A0B0
  \draw [->] (86.89bp,355.2bp) .. controls (86.89bp,347.59bp) and (86.89bp,338.37bp)  .. (86.89bp,318.47bp);
  % Edge: A1B1 -> A3B3
  \draw [->] (70.499bp,202.98bp) .. controls (59.926bp,197.13bp) and (47.04bp,187.93bp)  .. (40.89bp,175.52bp) .. controls (36.728bp,167.13bp) and (36.008bp,157.18bp)  .. (38.379bp,136.75bp);
  \draw (72.39bp,167.27bp) node {A$\to$B:v};
  % Edge: A1B1 -> A2B2
  \draw [->] (96.475bp,195.86bp) .. controls (100.28bp,189.71bp) and (104.54bp,182.41bp)  .. (107.89bp,175.52bp) .. controls (111.69bp,167.71bp) and (115.29bp,159.06bp)  .. (122.42bp,140.17bp);
  \draw (146.28bp,167.27bp) node {A$\to$B:v};
  % Edge: A1B4 -> A3B3
  \draw [->] (55.235bp,36.979bp) .. controls (59.096bp,43.289bp) and (62.929bp,50.814bp)  .. (64.89bp,58.262bp) .. controls (67.676bp,68.85bp) and (64.837bp,80.229bp)  .. (55.255bp,99.999bp);
  \draw (97.489bp,66.512bp) node {A$\to$B:v};
  % Node: A3B3
\begin{scope}
  \definecolor{strokecol}{rgb}{1.0,1.0,0.0};
  \pgfsetstrokecolor{strokecol}
  \draw (43.89bp,116.89bp) ellipse (20.13bp and 20.13bp);
  \definecolor{strokecol}{rgb}{0.0,0.0,0.0};
  \pgfsetstrokecolor{strokecol}
  \draw (43.89bp,116.89bp) node {2};
\end{scope}
  % Node: A1B4
\begin{scope}
  \definecolor{strokecol}{rgb}{1.0,1.0,0.0};
  \pgfsetstrokecolor{strokecol}
  \draw (43.89bp,20.13bp) ellipse (20.13bp and 20.13bp);
  \definecolor{strokecol}{rgb}{0.0,0.0,0.0};
  \pgfsetstrokecolor{strokecol}
  \draw (43.89bp,20.131bp) node {4};
\end{scope}
  % Node: A2B2
\begin{scope}
  \definecolor{strokecol}{rgb}{0.0,0.0,0.0};
  \pgfsetstrokecolor{strokecol}
  \draw (129.89bp,116.89bp) ellipse (20.13bp and 20.13bp);
  \draw (129.89bp,116.89bp) ellipse (24.13bp and 24.13bp);
  \draw (129.89bp,116.89bp) node {3};
\end{scope}
  % Node: A0B0
\begin{scope}
  \definecolor{strokecol}{rgb}{0.0,0.0,0.0};
  \pgfsetstrokecolor{strokecol}
  \draw (86.89bp,300.02bp) ellipse (18.0bp and 18.0bp);
  \draw (86.89bp,300.02bp) node {0};
\end{scope}
  % Node: A1B1
\begin{scope}
  \definecolor{strokecol}{rgb}{0.0,0.0,0.0};
  \pgfsetstrokecolor{strokecol}
  \draw (86.89bp,211.52bp) ellipse (18.0bp and 18.0bp);
  \draw (86.89bp,211.52bp) node {1};
\end{scope}
  % Node: start_node
\begin{scope}
  \definecolor{strokecol}{rgb}{0.0,0.0,0.0};
  \pgfsetstrokecolor{strokecol}
  \draw (86.89bp,373.02bp) node {start};
\end{scope}
%
\end{tikzpicture}

    \caption{Synchronous Product of the Loop Example~\ref{exm:loop}}
    \label{fig:loop-sync}
\end{figure}


\end{example}
