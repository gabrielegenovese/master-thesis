\chapter{Weak-Realisability is Undecidable for Synch Global Types}\label{sec:proof}

The first contribution is Theorem~\ref{thm:main}, 
which establishes that \emph{Weak-realisability is undecidable for 
synchronous global types}. To prepare for this result, we have introduced 
in Chapter~\ref{sec:pre} the basic notions of MSCs, Global Types, and 
Weak-realisability.  
We now present the main objects used in the proof of 
Theorem~\ref{thm:main}, which we adapt from 
Alur et~al.~\cite{alur2005realizability}, and we highlight along the way 
the key differences with the original construction.

\section{Definitions}
The proof is a \emph{reduction} from the
\textbf{Relaxed Post Correspondence Problem (RPCP)}, a variant of
the classical Post Correspondence Problem (PCP). 
RPCP was shown to be undecidable by
Alur~et~al.~\cite{alur2005realizability}, via reduction from PCP.
The main idea is to encode the existence of a
solution to an RPCP instance into the (non-)realisability of our formal
specification. In the original proof, MSCs are directly used to build an
HMSC called $M^*$. In our case, we will define a
\emph{global type} (called $L^*$) built from synchronous global types.
A generic solution for the RPCP problem will correspond to the global
type $L^*$. Therefore, we need to prove:
$$
\Delta \in \text{RPCP} \quad\iff\quad L^* \text{ is not realisable}.
$$

\bigskip

\begin{definition}[Relaxed Post Correspondence Problem]
	Given a set of tiles $\{(v_1, w_1), (v_2, w_2), ..., (v_r, w_r)\}$, 
	determining whether there exist indices $i_1, ..., i_m$ such that
	$$x_{i_1}\cdots x_{i_m} = y_{i_1}\cdots y_{i_m},$$
	where $x_{i_j}, y_{i_j} \in \{v_{i_j}, w_{i_j}\}$, such that:
	\begin{itemize}
		\item there exists at least one index $i_\ell$ for which $x_{i_\ell}\neq y_{i_\ell}$, and
		\item for all $j \leq m$, $y_{i_1}\cdots y_{i_j}$ is a strict or 
		not-strict prefix of $x_{i_1}\cdots x_{i_j}$.
	\end{itemize}
\end{definition}

Intuitively, RPCP requires that the concatenation on the left-hand side always 
grows at least as fast as the right-hand side, while ensuring that at least one 
chosen tile differs between the two sequences. Moreover, in constructing the 
strings, we may freely choose which element of each tile (either $v_i$ or $w_i$) 
contributes to the left or right sequence.

\bigskip

\begin{example}[Simple RPCP instance]\label{exmp:rpcp}
Consider the tile set
\[
(v_1,w_1)=(\texttt{b},\ \texttt{bb}),\quad
(v_2,w_2)=(\texttt{a},\ \texttt{ab}),\quad
(v_3,w_3)=(\texttt{c},\ \texttt{c}).
\]
Take the index sequence $(2,1,3)$ and the choices
\[
x_1 = w_2,\ y_1 = v_2;\quad
x_2 = v_1,\ y_2 = w_1;\quad
x_3 = v_3,\ y_3 = w_3.
\]
Then
\[
x_1 x_2 x_3 = \texttt{ab}\ \texttt{b}\ \texttt{c} = \texttt{abbc},
\qquad
y_1 y_2 y_3 = \texttt{a}\ \texttt{bb}\ \texttt{c} = \texttt{abbc},
\]
so the two sides are equal.

We now check the RPCP conditions:
\begin{itemize}
  \item \textbf{at least one mismatch:} here $x_1\neq y_1$ and
        $x_2\neq y_2$, so the ``some index differs'' condition holds;
  \item \textbf{prefix property:} for every prefix length $j$ we have
        $y_{1}\cdots y_{j}$ is a prefix of $x_{1}\cdots x_{j}$:
        \begin{itemize}
          \item $j=1$: $y_1=\texttt{a}$ is a prefix of $x_1=\texttt{ab}$;
          \item $j=2$: $y_1y_2=\texttt{abb}$ is a prefix of $x_1x_2=\texttt{abb}$;
          \item $j=3$: $y_1y_2y_3=\texttt{abbc}$ is a prefix of $x_1x_2x_3=\texttt{abbc}$.
        \end{itemize}
\end{itemize}
\end{example}

\bigskip

We have now identified the main problem to which our proof reduces.  
The next step is to encode an RPCP instance into the formal model.
In the original proof, MSCs are used, but, in our case, we need to give
an encoding using Global Types. We will give both definition.

\bigskip

\begin{definition}[$M^n_i$]\label{def:mni}
	Given the index $i$ of a tile $(v_i, w_i)$, and
	given an interger $n\in\{0,1\}$, where:
	\begin{itemize}
		\item if $n=0$, then $x_i=v_i$;
		\item if $n=1$, then $x_i=w_i$;
	\end{itemize}
	The behavior of the MSC $M^n_i$ is as follows:
	first, Process~1 synchronously sends message
	$m_1 = (i, n)$ to Process~2, then Process~1 transmits the index $m_2=i$
	to Process~4. Subsequently, Process~4 sends $m_3 = (i, n)$
	synchronously to Process~3. After these control messages, Process~2
	sends the characters $m_i^1 = x_i^1,..., m_i^c = x_i^c$
	synchronously to Process~3 (where $c$ is the length of $x_i$).
	This MSC is depicted in Figure~\ref{fig:mni}, 

	\begin{figure}[!ht]
		\centering
		\begin{msc}[draw frame=none, draw head=none, msc keyword=, head height=0px, label distance=0.5ex, foot height=0px, foot distance=0px]{}
			\declinst{P1}{P1}{}
			\declinst{P2}{P2}{}
			\declinst{P3}{P3}{}
			\declinst{P4}{P4}{}

			\syncmscmess{$(i,n)$}{P1}{P2}
			\syncmscmess{$i$}{P1}{P4}
			\syncmscmess{$(i,n)$}{P4}{P3}
			\syncmscmess{$x_i^1$}{P2}{P3}
			\syncmscmess{...}{P2}{P3}
			\syncmscmess{$x_i^c$}{P2}{P3}
		\end{msc}
		\caption{The $M_i^n$ MSC.}
		\label{fig:mni}
	\end{figure}

\end{definition}

Given a RPCP instance $\{(v_1,w_1),\ldots,(v_m,w_m)\}$, we associate  
with each pair $(v_i,w_i)$ two MSCs $M^0_i$ and $M^1_i$, following  
Definition~\ref{def:mni}. Each MSC $M^n_i$ is \emph{synchronous}  
(Lemma~\ref{lemma:minsynch}). Intuitively, the MSC $M_i^n$ encodes the  
construction of a string given some tiles through the interaction of four processes.  
Processes~2 and~3 are responsible for building the string itself,  
while Processes~1 and~4 transmit the index information to Processes~2  
and~3, respectively. In particular, Process~1 initiates the choice and  
forwards it to Process~4. This encoding applies equally to definition~\ref{def:gni}.

\bigskip

\begin{lemma}\label{lemma:minsynch}
	The MSC $M_i^n$ belongs to $\mscsetofmodel{\synchmodel}$.
\end{lemma}


\begin{proof}
	By Definition~\ref{def:linearisable-msc} and
	Definition~\ref{def:synchronous},
	we need to show a linearization with all send operations
	followed by their corresponding receive operations:
	$$
	\{~!m_1?m_1\ !m_2?m_2\ !m_3?m_3\ !m_i^1?m_i^1 \ldots !m_i^c?m_i^c~\}.
	$$
	Such a linearization exists by construction, hence $M_i^n$ is synchronous.
\end{proof}

We now give the definition of the encoding in a Global Type format.

\bigskip

\begin{definition}[$G_i^n$]\label{def:gni}
	Given a tile $(v_i,w_i)$ and a bit $n\in\{0,1\}$, define
    $x_i = v_i$ if $n=0$, and $x_i = w_i$ if $n=1$. The global
    type $G_i^n$ is composed of:
    \begin{itemize}
        \item $\mathbb{P}=\{p,q,r,s\}$;
        \item $\mathbb{M}=\{m_1,m_2,m_3,m_{x_i^1},\ldots,m_{x_i^c}\}$,
              where $m_1=(i,n),\ m_2=i,\ m_3=(i,n),$ and
              $m_{x_i^j}=x_i^j$ for $1\leq j\leq c$, with
              $c=|x_i|$;
        \item Arr $=\{$\arrmess{p}{q}{m_1}, \arrmess{p}{s}{m_2},
              \arrmess{s}{r}{m_3}, \arrmess{q}{r}{m_{x_i^1}}, $\ldots$,
              \arrmess{q}{r}{m_{x_i^c}}$\}$, where each arrow denotes
              a synchronous message with acknowledgment. % TODO : CHIEDERE CINZIA : avendo semantica syncrona ho bisogno dei messaggi di ack?
    \end{itemize}
    The automaton of $G_i^n$ is shown in Figure~\ref{fig:gni}.
	
	\begin{figure}[!ht]
		\centering
		\begin{tikzpicture}[->, node distance=35mm, on grid, auto]
			\node[state] (q0) {$q_0$};
			\node[state] (q1) [right=of q0] {$q_1$};
			\node[state] (q2) [right=of q1] {$q_2$};
			\node[state] (q3) [below left=of q0] {$q_3$};
			\node[state] (q4) [right=of q3] {$q_4$};
			\node[state] (q5) [right=of q4] {$\cdots$};
			\node[state,accepting] (q6) [right=of q5] {$q_x$};

			\path (q0) edge[] node[above] {\arrmess{p}{q}{(i,n)}} (q1);
			\path (q1) edge[] node[above] {\arrmess{p}{s}{i}} (q2);
			\path (q2) edge[] node[above left] {\arrmess{s}{r}{(i,n)}} (q3.60);
			\path (q3) edge[] node[above] {\arrmess{q}{r}{S_1}} (q4);
			\path (q4) edge[] node[above] {\arrmess{q}{r}{...}} (q5);
			\path (q5) edge[] node[above] {\arrmess{q}{r}{S_c}} (q6);
		\end{tikzpicture}
		\caption{The global type $G_i^n$.}
		\label{fig:gni}
	\end{figure}

\end{definition}

Intuitively, $G_i^n$ specifies the same communication pattern as the MSC  
$M^n_i$ introduced in Definition~\ref{def:mni}. This structural  
correspondence will be made precise in the next lemma (Lemma~\ref{lmm:msgs}).  

Before establishing the connection between MSCs and Global Types,
we briefly summarize the rationale behind the design of $M_i^n$ and $G_i^n$.

Suppose that $\Delta=(i_1,a_1,b_1,\ldots,i_m,a_m,b_m)$ is a  
solution to the RPCP instance. From this solution we construct   
two MSCs sequences:
\[
M_x = M^{a_1}_{i_1}\cdots M^{a_m}_{i_m}, \qquad  
M_y = M^{b_1}_{i_1}\cdots M^{b_m}_{i_m}.
\]
Both $M_x$ and $M_y$ are synchronous concatenations of synchronous  
MSCs. We then define a third MSC $M_{\texttt{sol}}$, obtained by  
projecting $M_y$ onto processes $P1,P2$ and $M_x$ onto processes  
$P3,P4$. Intuitively, processes $P1,P2$ represent the construction of  
the \emph{right-hand string} $y_{i_1}\cdots y_{i_m}$, while processes  
$P3,P4$ represent the construction of the \emph{left-hand string}  
$x_{i_1}\cdots x_{i_m}$. The prefix property of RPCP guarantees that  
$M_{\texttt{sol}}$ is acyclic and \emph{synchronous}. Establishing the  
synchrony of $M_{\texttt{sol}}$ is non-trivial, and this step is an  
addition to the original proof. By construction, $L^*$ weakly implies  
$M_{\texttt{sol}}$, but $M_{\texttt{sol}} \notin L^*$, since at least  
one tile differs. Consequently, $L^*$ is \emph{not realisable}.
  
With these constructions in place, we proceed to introduce the main  
objects used in the proof. Specifically, we first show how a Global  
Type can represent a single \verb|synch| MSC.

\bigskip

\begin{lemma}[$G_M$]\label{lem:gm}
Given a synchronous MSC $M\in \mscsetofmodel{\synchmodel}$, there
exists a global type $\gt_M$ such that $M \in
\existentialmsclanguageof{\gt_M}$.
\end{lemma}

\begin{proof}
Since $M\in\mscsetofmodel{\synchmodel}$, there is a
\emph{synchronous linearisation} $w$ of $M$ in which every send
immediately precedes its matched receive. Let
$w = \alpha_1\alpha_2\ldots\alpha_k$ where each $\alpha_j$ is a
synchronous communication step of the form $!m_j ?m_j$.
$\text{send}(\alpha_k)$ and $\text{recv}(\alpha_k)$ denotes respectively
the send and receive process of the communication $\alpha_k$.
We construct the global type $\gt_M$ as a finite-state automaton
that accepts exactly a language containing the linearisation $w$.
$\gt_M$ is the sequence automaton that performs the
interactions $\alpha_1,\alpha_2,\dots,\alpha_k$ in order:
for each $j\in\{1,\dots,k\}$, add a word to the alphabet $\labelalphabet$
$\text{send}(\alpha_j)\xrightarrow{m_j}\text{recv}(\alpha_j)$ labelled by
the synchronous interaction corresponding to $\alpha_j$.
By construction the only global executions (under synchronous
semantics) generated by $\gt_M$ are linearisations that follow the
sequence $w$; hence $\mscof{w}=M$ is one of the MSCs in the
existential MSC-language of $\gt_M$. Therefore,
$M\in\existentialmsclanguageof{\gt_M}$.
\end{proof}


Lemma~\ref{lem:gm} establishes a direct correspondence between a  
single synchronous MSC and a Global Type. In particular, every  
synchronous MSC can be captured precisely by a Global Type whose  
language contains that MSC. This correspondence will be useful  
when embedding RPCP instances into the Global Type framework.
We now introduce a more structured Global Type, parameterized by a  
string $S$, which will serve as the building block in the reduction.

\bigskip

\begin{lemma}\label{lmm:msgs}
Assume $\acommunicationmodel$ is the $\synchmodel$ model and $i,n$ are integers.
The MSC $M^n_i$ (Definition~\ref{def:mni}) is included in 
$\msclanguageof{G_i^n}{\synchmodel}$ (Definition~\ref{def:gni}).
\end{lemma}

% TODO: Da rifare meglio, non ha senso quel L(M) = L(G)
\begin{proof}
	Both $M^n_i$ and $G_i^n$ describe the same communication structure:
	process $p$ sends $(i,n)$ to $q$ and $i$ to $s$;
	process $s$ relays $(i,n)$ to $r$;
	process $q$ then sends the characters of $x_i$ to $r$.
	The sequence of messages is identical in both 
	$M^n_i$ and $G_i^n$. Since both models enforce synchronous communication, 
	their linearisations coincide. 
	Hence, $M^n_i \in \msclanguageof{G_i^n}{\synchmodel}$.
\end{proof}

Having established the correspondence between an individual MSC and its
associated global type, we now extend this construction to sets of global
types. The following definition introduces the global type $L^*$, which
encapsulates all possible compositions derived from a given RPCP instance.
Intuitively, for each MSC $M \in \setmsc$, there exists a corresponding
global type $G \in G^*$ that captures the behaviour described by $M$.
The automaton defining $L^*_N$ then combines all such global types in $G^*$
into a single structure, allowing transitions between them through
$\varepsilon$-moves. The determinisation of this automaton yields the
global type $L^*$, representing the full set of possible interactions
generated by the collection of MSCs.

\bigskip

\begin{definition}[The $L^*$ global type]\label{def:lstar}
	Given an instance $\{(v_1, w_1), \ldots, (v_m, w_m)\}$ of RPCP, we
	construct a set $G^* = \{G_i^0, G_i^1 \mid i \in \{1, \ldots, m\}\}$ of
	global types over four processes as follows. For each pair $(v_i, w_i)$,
	we define two global types, $G_i^0$ and $G_i^1$, as specified in
	Definition~\ref{def:gni} and illustrated in Figure~\ref{fig:gni}.
	We define the global type $L^*_{N}$ as the automaton
	$\mathcal A = (Q,\Sigma, \delta, l_0, F)$ where:
	\begin{itemize}
		\item $Q = \{v_I,v_T\}\cup \bigcup_{G\in G^*} Q^G$;
		\item $\Sigma = \{\epsilon\}\cup\bigcup_{G\in G^*} \Sigma^G$;
		\item $\delta: Q \times \Sigma \rightarrow 2^Q$ is defined by:
			      \begin{enumerate}
				       \item $\forall G \in G^*,\ \delta(v_I, \varepsilon) = q_0^G$ where $q_0^G$ is the initial state of $G$,
				       \item $\forall G \in G^*,\ \forall q_f^G \in F^G,\ \delta(q_f^G, \varepsilon) = v_T$,
				       \item $\forall G, G' \in G^*,\ \forall q_f^G \in F^G,\ \delta(q_f^G, \varepsilon) = q_0^{G'}$.
			      \end{enumerate}
		\item $l_0 = v_I$ is the initial state;
		\item $F = v_T$ is the accepting state.
	\end{itemize}
	The automaton of $L^*_{N}$ is shown in Figure~\ref{fig:lstar}.  
	Finally, $L^*$ is obtained as the determinisation of $L^*_{N}$.
\end{definition}

\begin{figure}[!ht]
	\centering
	\begin{tikzpicture}[->, node distance=35mm, on grid, auto]
		\node[state] (vI) {$v_I$};
		\node[state] (qI2) [right=of vI] {$\cdots$};
		\node[state] (qI1) [above=of qI2] {$q_0^{G^1}$};
		\node[state] (qI3) [below=of qI2] {$q_0^{G^n}$};
		\node[state] (qM1) [right=of qI1] {$\cdots$};
		\node[state] (qM2) [right=of qI2] {$\cdots$};
		\node[state] (qM3) [right=of qI3] {$\cdots$};
		\node[state] (qF1) [right=of qM1] {$q_f^{G^1}$};
		\node[state] (qF2) [right=of qM2] {$\cdots$};
		\node[state] (qF3) [right=of qM3] {$q_f^{G^n}$};
		\node[state,accepting] (vT) [right=of qF2] {$v_T$};

		\path (vI) edge[] node[above] {$\epsilon$} (qI1);
		\path (vI) edge[] node[above] {$\epsilon$} (qI2);
		\path (vI) edge[] node[above] {$\epsilon$} (qI3);
		\path (qI1) edge[] node[above] {\arrmess{p}{q}{(i^{G^1},n^{G^1})}} (qM1);
		\path (qI2) edge[] node[above] {} (qM2);
		\path (qI3) edge[] node[above] {\arrmess{p}{q}{(i^{G^n},n^{G^n})}} (qM3);
		\path (qM1) edge[] node[above] {\arrmess{q}{r}{x^{G^1}_c}} (qF1);
		\path (qM2) edge[] node[above] {} (qF2);
		\path (qM3) edge[] node[above] {\arrmess{q}{r}{x^{G^n}_c}} (qF3);
		\path (qF1) edge[] node[above] {$\epsilon$} (vT);
		\path (qF2) edge[] node[above] {$\epsilon$} (vT);
		\path (qF3) edge[] node[above] {$\epsilon$} (vT);
		
		\draw (qF1.135) to [bend right=30] node[above] {$\epsilon$} (qI1.45);
		\draw (qF2.135) to [bend right=30] node[above] {$\epsilon$} (qI2.45);
		\draw (qF3.135) to [bend right=30] node[above] {$\epsilon$} (qI3.45);

		\draw (qF1.225) to node[above] {$\epsilon$} (qI2.60);
		\draw (qF3.120) to node[above] {$\epsilon$} (qI2.315);
		
		\draw (qF3) .. controls +(8,10) and +(1,3) .. node[midway,above] {$\epsilon$} (qI1);
		\draw (qF1) ..  controls +(8,-10) and +(1,-3) .. node[midway,above] {$\epsilon$} (qI3);
	\end{tikzpicture}
	\caption{The automaton of the global type $L^*_N$.}
	\label{fig:lstar}
\end{figure}

In other words, $L^*$ denotes the set of all possible executions  
arising from a family of global types that comes from a generic solution 
to the RPCP problem. This construction provides the   
structure used to demonstrate non-realisability.

\section{Undecidability proof}

Given the definitions and lemmas stated in the last section, we are now ready
to present the proof for the undecidability result.

\bigskip

\begin{theorem}\label{thm:main}
	Given a global type $G$, checking if $G$ is weakly-realisable is undecidable.
\end{theorem}

\begin{proof}
	The proof proceeds via a reduction from the RPCP problem.
	Given an instance $\{(v_1, w_1), \ldots, (v_m, w_m)\}$ of RPCP, we
	construct $L^*$ as specified in Definition~\ref{def:lstar}.
	Observe that each component of $L^*$ is strongly connected and involves
	all four processes. Therefore, the global type represented by $L^*$,
	derived from the collection of component global types, is bounded.
	We need to prove:
	\begin{center}
		$\Delta \in \text{RPCP}$ iff the global type $L^*$ is not weakly-realisable.
	\end{center}

	\begin{itemize}
		\item[$\Rightarrow$]
		      Assume that
		      $\Delta = (i_1, a_1, b_1, i_2, a_2, b_2, \ldots, i_m, a_m, b_m)$ are the indices
		      for a solution to a generic RPCP problem instance, and the bits $a_j$ and
		      $b_j$ indicate which string ($v_{i_j}$ or $w_{i_j}$) is chosen to go into
		      the two (left and right) long strings. Assume also synchronous communication semantic.
			  Consider the MSCs $M_x$ and $M_y$ obtained from the concatenation of
		      $M_x = M^{a_1}_{i_1} \cdots M^{a_m}_{i_m}$ 
			  and $M_y = M^{b_1}_{i_1} \cdots M^{b_m}_{i_m}$.
		      The language of the executions of both of these (sequences of) MSCs 
			  must be included in the language of execution of $L^*$. 
			  Additionally, the language generated by these MSCs are 
			  in $\mscsetofmodel{\synchmodel}$ because they are sequences of MSCs included in
			  $\mscsetofmodel{\synchmodel}$ (Lemma~\ref{lemma:minsynch}).
		      $M_x$ corresponds to the construction of the left side of the equivalence of the RPCP
		      problem, and, instead, $M_y$ represents the construction of the right side.
		      We then look at the projections $M_x|_{P1}$, $M_x|_{P2}$, $M_x|_{P3}$,
		      and $M_x|_{P4}$ of $M_x$, and $M_y|_{P1}$, $M_y|_{P2}$, $M_y|_{P3}$, $M_y|_{P4}$ of $M_y$ onto the
		      4 processes. Now consider the MSC $M_{\texttt{sol}}$ 
			  formed from $M_y|_{P1}$, $M_y|_{P2}$, $M_x|_{P3}$, and $M_x|_{P4}$.
		      This MSC represents the construction of the solution to
		      the problem. Processes 1 and 2 construct the right part ($y_{i_1}...y_{i_m}$)
		      and processes 3 and 4 construct the left part ($x_{i_1}...x_{i_m}$).
		      The claim is that the combined MSC $M_{\texttt{sol}}$ is 
		      implied by $L^*$, but it is not part of its language. 
			  In other words, the language of the execution of $M_{\texttt{sol}}$ is included
			  in the execution of the system, but it is not included in the execution of $L^*$. 
			  By definition, the only thing to establish is that $M_{\texttt{sol}}$
		      is indeed an MSC, in the sense that it is well-formed, and synchronous.
		      The only new situation in terms of communication in $M_{\texttt{sol}}$ is the
		      communication between $P_1$ and $P_4$, and between $P_2$ and $P_3$.
		      But the communication between $P_1$ and $P_4$ is consistent in
		      $M_y|_{P1}$ and $M_x|_{P4}$ (i.e., the sequence of messages sent from $P_1$ to
		      $P_4$ in $M_y|_{P1}$ is equal to the sequence of messages received in $M_x|_{P4}$),
		      and the communication between $P_2$ and $P_3$ is consistent in
		      $M_y|_{P2}$ and $M_x|_{P3}$ because $R$ is a solution to the RPCP.
		      Furthermore, the acyclicity of $M_{\texttt{sol}}$ follows from the property of the
		      solution that the string formed by the first $j$ words on processes 1
		      and 2 is always a prefix of the string formed by the first $j$ words
		      on processes 3 and 4. Consequently, each message from $P_1$ to $P_4$
		      is sent before it needs to be received.

		      Finally, we prove that 
			  $M_{\texttt{sol}} \in \mscsetofmodel{\synchmodel}$.
		      Assume, for contradiction, that 
			  $M_{\texttt{sol}} \notin \mscsetofmodel{\synchmodel}$.
		      Then, there should be a cycle of dependencies in the communication pattern.
		      There are no communication between $P_2$ and $P_4$, and between $P_1$
		      and $P_3$. Therefore, this cycle must involve all processes, starting
		      for example from $P_1$ and having this dependency graph
		      $P_1\leftrightarrow P_2\leftrightarrow P_3\leftrightarrow P_4\leftrightarrow P_1$.
		      The only new situation that can cause a cycle are the communication
		      between $P_1$ and $P_4$, and between $P_2$ and $P_3$.
		      We don't need to analyse the new communication between $P_1$ and $P_4$ because
		      it's not feasible in any communication model, but we need to analyse the one
		      between $P_2$ and $P_3$ because it's feasible in FIFO.

		      % For the fist comunication, the only possible cycle pattern is depicted
		      % in Fig.~\ref{fig:cycle1}

		      % \begin{figure}[!ht]
		      %  \centering
		      %  \begin{msc}[draw frame=none, draw head=none, msc keyword=, head height=0px, label distance=0.5ex, foot height=0px, foot distance=0px]{}
		      %   \declinst{P1}{P1}{}
		      %   \declinst{P2}{P2}{}
		      %   \declinst{P3}{P3}{}
		      %   \declinst{P4}{P4}{}

		      %   \mess[label position=above right,pos=0.45]{$i_z$}{P1}{P4}[8]
		      %   \nextlevel
		      %   \nextlevel
		      %   \nextlevel
		      %   \syncmscmess{($i_k,n_k)$}{P1}{P2}
		      %   \mess[label position=above,pos=0.62]{$i_k$}{P1}{P4}
		      %   \mess{}{P4}{P1}
		      %   \nextlevel
		      %   \syncmscmess{$(i_k,n_j)$}{P3}{P4}
		      %   \nextlevel
		      %   \nextlevel
		      %   \mess{}{P4}{P1}[-8]
		      %  \end{msc}
		      %  \caption{The $M_i^n$ MSC.}
		      %  \label{fig:cycle1}
		      % \end{figure}

		      % This cycle is not possible because it does not represent a
		      % solution to the RPCP problem:
		      % $x_1...x_{i_k}...x_{i_z}...x_m \neq y_1...y_{i_z}...y_{i_k}...y_m$.

		      \begin{figure}[!ht]
			      \centering
			      \begin{msc}[draw frame=none, draw head=none, msc keyword=, head height=0px, label distance=0.5ex, foot height=0px, foot distance=0px]{}
				      \declinst{P1}{P1}{}
				      \declinst{P2}{P2}{}
				      \declinst{P3}{P3}{}
				      \declinst{P4}{P4}{}

				      \mess[label position=above right, pos=0.3]{$c$}{P2}{P3}[4]%
				      \nextlevel
				      \syncmscmess{$(i_k,n_k)$}{P1}{P2}
				      \mess[pos=0.62]{$i_k$}{P1}{P4}%
				      \mess{}{P4}{P1}
				      \nextlevel
				      \syncmscmess{$(i_k,n_j)$}{P3}{P4}
				      \mess{}{P3}{P2}[-4]
			      \end{msc}
			      \caption{MSC communication that breaks synchrony.} % todo: modifica
			      \label{fig:cycle2}
		      \end{figure}

		      For the communication between $P_2$ and $P_3$, the only possible cycle
		      pattern is depicted in Figure~\ref{fig:cycle2} showed as an MSC.
		      Suppose $P_2$ wants to send a character $c$, but $P_3$
		      is not expecting any further characters. In order for
		      $P_3$ to resume receiving, it must first receive an index
		      from $P_4$. However, $P_4$ can only send this index
		      after receiving it from $P_1$, which in turn must first
		      communicate the index to $P_2$.
		      At this point, $P_2$ needs to receive the index from
		      $P_1$, but it cannot do so until it finishes sending
		      character $c$. This creates a circular dependency among the
		      processes, making the communication pattern impossible. % TODO: chiarire che non è come un caso di deadlock
		      This cycle would break the prefix property as
		      $x_1...x_{k-1}...x_m= y_1...y_{k-1}...y_m$, but the character $c$ appears
		      in $y_1...y_{k-1}$ but not in $x_1...y_{k-1}$ contradicting the
		      assumption that $y_1...y_{k-1} \leq x_1...x_{k-1}$.
		      Therefore, we conclude that 
			  $M_{\texttt{sol}} \in \mscsetofmodel{\synchmodel}$.

			  Assume that the system of CFSM $\cfsms_G$ is the system of $G_{\texttt{sol}}$,
			  where $G_{\texttt{sol}}$ is constructed based on $M_{\texttt{sol}}$, as 
			  described in Lemma~\ref{lem:gm}. We need to prove that 	
			  $\executionsof{\cfsms_G}{\synchmodel} \neq \executionsof{L^*}{\synchmodel}$.
		      Note that $\executionsof{\cfsms_G}{\synchmodel}$ cannot 
			  itself be in $\executionsof{L^*}{\synchmodel}$ because there must be
		      some index $i_j$ where $a_j \neq b_j$, and no execution of the Global 
			  Type exists in $L^*$ where,
		      after $P_1$ announces the index, what $P_2$ sends is not
		      identical to what $P_3$ receives.

		\item[$\Leftarrow$]
		      Suppose there is some MSC $M^@$ which
		      exists in any realisation of $L^*$, but is not part of $L^*$'s languague
			  of MSCs. We want to derive a solution to $\Delta$ from $M^@$.
		      First, it is clear that the projection $M^@|_{P1}$ must consist of a sequence
		      of pairs of messages (the first of each pair acknowledged), sent from
		      process 1 to processes 2 and 4, respectively, with messages $(i, b)$ and $i$.
		      Likewise, it is clear that, in order for process 2 to receive those messages,
		      $M^@|_{P2}$ must consist of a sequence of receipts of $(i, b)$ pairs, and after
		      each $(i, b)$, either $v_i$ or $w_i$ is sent to process 3, based on whether
		      $b = 0$ or $b = 1$, before the next index pair is received.
		      Likewise, $M^@|_{P4}$ consists of a sequence of receipts of index $i$ from
		      process 1, followed by sending of $(i, 0)$ or $(i, 1)$ to process 3, and
		      $M^@|_{P3}$ consists of a sequence of receipt of $(i, 0)$ or $(i, 1)$ followed
		      by receipt of $v_i$ or $w_i$, respectively.
		      Now, since $M^@$ is not in $L^*$, for some index $i$ the choice of $v_i$ or
		      $w_i$ must differ on process 2 and process 3. (Note, we are assuming that
		      the buffers between processes are FIFO.)
		      Furthermore, because of the precedences, the prefix formed by the first
		      $j$ words on process 2 must precede the $(j + 1)$-th message from
		      process 1 to process 4, which in turn precedes the $(j + 1)$-th message
		      from 4 to 3, and hence the $(j + 1)$-th word on process 3. That is, the
		      string formed by the first $j$ words on process 2 is a prefix of the string
		      formed by the first $j$ words on process 3. Therefore, we can readily
		      build a solution for $\Delta$ from $M^@$ by building the strings of the solution
		      taking the projections of $P_1$ and $P_4$. In fact, $P_1$ builds 
			  $y_{i_1}\cdots y_{i_m},$ and $P_4$ builds $x_{i_1}\cdots x_{i_m}$.

	\end{itemize}

\end{proof}

In this example, we will show the step-by-step construction 
of $M_{\texttt{sol}}$ from Theorem~\ref{thm:main}.

\bigskip

% TODO: dare un esempietto per far vedere che la prova regge
% Fatto... Andrà bene?
\begin{example}[$M_{\texttt{sol}}$ Example of Theorem~\ref{thm:main}]
Consider the tiles and the solution of the RPCP instance 
in Example~\ref{exmp:rpcp}, with the tile set and the solution
with index sequence $(2,1,3)$
$$
 (v_1,w_1)=(\texttt{b},\texttt{bb}),\ 
 (v_2,w_2)=(\texttt{a},\texttt{ab}),\ 
 (v_3,w_3)=(\texttt{c},\texttt{c}).
$$
$$
 x_1=w_2,\ y_1=v_2;\quad x_2=v_1,\ y_2=w_1;\quad x_3=v_3,\ y_3=w_3
$$
This sequence is a solution because
$x_1x_2x_3=\texttt{ab}\,\texttt{b}\,\texttt{c}=\texttt{abbc}$ and
$y_1y_2y_3=\texttt{a}\,\texttt{bb}\,\texttt{c}=\texttt{abbc}$. The
prefix property and the ``some index differs'' condition are satisfied.

Therefore, the encoding of the solution is
$$\Delta = (i_1=2,a_1=1,b_1=0,i_2=1,a_2=0,b_2=1,i_3=3,a_3=0,b_3=1)$$

Recall that for each tile index \(i\) we have two synchronous MSCs
\(M_i^0\) and \(M_i^1\) (see Definition~\ref{def:mni}), where the bit
indicates choosing \(v_i\) (0) or \(w_i\) (1) for the character stream.
Using the concrete index sequence \((2,1,3)\) we form two
concatenated MSCs:
\[
  M_x\;=\; M^{1}_{2}\ \cdot\ M^{0}_{1}\ \cdot\ M^{0}_{3},
\qquad
  M_y\;=\; M^{0}_{2}\ \cdot\ M^{1}_{1}\ \cdot\ M^{1}_{3}.
\]
Here \(M_x\) encodes the \(\mathbf{x}\)-concatenation
\((x_1,x_2,x_3)=(w_2,v_1,v_3)\) (depicted 
in Figure~\ref{fig:exmp-mx}) and \(M_y\) encodes 
the \(\mathbf{y}\)-concatenation (depicted in 
Figure~\ref{fig:exmp-my}) \((y_1,y_2,y_3)=(v_2,w_1,w_3)\).

\begin{figure}[!ht]
\centering
\begin{msc}[draw frame=none, draw head=none, msc keyword=, head height=0px, label distance=0.5ex, foot height=0px, foot distance=0px]{}
	\declinst{P1}{P1}{}
	\declinst{P2}{P2}{}
	\declinst{P3}{P3}{}
	\declinst{P4}{P4}{}

	\syncmscmess{$(2,1)$}{P1}{P2}
	\syncmscmess{$2$}{P1}{P4}
	\syncmscmess{$(2,1)$}{P4}{P3}
	\syncmscmess{$a$}{P2}{P3}
	\syncmscmess{$b$}{P2}{P3}

	\syncmscmess{$(1,0)$}{P1}{P2}
	\syncmscmess{$1$}{P1}{P4}
	\syncmscmess{$(1,0)$}{P4}{P3}
	\syncmscmess{$b$}{P2}{P3}

	\syncmscmess{$(3,0)$}{P1}{P2}
	\syncmscmess{$3$}{P1}{P4}
	\syncmscmess{$(3,0)$}{P4}{P3}
	\syncmscmess{$c$}{P2}{P3}
\end{msc}
\caption{The MSC $M_x$.}
\label{fig:exmp-mx}
\end{figure}

\begin{figure}[!ht]
\centering
\begin{msc}[draw frame=none, draw head=none, msc keyword=, head height=0px, label distance=0.5ex, foot height=0px, foot distance=0px]{}
	\declinst{P1}{P1}{}
	\declinst{P2}{P2}{}
	\declinst{P3}{P3}{}
	\declinst{P4}{P4}{}

	\syncmscmess{$(2,0)$}{P1}{P2}
	\syncmscmess{$2$}{P1}{P4}
	\syncmscmess{$(2,0)$}{P4}{P3}
	\syncmscmess{$a$}{P2}{P3}

	\syncmscmess{$(1,1)$}{P1}{P2}
	\syncmscmess{$1$}{P1}{P4}
	\syncmscmess{$(1,1)$}{P4}{P3}
	\syncmscmess{$b$}{P2}{P3}
	\syncmscmess{$b$}{P2}{P3}

	\syncmscmess{$(3,1)$}{P1}{P2}
	\syncmscmess{$3$}{P1}{P4}
	\syncmscmess{$(3,1)$}{P4}{P3}
	\syncmscmess{$c$}{P2}{P3}
\end{msc}
\caption{The MSC $M_y$.}
\label{fig:exmp-my}
\end{figure}

Recall that $G|_p$ denotes the projection of $G$ onto process $p$. We
construct the MSC
$$
  M_{\texttt{sol}} = (M_y|_{P1},\; M_y|_{P2},\; M_x|_{P3},\; M_x|_{P4}),
$$
i.e.\ processes $1,2$ follow $M_y$ while $3,4$ follow $M_x$.
Intuitively, $M_{\texttt{sol}}$ pairs the right-side construction (from $M_y$)
with the left-side construction (from $M_x$). 
Figure~\ref{fig:exmp-msol} illustrates the behaviour of the MSC
$M_{\texttt{sol}}$. Observe that when process~3 expects to receive the
second character $\texttt{b}$ right after $a$, 
but process~2 cannot send it immediately:
it must first obtain the corresponding index and bit from process~1.
The prefix property guarantees
that every partial construction of the right-hand side is aligned with
a prefix of the left-hand side, therefore preserving synchronous
semantics throughout the execution.

\begin{figure}[!ht]
\centering
\begin{msc}[draw frame=none, draw head=none, msc keyword=, head height=0px, label distance=0.5ex, foot height=0px, foot distance=0px]{}
	\declinst{P1}{}{$M_y|_{P1}$}
	\declinst{P2}{}{$M_y|_{P2}$}
	\declinst{P3}{}{$M_x|_{P3}$}
	\declinst{P4}{}{$M_x|_{P4}$}

	\syncmscmess{$(2,1)$}{P1}{P2}
	\syncmscmess{$2$}{P1}{P4}
	\syncmscmess{$(2,0)$}{P4}{P3}
	\syncmscmess{$a$}{P2}{P3}
	\mess{}{P3}{P2}[4]
	\nextlevel

	\syncmscmess{$(1,0)$}{P1}{P2}
	\mess[pos=0.4]{$1$}{P1}{P4}
	\mess{}{P4}{P1}
	\nextlevel
	\syncmscmess{$(1,1)$}{P4}{P3}
	\mess[pos=0.2]{$b$}{P2}{P3}[-4]
	\nextlevel
	\syncmscmess{$b$}{P2}{P3}
	
	\syncmscmess{$(3,0)$}{P1}{P2}
	\syncmscmess{$3$}{P1}{P4}
	\syncmscmess{$(3,1)$}{P4}{P3}
	\syncmscmess{$c$}{P2}{P3}
\end{msc}
\caption{The MSC $M_\texttt{sol}$.}
\label{fig:exmp-msol}
\end{figure}

\end{example}

The sequence of lemmas and the main theorem collectively establish the
undecidability of weak-realisability for global types. Having developed the
theoretical foundation, we now move to the next section, where we focus on the
practical aspects of analysing realisability, and introduce the \textsc{ReSCu} tool.